\documentclass[10pt]{article}

\usepackage[
a4paper,landscape,
total={284mm,170mm},
bottom=7mm,
top=14mm,headsep=3mm]{geometry}
\usepackage[utf8]{inputenc}
\usepackage[T1]{fontenc}
\usepackage{textcomp}
\usepackage{amsmath, amssymb, amsthm}
% \usepackage[outputdir=tmp]{minted}
\usepackage{lmodern}
\usepackage{multicol}
\setlength{\columnsep}{4mm}
\usepackage{fancyhdr}
\usepackage{lastpage}
\pagestyle{fancy}

% figure support
\usepackage{import}
\usepackage{xifthen}
\pdfminorversion=7
\usepackage{pdfpages}
\usepackage{transparent}
\newcommand{\incfig}[1]{%
  \def\svgwidth{\columnwidth}
  \import{./figures/}{#1.pdf_tex}
}

\pdfsuppresswarningpagegroup=1

\newtheorem{thm}{Theorem}[section]
\newtheorem{crl}{Corollary}[thm]
\newtheorem{lemma}{Lemma}[thm]
\newtheorem{note}{Note}[thm]
\newtheorem{defn}{Definition}[section]
\newtheorem{ex}{Example}[section]
\newtheorem{prop}{Proposition}[section]
\newtheorem{obs}{Observation}
\newtheorem{claim}{Claim}

\newcommand{\pmat}[1]{ \begin{pmatrix}#1\end{pmatrix} }
\newcommand{\seqn}[1]{(#1)^\infty_{n=1}}
\newcommand{\seqk}[1]{(#1)^\infty_{k=1}}
% (series term): returns a series with counter n=1 to \infty.
\newcommand{\infsrsn}[1]{\sum\limits^\infty_{n=1}#1}
\newcommand{\infsrsk}[1]{\sum\limits^\infty_{k=1}#1}
\newcommand{\real}{\mathbb{R}}
\newcommand{\nat}{\mathbb{N}}
\newcommand{\rat}{\mathbb{Q}}
\newcommand{\cmm}{C(M_1,M_2)}
\newcommand{\met}[1]{\langle M_{#1},\rho_{#1}\rangle}
\newcommand{\ntoinf}{\limits_{n\to\infty}}
\newcommand{\ktoinf}{\limits_{k\to\infty}}
% \newcommand{\onetoinf}[]{^\infty_{n=1}}
\newcommand{\limn}[1]{\lim\ntoinf #1}
\newcommand{\limk}[1]{\lim\ktoinf #1}
\newcommand{\ltri}{\triangleleft}
\newcommand{\module}{MA2202S }

\DeclareMathOperator{\spn}{span}
\DeclareMathOperator{\diam}{diam}
\DeclareMathOperator{\aut}{Aut}
\DeclareMathOperator{\inn}{Inn}

\fancyhf{}
\rhead{Page \thepage/\pageref{LastPage}}
\lhead{Made by Tan Yee Jian (\texttt{@swampertx})}
\chead{\module Cheat Sheet}

\title{\module Cheatsheet}
\author{Tan Yee Jian}
\date{\today}

% philosophy: group results based on topic

\begin{document}
\begin{center}
  \Large{\textbf{\module Cheat Sheet}}
\end{center}
\begin{multicols*}{3}
  \section{Normal Subgroups}
  \begin{enumerate}
    \item $N\ltri G$ is equivalent to for all $g\in G$:
      \begin{enumerate}
          \item $gng^{-1}\in N$ for all $n\in N$.
        \item $gNg^{-1}\subseteq N$
        \item $gNg^{-1}=N$
        \item $gN=Ng$
      \end{enumerate}
      \item $N\ltri G, K\leq G$. Then $N\cap K\ltri K$.
      \item $K\ltri H\leq G$. Then if $N\ltri G$, $NK\ltri NH$.
  \end{enumerate}

  \section{Cyclic Groups}
  \begin{enumerate} \item $|G|=n$ is cyclic is iff:
      \begin{enumerate}
        \item There is a \textbf{unique} subgroup of order $d$ for every
          positive divisor $d$ of $n$.
          \item $\gcd(\varphi(n),n)=1$.
      \end{enumerate}
    \item $|G|=n$ is cyclic if:
    \begin{enumerate}
      \item $|G|=pq$ and $p>q, p\not\equiv 1(\mod q)$.
    \end{enumerate}
    \item $\aut(C_{n})\cong C_{\phi(n)}$
  \end{enumerate}

  \section{Homomorphism}
  \begin{enumerate}
    \item The kernel of a homomorphism is a normal subgroup.
  \end{enumerate}

  \section{Simple Groups}
  \begin{enumerate}
    \item For $n\geq5,A_{n}$is simple.
    \item If $|G|=n, G$ simple, then
    \begin{enumerate}
      \item $|G|\mid k!$ for any subgroup of index $k$. Furthermore, if not
      $k=|G|=2$, then $|G|\mid k!/2$.
    \end{enumerate}
  \end{enumerate}

  \section{Abelian Groups}
  \begin{enumerate}
    \item A group $G$ is Abelian iff
    \begin{enumerate}
      \item $G/Z(G)$ is cyclic.
      \item All Sylow subgrops are normal and Abelian.
    \end{enumerate}
  \end{enumerate}

  \section{X-groups, X-Composition series}
  \begin{enumerate}
    \item $G$ is a \textbf{$X$-group} if $X$ acts on $G$ such that
    $x\cdot(g_{1}g_{2})=(x\cdot g_{1})(x\cdot g_{2})$. \begin{enumerate}
      \item $H\leq G$ is a \textbf{$X$-subgroup} if $x\cdot H\subseteq H$.
      \item $H\ltri G$ is \textbf{$X$-normal} if $x\cdot H\subseteq H$.
      \item $H\leq G$ is \textbf{$X$-simple} if it has no $X$-subgroups. $H$ might
      not be simple.
    \end{enumerate}
    \item If $X=\inn(X)$, then \begin{enumerate}
      \item $H\ltri G \iff H\ltri_{X}G$
      \item $H$ simple in $G\iff H$ is $X$-simple in $G$
      \item $G/N$ simple $\iff G/N$ is $X$-simple.
    \end{enumerate}
    \item $G$ has composition series $\iff$ $G/H$ and $H$ has composition series.
    \item (Zassenhaus' Lemma) $A_{1}\ltri A_{2}$,$B_{1}\ltri B_{2}$ are
    $X$-subgroups of G, then \[\frac{A_{1}(A_{2}\cap B_{2})}{A_{1}(A_{2}\cap B_{1})}\cong\frac{B_{1}(A_{2}\cap B_{2})}{B_{1}(A_{1}\cap B_{2})} \]
    \item (Dedekind Modular Law) $AB\cap C=A(B\cap C)$.
    \item $G=H\times K, H, K$ simple. How many composition/chief series?
    \begin{enumerate}
      \item If $K$ is non-Abelian, then ${1}\ltri H\ltri G$ and
      ${1}\ltri K\ltri G$ are the only chief/composition series.
      \item If both are Abelian ($\implies C_{p}$): if $C_{p}\times C_{q}$ then same
      as above, otherwise $C_{p}\times C_{p}$ then  there are $p+1$ normal subgroups.
    \end{enumerate}
    \item Isomorphic $\implies$ isomorphic X-composition series. Converse is false.
  \end{enumerate}

  \section{Nilpotent and Soluble Groups}
  \begin{enumerate}
    \item $[H,N]\leq N$.
    \item $H,K\ltri G\implies [H,K]\ltri G$ and $[H,K]\leq H\cap K$.
    \item Nilpotent group iff:\begin{enumerate}
      \item Any maximal subgroup must be normal.
      \item Cannot have self-normalizing subgroups.
      \item All Sylow p-subgroups are normal.
      \item Is a direct product of all its Sylow subgroups.
      \item $\gamma_{n}={1}$, where $\gamma_{n}=[\gamma_{n-1},G]$.
      \item $\zeta_{n}=G$, where $\zeta_{i}/\zeta_{i-1}=Z(G/\zeta_{i-1})$. (pre-image of center
      of quotient group)
    \end{enumerate}
  \end{enumerate}

  \section{p-groups}
  \begin{enumerate}
    \item For each divisor, there is a normal subgroup of that order.
    \item Every chief and composition factor are $\cong C_{p}$, composition length = power.
    \item Every chief series is a composition series but converse is false.
  \end{enumerate}
\end{multicols*}
\end{document}
