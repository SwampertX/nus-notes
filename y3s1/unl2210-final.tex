\documentclass[a4paper]{article}
\usepackage[a4paper, total={140mm, 210mm}]{geometry}

\usepackage[english]{babel}
\usepackage{hyperref}
\usepackage[utf8]{inputenc}
\usepackage{amsmath}
\usepackage{graphicx}
\usepackage{pdfpages}
\usepackage{listings}
\usepackage{graphicx}
\usepackage[citestyle=numeric,bibstyle=apa, hyperref=true,backref=true,maxcitenames=3,url=true,backend=biber,natbib=true] {biblatex}
\addbibresource{unl2210.bib}

\renewcommand{\baselinestretch}{1.35} % line spacing
\setlength{\parindent}{4em} % paragraph indentation
\setlength{\parskip}{1em} % inter-paragraph space

\newcommand{\0}{\{\}}
\newcommand{\1}{\{\0\}}
\newcommand{\2}{\{\0,\1\}}
\newcommand{\3}{\{\0,\1,\2\}}
\newcommand{\4}{\{\0,\1,\2,\3\}}
\newcommand{\5}{\{\0,\1,\2,\3,\4\}}

\title{The (fairly) Reasonable Effectiveness of Mathematics -\\ a Mathematical Viewpoint}
\author{Tan Yee Jian}
\date{\today}

\begin{document}
\maketitle
\begin{abstract}
  Mathematics has been found an immensely useful tool in science, and
  subsequently a irreplaceable method to understand our nature through the lens
  of science. This “unreasonable” effectiveness \cite{wigner} has been the topic
  of discussion for a long time. I propose a few conditions which account for
  the effectiveness of mathematics - suitable axioms that arose from nature,
  human’s innate ability to do logical deduction, and the availability of
  suitable mathematical notation. This precise notation, coupled with the
  ability of humans to abstract and compose mathematical structures with one
  another, has allowed new knowledge to be discovered quickly.

  \medskip We also look at the Mathematical Universe Hypothesis (MUH)'s claim of
  the universe is mathematical\cite{muh}, and why we do not understand it yet is
  due to the baggage such as metaphors, constructed structures that obscure the
  underlying mathematical nature.

\end{abstract}

What makes something effective? It implies a tool - problem relationship: if
using a tool makes the problem significantly easier to solve, then we say the
tool is effective. Just like a hammer is much more effective in driving a nail into
wood than say, a knife, we could either conclude how the structure of a hammer,
its precisely weighted head which swings and creates a huge moment that makes

What makes something effective? The evaluation of effectiveness implies a
problem-solution relationship: if using a particular solution or a tool makes
the given problem significantly easier to solve, then we say the tool is
effective. For example, a hammer is much more effective in driving a nail into
wood than say, a knife. However, what makes the hammer so effective? In this
case, the hammer is precisely designed to drive nails, therefore it is as
effective as we can possibly get.

But this is not the case for mathematics as a tool to understand reality -
mathematics was never designed to explain or mimic nature. Mathematicians do not
limit their conceptions to the real world or the observable reality; whether
abstract structures such as topological spaces or symmetric groups have a mirror
image in the real world is not a concern for Mathematics. In spite of that,
mathematics is surprisingly effective in understanding the real world. It is
almost as if we pick up a tool that was never designed for driving nails, but it
happens to do the job perfectly, or in the words of Wigner, “unreasonably
effective”\cite{wigner}. He gave an analogy for this situation: we are given a
bunch of keys and many doors, which are not related at first glance. However as
every door opens effortlessly after a few tries, we can’t help but to think that
the keys might be designed perfectly for the doors.

In this essay, I will be discussing why mathematics is an effective tool for
humans to discover truths in nature, from three points: one, the mathematical
language is precise, concise and universal and welcomes contribution and
development in mathematics from around the world; two, mathematics acts as a
collection of results of deductive logic which science and other fields can use
easily and save a lot of time; three, mathematics has most of its axioms rooted
in our perceived reality, therefore producing results that can explain our
reality well. We will also look at the Mathematical Universe Hypothesis (MUH)
raised by Max Tegmark, which hypothesizes that the underlying universe is
mathematical\cite{muh}. The reason why we do not perceive our reality as mathematical is
due to the “baggage” imposed on mathematical equations, which are abstract
concepts and constructs acting as shortcuts and intuitions. For example, a cup
in reality is a container that is meant to hold drinks. However, the idea of a
“cup” is a baggage which obscures its underlying mathematical nature. Throwing
away the abstraction, a cup might be just plastic, molded in a specific shape to
allow the containment of liquid, or more particularly a specific series of
chemical elements bonded together in an intricate way to create its texture. One
can keep stripping off “baggage” like these in our reality and finally,
hypotheses Tegmark, reach an eternal, irreducible mathematical reality. We will
take a look at MUH, and argue about its possibility how it induces a paradox in
scientific or mathematical development.

First of all, mathematics is so successful despite its complexity, is partly due
to the excellent qualities of the mathematical language. It is a concise and
precise way to record the results of deductive logic, which is what mathematics
is built upon. In general, languages that encode information can run into either
of these two extremes: one, being extremely clear but terse, such as how
computers store information in strings of $0$ and $1$s, or two, being as
ambiguous as natural language, where there are many ways to encode information,
which are up to interpretation when decoding. The formal is ideal for storage
and transfer, however it sacrifices the readability and intuition of the
information. Should mathematics fall within this category, it would be difficult
for mathematicians to decipher information from the mathematical language, much
less to internalize the language into meaningful structures and discover new
theorems on existing ones. On the other hand, natural language is ambiguous at
its core, and cannot reliably transfer information, despite being easier to
understand. Mathematics described in natural language will likely cause more
misunderstanding and impede the development of it. Fortunately, mathematics as
formulated today emerges as a precise language, although restricted, but still
remains understandable to the trained individuals. This is an important
condition for mathematics to develop rapidly and reach advanced enough results
of purely deductive logic, which is far more than what science and other fields
can leverage at the moment.

Second, while science relies on inductive logic, it aims towards theories that
are consistently true, thus it tends to rely heavily on mathematical models and
constructs. Oftentimes, scientific theories need a complex interaction of logic
between its laws, how different objects in the system interact with each other,
which is difficult to grasp. However, mathematics comes to help as it provides
readily available results in deductive logic which describes interactions
between highly abstract mathematical objects, and can be suitably specialized to
explain results in nature while promising coherence due to the a priori nature
of mathematics. This means that science can save up a lot of time needed to
build a model from scratch that meets the circumstances on the topic at hand.
For example, the invention of complex numbers, which has seemingly no
counterpart in the real world (unlike natural numbers as counting objects, or
real numbers as results of measurements), is the bread and butter in quantum
mechanics - you find it in Schrodinger’s equations, Heisenberg’s commutation
relation and more.

\[i \hbar \frac{\partial}{\partial t}\Psi(\mathbf{r},t) = \hat H \Psi(\mathbf{r},t)\]
\begin{center}
Schrodinger's equation.
\end{center}
% TODO Schrodinger and Heisenberg

How do we account for this unexpected effectiveness of complex numbers in
Quantum Mechanics? If we think of mathematics theories as intermediate results
of deductive logic, then this effectiveness can be attributed to the fact that
complex numbers stored just the appropriate results in deductive logic in an
abstract form, that fits the need of Quantum Mechanics theories. The whole
theory of complex numbers is a crystallization of deductive logic, encapsulated
in this man-made, mathematical concept called complex numbers. Although they do
not correspond directly to reality as much as constructs such as natural numbers
do, since they are first devised as solutions for equations like $x^{2}+1=0$, which
is impossible in real life. This means the theory of complex numbers is more of
an exercise in deductive logic rather than real-world modelling, and its
intermediate results are stored in the system of mathematics as this abstract
construct. Should humans not have this tool, humans would struggle with the
concept of quantum mechanics and how to explain it systematically. These of
available results are like the table of integrals in calculus, or table of
constants in physics, which is theoretically derivable from first principles,
but is much more efficient in application when was recorded down, encapsulated
in easy-to-remember format for application. Mathematics is effective precisely
because it provides a shortcut in deductive logic to any field that requires it.


The third reason lies in the axiomatic nature of mathematics, which I argue is
inspired from our reality and partly contributes to mathematics’s effectiveness
in describing nature. Axioms are indisputable truths which mathematicians take
for granted, and are arbitrarily defined by mathematicians themselves. They can
be as imaginative as possible, as long as it has the most important quality of
axioms: coherence, which is the inability to cause contradictions. In order for
mathematics to be useful, axioms act as the foundation of all mathematical
theories, therefore any small paradoxes in an axiomatic system could cause a
crisis in mathematics. As evident from the Foundational Crisis of Mathematics in
the 20th century due to the paradoxes found when formalizing mathematics with
naive set theory. The most famous of all is Russell’s Paradox, which one applied
version goes as follows:

\emph{“A barber only shaves those who do not shave themselves. Does the barber
  shave himself?”}

Both answers lead to a contradiction here: if the barber shaves himself, then he
is impossible as a customer to his own service by his principle; if he does not,
then he qualifies as one of his own customers since he does not shave himself. A
set-theoretic version of the statement has caused trouble in the early formation
of set theory, now the widely accepted foundation of mathematics, the
Zermelo-Fraenkel (ZF) axioms, has an axiom to avoid this situation.

Since there is a need for rigorous and coherent axioms, the formulation of such
axioms has to be intuitive and logical, fitting into the patterns we observe. I
claim that this means that throughout history, many axioms of formal systems are
inspired by reality. For example, Euclid’s axioms for geometry include facts
such as “a circle can be described by a centre and a radius” are probably
inspired by objects in the reality, although highly idealized. Peano axioms, on
the other hand, formalized natural numbers in a series of axioms such as “every
natural number has a successor, which is also a natural number” which appeals
directly to our understanding of “adding one more”. Even in the ZF axioms,
axioms such as “two sets are equal if they have the same elements” is logical
and applicable to reality. These axioms are tied to our reality, which causes
mathematics to generate results that are highly effective in describing reality,
even though the process of mathematics are analytic a priori and self-contained.
These axioms can act as suitable initial conditions for mathematics, therefore
influencing our developed mathematics to be largely coherent with reality.

\textbf{Conclusion}: The three reasons mathematics is an effective tool in
understanding nature:
\begin{enumerate}
  \item The mathematical language is concise and precise. This allowed effective
    communication between mathematicians and facilitated the development of
    mathematics.
  \item Mathematics acts as a table of pre-computed results of deductive logic,
    encoded in abstract mathematical objects and theories, which Physics and
    other domains can specialize and utilize, saving time as compared to
    deducing all these facts from scratch.
  \item Mathematical axioms which we use are “Intuitive”, hinting to its roots
    in the reality. Then its effectiveness in understanding the world is no
    longer elusive, as it is a system of knowledge built on top of nature.
\end{enumerate}

Having looked at the reasons for the effectiveness of mathematics, one might not
be satisfied and want to further claim that the reality is mathematics. We thus
look at the Mathematical Universe Hypothesis (MUH), raised by Max Tegmark, which
states that “our external physical reality is a mathematical structure”. He
further postulates that all theories about nature have two components:
mathematical equations and “baggage”, that serve as a mental model to connect
what we observe and intuitively understand about the mathematical equation.
Examples of baggage can be atoms, elements, cells, organs, animals, or the earth
and its atmosphere - they obscure the underlying mathematical structure that can
explain anything about the world. Once we strip away all this baggage, for
example, reduce a cup to its subatomic particles and the energy level, or
something even lower-level, we will discover the theory of everything, which is
just one mathematical reality. To understand this better, consider Newtonian
Mechanics, which is just the three laws of motion by Newton, expressed in
mathematical equations, combined with the abstract ideas of Forces, velocity,
acceleration and other “baggage”. In terms of mathematical concepts, the idea of
a function, which maps some input to certain fixed outputs, is a baggage on its
own, since it can be represented in terms of simple things, namely just sets, in
the set-theoretic formulation.

If we were to consider this hypothesis for a moment, we would realize a paradox
- the development of mathematics, or any other theory lies upon its “baggage”
already. It would be hard for mathematicians to develop a theory as complex as
calculus, which involves many abstract concepts (or “baggage”) such as
functions, continuity, variables, and algebra, if we were to strip away the
baggage. In fact, humans did not first come up with axioms of mathematics before
discovering some of the biggest mathematical results. Formal axiomatic systems
such as Peano Axioms did not appear for a 200 years after Newton and Leibniz
discovered calculus without the need to strip away the “baggage” of
mathematics. It is precisely these abstractions, or baggage, obscured away the
trivial details which interfered, rather than helped with mathematical
development. Even if it is possible to strip away the baggage and peek at the
underlying mathematical structure, it would be too complicated for us to
understand, much less utilize them.

Let us illustrate this viewpoint with an example from the modern foundation of
mathematics, set theory and see how numbers can be represented as sets in
reality\cite{set}.

In the beginning, let us define $0$ as the empty set $\{\}$. Now suppose a set
contains only the empty set, and we denote it be $1=\{\{\}\}$. From now onwards,
given a number $n$ and its set notation, define $n+1$ by the union of n and the
set containing only $n$. Now we can repeat this process and create
$2,3,4,5,\dots$ represented only be sets as below:
\begin{align*}
0&=\0\\
1&=0\cup\{0\}=\1\\
2&=1\cup\{1\}=\2\\
3&=2\cup\{2\}=\3\\
4&=3\cup\{3\}=\4\\
5&=4\cup\{4\}=\5\\
\end{align*}
There is as little baggage here as possible if we deem the baseline to be the ZF
Axioms, which describes sets in detail.

Now as an example, let’s look at the representation of the natural number 5 in
set theory. It is unnecessarily complicated without abstraction.
Representing 5 as an integer (not just a natural number), is a lot more
complicated still, as it is defined as the the set of all pairs of numbers (also
represented as sets) like $(5, 0), (6, 1), (7, 2), (3, -2),\dots$ as all these
pairs have a “difference” of 5. This is already an infinite structure of pairs,
just enough to define a single number. Flipping the position of every single
element in the pairs in 5, we will get -5, since $(0,5), (1, 6), (2, 7),\dots$
etc all have a difference of -5.


Now, we all know intuitively 5 + (-5) = 0. If you have 5 dollars, and you then
lose 5, you will have nothing left. However, how do we begin to make sense of
these infinite structures of pairs? How can they combine to get 0? What is this
“combine” operation? Will the result of the “combine” operation always return me
a number? These are hardly workable already to derive some everyday, intuitive
facts we have about mathematics, although it is already proven by mathematicians
to be true. This gets crazier as we further extend 5 to a rational number (it is
the infinite collection of integer pairs whose quotient is 5), or a real number
(too complicated for the scope of this essay), which are essentially infinite
structures just like above, but rational numbers are defined upon the structure
of integers, and the reals upon the rationals. Baggages like the idea of 5
concretizes the idea of such mathematical objects and make it useful for
mathematical or scientific advancements. How one can carry out any meaningful
calculation with such cumbersome structures is a miracle, much less develop
theory as advanced as quantum mechanics when we have mathematics and its baggage
stripped. Granted, the underlying reality might be mathematical, but to reach
the state where we can uncover this through by stripping away baggage, humans
are going to have a really difficult time understanding concepts intuitively as
we do now.

Coincidentally, this is what Tegmark explained in his paper about MUH: such
“baggage” is needed for intuition and shortcuts, which is the invariant in
gaining further understanding and developing new theories. Similar to how
scientists use mathematical models more than questioning them, mathematicians
themselves use mathematical concepts as baggage more than they question them.
Paradoxically, MUH’s hint at a higher reality requires to have all its “baggage”
stripped, which means further development in mathematics will be impeded.

Therefore we conclude that even if MUH is true, humans might never uncover a
satisfying theory of everything via mathematics. Therefore mathematics is only
reasonably effective.

\printbibliography
\end{document}
