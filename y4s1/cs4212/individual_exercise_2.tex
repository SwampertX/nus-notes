% Created 2021-11-01 Mon 19:39
% Intended LaTeX compiler: pdflatex
\documentclass[11pt]{article}
\usepackage[utf8]{inputenc}
\usepackage[T1]{fontenc}
\usepackage{graphicx}
\usepackage{longtable}
\usepackage{wrapfig}
\usepackage{rotating}
\usepackage[normalem]{ulem}
\usepackage{amsmath}
\usepackage{amssymb}
\usepackage{capt-of}
\usepackage{hyperref}
\usepackage{amsthm}
\usepackage{minted}
\usepackage{hyperref}
\linespread{1.0}
\usepackage[a4paper, total={6.5in, 8.5in}]{geometry}
\usepackage{crimson}
\author{Tan Yee Jian}
\date{\today}
\title{Individual Exercise 2}
\hypersetup{
 pdfauthor={Tan Yee Jian},
 pdftitle={Individual Exercise 2},
 pdfkeywords={},
 pdfsubject={},
 pdfcreator={Emacs 28.0.60 (Org mode 9.6)}, 
 pdflang={English}}
\begin{document}

\maketitle
\tableofcontents

Each section corresponds to each of the problems raised in the Assignment.

\section{General flow of compilation}
\label{sec:org7e11e74}
The main idea in the design of GCC is \textbf{``passes''}\cite{passes}. GCC
compiles source code (eg. from \texttt{C}) to assembly (eg. to \texttt{x86-64}
assembly) by using many, many passes. On the top level, passes translate one
representation of the code to another representation, that gets closer and
closer to machine code, and eventually assembly code.

Compiling code in GCC while dumping all passes yield all the passes as files:
\begin{listing}[H]
\begin{minted}{c}
a-test.c.005t.original
a-test.c.006t.gimple
a-test.c.009t.omplower
a-test.c.010t.lower
a-test.c.013t.eh
a-test.c.015t.cfg
a-test.c.017t.ompexp
a-test.c.022t.fixup_cfg1
a-test.c.023t.ssa
a-test.c.025t.nothrow
...
a-test.c.327r.shorten
a-test.c.328r.nothrow
a-test.c.329r.dwarf2
a-test.c.330r.final
a-test.c.331r.dfinish
a-test.c.332t.statistics
a-test.c.333t.earlydebug
a-test.c.334t.debug
test.c
test.s
\end{minted}
\caption{Files generated in the passes.}
\end{listing}

\subsection{Source code}
\label{sec:org43f51fe}

 This is the user input in any of the GCC-supported language such as \texttt{C} and
\texttt{Fortran}.

\subsection{GENERIC (language-independent Abstract Syntax Tree)}
\label{sec:org5658ca2}

User-input source code is first parsed using recursive descent into an
Abstract Syntax Tree. For example, the \texttt{C} code is parsed in the function
below as an entry point, in This is parsed in \texttt{gcc/c/c-parser.c:21961}:

\begin{minted}[linenos,firstnumber=21961]{c}
// Begin C-parser entry point
void c_parse_file (void)
{
  /* Use local storage to begin.  If the first token is a pragma, parse it.
     If it is #pragma GCC pch_preprocess, then this will load a PCH file
     which will cause garbage collection.  */
  c_parser tparser;

  memset (&tparser, 0, sizeof tparser);
  tparser.translate_strings_p = true;
  tparser.tokens = &tparser.tokens_buf[0];
  the_parser = &tparser;

  if (c_parser_peek_token (&tparser)->pragma_kind == PRAGMA_GCC_PCH_PREPROCESS)
    c_parser_pragma_pch_preprocess (&tparser);
  else
    c_common_no_more_pch ();

  the_parser = ggc_alloc<c_parser> ();
  *the_parser = tparser;
  if (tparser.tokens == &tparser.tokens_buf[0])
    the_parser->tokens = &the_parser->tokens_buf[0];

  /* Initialize EH, if we've been told to do so.  */
  if (flag_exceptions)
    using_eh_for_cleanups ();

  c_parser_translation_unit (the_parser);
  the_parser = NULL;
}
// End C-parser entry point
\end{minted}

Since each language has a different syntax, hence different syntax tree
structures. \textbf{GENERIC} specifies a \textbf{language-independent AST}
structure to abstract all the AST of different languages into one common
\textbf{AST}\cite{parsing-pass}, facilitating translation to further IRs (such
as GIMPLE, Tree-SSA, RTL) and optimization.

There are differences in how \textbf{GENERIC} is used in parsing.
\begin{itemize}
\item \texttt{C} parses directly into \textbf{GENERIC} but

\item \texttt{Fortran} however parses first into a private representation to \textbf{GENERIC},
which is later then ``lowered'' into \textbf{GENERIC} and \textbf{GIMPLE}\cite{passes}.
\end{itemize}

\subsection{GIMPLE (``three-address codes'' that has repeated assignment to variables)}
\label{sec:org1161d48}

This process of translating from other representations, such as \textbf{GENERIC} to
\textbf{GIMPLE} is called \textbf{``gimplification''}. The entry point for this pass is the
function \texttt{gimplify\_function\_tree()} in \texttt{gcc/gimplify.c:15447}:

\begin{listing}[H]
\begin{minted}[linenos,firstnumber=15453]{c}
void
gimplify_function_tree (tree fndecl) {
  gimple_seq seq;
  gbind *bind;

  gcc_assert (!gimple_body (fndecl));

  if (DECL_STRUCT_FUNCTION (fndecl))
    push_cfun (DECL_STRUCT_FUNCTION (fndecl));
  else
    push_struct_function (fndecl);

  /* OMITTED CODE */

      /* Replace the current function body with the body
         wrapped in the try/finally TF.  */
      seq = NULL;
      gimple_seq_add_stmt (&seq, new_bind);
      gimple_set_body (fndecl, seq);
      bind = new_bind;
    }

  if (sanitize_flags_p (SANITIZE_THREAD)
      && param_tsan_instrument_func_entry_exit)
    {
      gcall *call = gimple_build_call_internal (IFN_TSAN_FUNC_EXIT, 0);
      gimple *tf = gimple_build_try (seq, call, GIMPLE_TRY_FINALLY);
      gbind *new_bind = gimple_build_bind (NULL, tf, NULL);
      /* Replace the current function body with the body
	 wrapped in the try/finally TF.  */
      seq = NULL;
      gimple_seq_add_stmt (&seq, new_bind);
      gimple_set_body (fndecl, seq);
    }

  DECL_SAVED_TREE (fndecl) = NULL_TREE;
  cfun->curr_properties |= PROP_gimple_any;

  pop_cfun ();

  dump_function (TDI_gimple, fndecl);
}
\end{minted}
\caption{Entry point to gimplification.}
\end{listing}
\subsection{Tree-SSA (Single Static Assignment)}
\label{sec:org4b5736c}

From this point onwards, the passes are mainly optimizations. It is managed
by \texttt{gcc/passes.c} which executes passes as listed in \texttt{gcc/passes.def:29}.

\begin{listing}[H]
\begin{minted}[linenos,firstnumber=29]{c}
 /* All passes needed to lower the function into shape optimizers can
    operate on.  These passes are always run first on the function, but
    backend might produce already lowered functions that are not processed
    by these passes.  */
  INSERT_PASSES_AFTER (all_lowering_passes)
  NEXT_PASS (pass_warn_unused_result);
  NEXT_PASS (pass_diagnose_omp_blocks);
  NEXT_PASS (pass_diagnose_tm_blocks);
  NEXT_PASS (pass_omp_oacc_kernels_decompose);
  NEXT_PASS (pass_lower_omp);
  NEXT_PASS (pass_lower_cf);
  NEXT_PASS (pass_lower_tm);
  NEXT_PASS (pass_refactor_eh);
  NEXT_PASS (pass_lower_eh);
  NEXT_PASS (pass_coroutine_lower_builtins);
  NEXT_PASS (pass_build_cfg);
  NEXT_PASS (pass_warn_function_return);
  NEXT_PASS (pass_coroutine_early_expand_ifns);
  NEXT_PASS (pass_expand_omp);
  NEXT_PASS (pass_warn_printf);
  NEXT_PASS (pass_walloca, /*strict_mode_p=*/true);
  NEXT_PASS (pass_build_cgraph_edges);
  TERMINATE_PASS_LIST (all_lowering_passes)
 // many more passes
\end{minted}
\caption{Some of the passes specified in \texttt{gcc/passes.def}.}
\end{listing}
\subsection{RTL (Register Transfer Language) RTL is a machine code for an abstract}
\label{sec:org1339166}
machine with inifinitely many registers. RTL's syntax and several peephole
optimizations will be included in the sections below.
\subsection{Target assembly code}
\label{sec:org2f3f902}
This is the assembly code in the target machine architecture, such as
\texttt{ARMv7}, \texttt{RISCV} or \texttt{aarch64}. GCC's compilation end here, and the rest of
the job is given to the \texttt{as} GNU Assembler.

There are passes which do not further ``compile'' the language into the next, more
low-level version, which are known as ``optimization passes''.
\section{Parsing of C}
\label{sec:orgca0061d}
Recursive descent is used. Found in \texttt{gcc/c/c-parser.c:1800}

\begin{listing}[H]
\begin{minted}[linenos,firstnumber=1800]{c}
static void
c_parser_declaration_or_fndef (c_parser *parser, bool fndef_ok,
			       bool static_assert_ok, bool empty_ok,
			       bool nested, bool start_attr_ok,
			       tree *objc_foreach_object_declaration,
			       vec<c_token> omp_declare_simd_clauses,
			       bool have_attrs, tree attrs,
			       struct oacc_routine_data *oacc_routine_data,
			       bool *fallthru_attr_p)
{ ... }
\end{minted}
\caption{Recursive descent function to parse declarations}
\end{listing}

which parses function declarations and more. Recursive Descent allows for
better parsing error reporting as all cases can have individualized error
messages. \href{https://softwareengineering.stackexchange.com/a/254707}{StackExchange post} Example at \texttt{gcc/c/c-parser.c:1937}:

\begin{listing}[H]
\begin{minted}[linenos,firstnumber=1937]{c}
	{
	  /* This is not C++ with its implicit typedef.  */
	  richloc.add_fixit_insert_before ("struct ");
	  error_at (&richloc,
		    "unknown type name %qE;"
		    " use %<struct%> keyword to refer to the type",
		    name);
	}
      else if (tag_exists_p (UNION_TYPE, name))
	{
	  richloc.add_fixit_insert_before ("union ");
	  error_at (&richloc,
		    "unknown type name %qE;"
		    " use %<union%> keyword to refer to the type",
		    name);
	}
      else if (tag_exists_p (ENUMERAL_TYPE, name))
	{
	  richloc.add_fixit_insert_before ("enum ");
	  error_at (&richloc,
		    "unknown type name %qE;"
		    " use %<enum%> keyword to refer to the type",
		    name);
	}
      else
	{
	  auto_diagnostic_group d;
	  name_hint hint = lookup_name_fuzzy (name, FUZZY_LOOKUP_TYPENAME,
					      here);
	  if (const char *suggestion = hint.suggestion ())
	    {
	      richloc.add_fixit_replace (suggestion);
	      error_at (&richloc,
			"unknown type name %qE; did you mean %qs?",
			name, suggestion);
	    }
	  else
	    error_at (here, "unknown type name %qE", name);
	}
\end{minted}
\caption{Some detailed error messages in the \texttt{C}-parser}
\end{listing}
\section{Intermediate code formats}
\label{sec:org0d4d492}
In this section and the next, we use this running example of \texttt{C} code:

\begin{listing}[H]
\begin{minted}[]{c}
int main() {
  int x = 1 + 2;
  char c = 's';
  while (x < 10) {
    x += 1;
  }
  return x;
}
\end{minted}
\caption{Running example in this section, in \texttt{C}}
\end{listing}
\subsection{Motivation}
\label{sec:org9ebb2ab}
\begin{itemize}
\item Before GENERIC and GIMPLE were invented, the parsed AST for every language
were immediately translated into Register Transfer Language (RTL), which is
like ``assembly language with infinite number of registers''.
  \item The lack of a common structure caused each language to have to write a
        \textbf{AST to RTL} compiler, on top of the \textbf{parser}. The use of
        \textbf{GENERIC} abstracted this common work done by each language
        front-end.
  \item Directly translating from AST to RTL loses many properties of the code,
        ``for example, array references, data types, references to program
        variables, control flow structures''\cite{novillo_2004}. Even function
        calls are expanded to more than one instructions, loosing the structure
        of functions, hence losing opportunities for function optimizations such
        as \emph{folding} and \emph{dead code elimination}.
  \item Solution: ``GENERIC addresses the lack of a common tree representation
        among the various front ends. GIMPLE solves the complexity problems that
        facilitate the discovery of data and control flow in the
        program.''\cite{novillo_2004}
  \item \textbf{GIMPLE} in three-address codes provides a good structure for
        optimizations, developed actively in
        research(\cite{tree-ssa-passes},\cite{rtl-passes}).
\end{itemize}
\subsection{GENERIC}
\label{sec:org0ff633c}
\textbf{GENERIC} is a common AST structure for all gcc-compatible languages.
There are no optimizations happening at this point, until the code is
``gimplified'' or converted to \textbf{GIMPLE}, which is in three-address code.
\subsection{GIMPLE}
\label{sec:orgb283b4c}
The \texttt{C} code compiles to the following \textbf{GIMPLE} code:
\begin{listing}[H]
\begin{minted}[]{c}
int main ()
{
  int D.1950;

  {
    int x;
    char c;

    x = 3;
    c = 115;
    goto <D.1947>;
    <D.1948>:
    x = x + 1;
    <D.1947>:
    if (x <= 9) goto <D.1948>; else goto <D.1946>;
    <D.1946>:
    D.1950 = x;
    return D.1950;
  }
  D.1950 = 0;
  return D.1950;
}
\end{minted}
\caption{\textbf{GIMPLE} code generated by \texttt{C}}
\end{listing}

\textbf{GIMPLE} is basically three-address code, however many optimizations rely
on code being in the \textbf{Static Single Assignment} form, called
\textbf{Tree-SSA} or \textbf{SSA} in GCC. After some passes, GCC obtains the
\textbf{SSA} as below:

\begin{minted}[]{c}
int main ()
{
  char c;
  int x;
  int _4;

  <bb 2> :
  x_2 = 3;
  c_3 = 115;
  goto <bb 4>; [INV]

  <bb 3> :
  x_6 = x_1 + 1;

  <bb 4> :
  # x_1 = PHI <x_2(2), x_6(3)>
  if (x_1 <= 9)
    goto <bb 3>; [INV]
  else
    goto <bb 5>; [INV]

  <bb 5> :
  _4 = x_1;
  return _4;

}
\end{minted}
At this point, many interesting peephole optimizations can already be done, such
 as ``Tail-call optimization'' and ``Conditional constant propogation (CCP)'', which
 we will discuss in the last section.

\subsection{Register Transfer Language (RTL)}
\label{sec:org4eee840}
\textbf{RTL} is described as an ``assembly language with infinite number of
registers'', inspired by LISP lists and describes the instructions to be
outputed eventually in a slightly higher level. Internally it is a graph-like
structure with references, but printed as nested brackets similar to LISP.

This will be explained in detail in the next section.
\section{Explain RTL representation}
\label{sec:orga6f1ca1}
We will explain using this RTL code, taken from the previous section. All the
in-code comments are mine.

\begin{listing}[H]
\begin{minted}[]{elisp}
(note 1 0 3 NOTE_INSN_DELETED) ;; notes are just debugging information.
(note 3 1 13 2 [bb 2] NOTE_INSN_BASIC_BLOCK)
(note 13 3 2 2 NOTE_INSN_PROLOGUE_END)
(note 2 13 9 2 NOTE_INSN_FUNCTION_BEG)
(insn 9 2 10 2
;; insn are instructions that do not jump and are not function calls
    (set (reg/i:SI 0 ax)
    ;; sets the register with the constant below.
    ;; (reg/i:SI 0 ax):
    ;;   reg means registers
    ;;   /i here means the value is a scalar that is not part of an aggregate
    ;;   SI here represents the Single Integer mode of 4 bytes.
    ;;   0 is a hard register number.
        (const_int 10 [0xa])) "test.c":8:1 75 {*movsi_internal}
        ;; this is a constant integer of value 10, with representation 0xa.
     (nil))
(insn 10 9 14 2 (use (reg/i:SI 0 ax)) "test.c":8:1 -1
;; this is an instruction that loads data from the register for the return below.
     (nil))
(note 14 10 15 2 NOTE_INSN_EPILOGUE_BEG)
(jump_insn 15 14 16 2 (simple_return) "test.c":8:1 837 {simple_return_internal}
;; this is a jump-instruction.
     (nil)
 -> simple_return)
;; simple_return is the return instruction without the function-return epilogue.
(barrier 16 15 12)
;; barrier denotes the end of control flow.
(note 12 16 0 NOTE_INSN_DELETED)
\end{minted}
\caption{RTL compiled, with in-code comment explanations}
\end{listing}

\section{Three peephole optimizations in GCC}
Peephole optimizations are found in the directory =gcc/config= and are sorted by
each architecture. Peephole optimizations are defined with the following
structure:

\begin{listing}[H]
\begin{minted}{c}
     (define_peephole2
       [insn-pattern-1
        insn-pattern-2
        ...]
       "condition"
       [new-insn-pattern-1
        new-insn-pattern-2
        ...]
       "preparation-statements")
\end{minted}
\caption{Structure of peephole definition.}
\end{listing}

\subsection{Instruction compaction in \texttt{=gcc/config/arm/arm.md:1183}}
\begin{listing}
\begin{minted}{elisp}
(define_peephole2
  [(set (match_operand:SI 0 "arm_general_register_operand" "")
	(plus:SI (match_operand:SI 1 "arm_general_register_operand" "")
		 (const_int -1)))
   (set (match_operand 2 "cc_register" "")
	(compare (match_dup 0) (const_int -1)))
   (set (pc)
	(if_then_else (match_operator 3 "equality_operator"
		       [(match_dup 2) (const_int 0)])
		      (match_operand 4 "" "")
		      (match_operand 5 "" "")))]
  "TARGET_32BIT && peep2_reg_dead_p (3, operands[2])"
  [(parallel[
    (set (match_dup 2)
	 (compare:CC
	  (match_dup 1) (const_int 1)))
    (set (match_dup 0) (plus:SI (match_dup 1) (const_int -1)))])
   (set (pc)
	(if_then_else (match_op_dup 3 [(match_dup 2) (const_int 0)])
		      (match_dup 4)
		      (match_dup 5)))]
  "operands[2] = gen_rtx_REG (CCmode, CC_REGNUM);
   operands[3] = gen_rtx_fmt_ee ((GET_CODE (operands[3]) == NE
				  ? GEU : LTU),
				 VOIDmode,
				 operands[2], const0_rtx);"
)
\end{minted}
\caption{Compacting 3 instructions into 2.}
\end{listing}

This matches the instructions
\begin{listing}
\begin{minted}{as}
 sub  rd, rn, #1
 cmn  rd, #1	(equivalent to cmp rd, #-1)
 bne  dest
\end{minted}
\end{listing}

into

\begin{listing}
\begin{minted}{as}
 subs rd, rn, #1
 bcs  dest	((unsigned)rn >= 1)
\end{minted}
\end{listing}

which is a common loop idiom in cases such as \verb|while(n--)|.

\subsection{Replacing instruction in =gcc/config/arm/arm.md:9827}
\begin{listing}[H]
\begin{minted}{elisp}
(define_peephole2
  [(set (reg:CC CC_REGNUM)
	(compare:CC (match_operand:SI 1 "register_operand" "")
		    (const_int 0)))
   (cond_exec (ne (reg:CC CC_REGNUM) (const_int 0))
	      (set (match_operand:SI 0 "register_operand" "") (const_int 0)))
   (cond_exec (eq (reg:CC CC_REGNUM) (const_int 0))
	      (set (match_dup 0) (const_int 1)))
   (match_scratch:SI 2 "r")]
  "TARGET_32BIT && peep2_regno_dead_p (3, CC_REGNUM)"
  [(parallel
    [(set (reg:CC CC_REGNUM)
	  (compare:CC (const_int 0) (match_dup 1)))
     (set (match_dup 2) (minus:SI (const_int 0) (match_dup 1)))])
   (set (match_dup 0)
	(plus:SI (plus:SI (match_dup 1) (match_dup 2))
		 (geu:SI (reg:CC CC_REGNUM) (const_int 0))))]
)
\end{minted}
\caption{Replacing conditional instructions with parallel ones.}
\end{listing}

This code replaces patterns of the form \verb|Rd = (eq (reg1) (const_int0))|
into the instructions:
\begin{listing}[H]
\begin{minted}{as}
negs Rd, reg1
adc  Rd, Rd, reg1
\end{minted}
\end{listing}

which is more efficient.

\subsection{Parallelization of operations in \texttt{gcc/config/arm/arm.md:11401}}
\begin{listing}
\begin{minted}{elisp}
(define_peephole2
  [(set (match_operand:SI 0 "arm_general_register_operand" "")
	(match_operand:SI 1 "arm_general_register_operand" ""))
   (set (reg:CC CC_REGNUM)
	(compare:CC (match_dup 1) (const_int 0)))]
  "TARGET_ARM"
  [(parallel [(set (reg:CC CC_REGNUM) (compare:CC (match_dup 1) (const_int 0)))
	      (set (match_dup 0) (match_dup 1))])]
  ""
)
\end{minted}
\caption{Parallelizing two \texttt{set} instructions in \texttt{ARM}.}
\end{listing}

This peephole optimization basically takes two ``set'' (store) operations and
paralellize them, utilizing the special instruction in ARM architecture.

\section{Extra: Three non-peephole optimizations in GCC}
Other than peephole optimizations, there are many other optimizations in GCC
that are worthy of noting, occuring at both the RTL level and the Tree SSA
level. We list 3 below.
\label{sec:org3859f52}
\subsection{Tail-call optimization (in \texttt{gcc/tree-tailcall.c})}
\label{sec:orge8ff33e}
\begin{itemize}
\item Tail-calls occur when a recursive function returns a call to itself as part of
explicit recursion. Since it happens at the end of a function, it is named as
tail-call.
\item Tail-call optimization aims to remove unnecessary recursion (which takes up
the stack) with loops.
\item Furthermore, functions where the tail-call is not explicit is also optimized
by GCC using accumulators. Below is an example of the recursive \texttt{sum()}
function:

\begin{listing}[H]
\begin{minted}[]{c}
   int sum (int n)
   {
     if (n > 0)
       return n + sum (n - 1);
     else
       return 0;
   }
\end{minted}
\caption{Before tail-call optimization}
\end{listing}

is transformed into

\begin{listing}[H]
\begin{minted}[]{c}
   int sum (int n)
   {
     int acc = 0;

     while (n > 0)
       acc += n--;

     return acc;
   }
\end{minted}
\caption{After tail-call optimization}
\end{listing}
\end{itemize}

Example function in \texttt{gcc/tree-tailcall.c:1086}:
\begin{listing}[H]
\begin{minted}[linenos,firstnumber=1086]{c}
static unsigned int
tree_optimize_tail_calls_1 (bool opt_tailcalls)
{
  edge e;
  bool phis_constructed = false;
  struct tailcall *tailcalls = NULL, *act, *next;
  bool changed = false;
  basic_block first = single_succ (ENTRY_BLOCK_PTR_FOR_FN (cfun));
  tree param;
  gimple *stmt;
  edge_iterator ei;

  if (!suitable_for_tail_opt_p ())
    return 0;
  if (opt_tailcalls)
    opt_tailcalls = suitable_for_tail_call_opt_p ();

  FOR_EACH_EDGE (e, ei, EXIT_BLOCK_PTR_FOR_FN (cfun)->preds)
    {
      /* Only traverse the normal exits, i.e. those that end with return
	 statement.  */
      stmt = last_stmt (e->src);

      if (stmt
	  && gimple_code (stmt) == GIMPLE_RETURN)
	find_tail_calls (e->src, &tailcalls);
    }
\end{minted}
\caption{Entry point to tail-call optimization (\texttt{gcc/tree-tailcall.c})}
\end{listing}
\subsection{Conditional constant propogation \& Folding built-in functions (\texttt{tree-ssa-ccp.c})}
\label{sec:org92465e6}
Conditional constant propagation (CCP) propogates statements of the pattern
\texttt{VAR = CONSTANT} to the rest of the program. The algorithm is based on the
assumption that the code is SSA. The common operation of propogation can be seen
in this code snippet in \texttt{gcc/tree-ssa-propagate.c:1444}:

\begin{listing}[H]
\begin{minted}[linenos,firstnumber=1444]{c}
static void
replace_exp_1 (use_operand_p op_p, tree val,
    	       bool for_propagation ATTRIBUTE_UNUSED)
{
  if (flag_checking)
    {
      tree op = USE_FROM_PTR (op_p);
      gcc_assert (!(for_propagation
		  && TREE_CODE (op) == SSA_NAME
		  && TREE_CODE (val) == SSA_NAME
		  && !may_propagate_copy (op, val)));
    }

  if (TREE_CODE (val) == SSA_NAME)
    SET_USE (op_p, val);
  else
    SET_USE (op_p, unshare_expr (val));
}
\end{minted}
\caption{Entry point to CCP}
\end{listing}

After constants are propagated, built-in functions are also simplified if the
arguments are constants. In \texttt{gcc/tree-ssa-ccp.c:3255}

\begin{listing}[H]
\begin{minted}[linenos,firstnumber=3255]{c}
unsigned int
pass_fold_builtins::execute (function *fun)
{
  bool cfg_changed = false;
  basic_block bb;
  unsigned int todoflags = 0;

  FOR_EACH_BB_FN (bb, fun)
    {
      gimple_stmt_iterator i;
      for (i = gsi_start_bb (bb); !gsi_end_p (i); )
	{
	  gimple *stmt, *old_stmt;
	  tree callee;
	  enum built_in_function fcode;

	  stmt = gsi_stmt (i);
      // --snip--
    }
    // --snip--
    }
// --snip--
}
\end{minted}
\caption{Entry point to built-in function folding}
\end{listing}
\subsection{Delayed branch scheduling (in \texttt{reorg.c})}
\label{sec:orgb52a035}
Recall in CS2100, we know that branch instructions incur Branch Penalty, which
are extra cycles that are needed compared to a, say, \texttt{ADD} instruction. These
extra cycles are made use of by rearranging non branch-critical instructions to
execute during the Branch Penalty to offset the cost.

From \texttt{gcc/reorg.c:3738}:
\begin{minted}[linenos,firstnumber=3738]{c}
/* Try to find insns to place in delay slots.  */

static void
dbr_schedule (rtx_insn *first)
{
  rtx_insn *insn, *next, *epilogue_insn = 0;
  int i;
  bool need_return_insns;

  /* If the current function has no insns other than the prologue and
     epilogue, then do not try to fill any delay slots.  */
  if (n_basic_blocks_for_fn (cfun) == NUM_FIXED_BLOCKS)
    return;

  /* Find the highest INSN_UID and allocate and initialize our map from
     INSN_UID's to position in code.  */
  for (max_uid = 0, insn = first; insn; insn = NEXT_INSN (insn))
    {
      if (INSN_UID (insn) > max_uid)
	max_uid = INSN_UID (insn);
      if (NOTE_P (insn)
	  && NOTE_KIND (insn) == NOTE_INSN_EPILOGUE_BEG)
	epilogue_insn = insn;
    }
  // CODE OMITTED FOR BREVITY
}
\end{minted}

\bibliographystyle{plain}
\bibliography{references.bib}
\end{document}
