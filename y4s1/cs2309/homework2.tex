\documentclass{article}

\newcommand{\myname}{Tan Yee Jian (A0190190L)}
\newcommand{\mytitle}{CS2309 Assignment 2}
\title{\mytitle}
\author{\myname}
\date{\today}

\usepackage[a4paper, total={6in, 9.7in}]{geometry}
\usepackage[utf8]{inputenc}
\usepackage[T1]{fontenc}
\usepackage{textcomp}
\usepackage{amsmath, amssymb, amsthm}
\theoremstyle{plain}
% \usepackage[outputdir=tmp]{minted}
% \usepackage{lmodern}
\usepackage{tgpagella}
\usepackage{fancyhdr}
\usepackage{lastpage}
\pagestyle{fancy}
\fancyhf{}
% \rhead{Page \thepage/\pageref{LastPage}}
\rhead{Page \thepage}
\lhead{\myname}
\chead{\mytitle}

\usepackage[backend=biber, style=alphabetic]{biblatex}
\addbibresource{references.bib}

% \usepackage{tocloft}
% \usepackage[thinc]{esdiff}
% \renewcommand{\thesection}{Question \arabic{section}}
% \renewcommand{\thesubsection}{Part \arabic{section}(\roman{subsection})}
% \renewcommand{\thesubsubsection}{Solution}
% \cftsetindents{subsection}{1.5em}{4.5em}
% \cftsetindents{subsubsection}{3.8em}{5.5em}

\newcommand{\pmat}[1]{ \begin{pmatrix}#1\end{pmatrix} }
\newcommand{\seqn}[1]{(#1)^\infty_{n=1}}
\newcommand{\seqk}[1]{(#1)^\infty_{k=1}}
% (series term): returns a series with counter n=1 to \infty.
\newcommand{\infsrsn}[1]{\sum\limits^\infty_{n=1}#1}
\newcommand{\infsrsk}[1]{\sum\limits^\infty_{k=1}#1}
\newcommand{\R}{\mathbb{R}}
\newcommand{\N}{\mathbb{N}}
\newcommand{\Q}{\mathbb{Q}}
\newcommand{\Z}{\mathbb{Z}}
\newcommand{\C}{\mathbb{C}}
\newcommand{\F}{\mathbb{F}}
\newcommand{\cmm}{C(M_1,M_2)}
\newcommand{\met}[1]{\langle M_{#1},\rho_{#1}\rangle}
\newcommand{\ntoinf}{\limits_{n\to\infty}}
\newcommand{\ktoinf}{\limits_{k\to\infty}}
% \newcommand{\onetoinf}[]{^\infty_{n=1}}
\newcommand{\limn}[1]{\lim\ntoinf #1}
\newcommand{\limk}[1]{\lim\ktoinf #1}

\DeclareMathOperator{\spn}{span}
\DeclareMathOperator{\diam}{diam}

\begin{document}
\maketitle
\section{Problem Statement}

\cite{10.1145/2213836.2213862} Log-structured Merge Trees (LSM Trees) have a
good write throughput, but high read amplification. This is the opposite of the
commonly used update-in-place storage systems (such as B-Trees) which have great
read performance, but low write throughput. This is costly because as observed
at Yahoo!, the low latency workloads that emphasize on writes are increasing
from 10-20\% to about 50\%, which is getting unsuitable for update-in-place
systems. LSM Trees are a good candidate for write-intensive tasks, but
modifications need to be made so it has a better real-world performance.

\subsection{Solution}
Therefore, the researchers have come up with \emph{bLSM Trees}, a LSM Tree with the
advantages of B-Trees. This is done by adding Bloom filters in LSM Trees (hence
the \emph{b} in \emph{bLSM}) to increase index performance and replacing the
merge scheduler in LSM Trees with the new ``spring and gear'' merge scheduler.
When measured under a new performance metric, \emph{read fanout} that is claimed
to better characterize real-world index operations, \emph{bLSM} outperform
B-Trees in most cases.

\section{Pros and Cons}
\subsection{Pros of \emph{bLSM}}
\begin{enumerate}
\item Using Bloom filter reduces read amplification by trimming the search tree,
from $N$ to $1 + \frac{N}{100}$. This allows \emph{bLSM} to have near-optimal
read performance.
\item The new ``spring and gear'' scheduler solves the problem of
\emph{snowshoveling}, so application writes are not blocked for long period of
times, improving write latencies.
\end{enumerate}


\subsection{Cons}
\begin{enumerate}
  \item Merging in \emph{bLSM} Trees is difficult to implement, solving the issues
        using mutex introduces additional concurrency overheads.
    \item Even though existing LSM Trees implement write-ahead logs for
    consistency upon crash, the bloom filters are too huge to write to disk and
    therefore complicate crash recovery.
\end{enumerate}

\printbibliography
\end{document}
