\documentclass{article}

\newcommand{\myname}{Tan Yee Jian (A0190190L)}
\newcommand{\mytitle}{MA3219 HW2}
\title{\mytitle}
\author{\myname}
\date{\today}

\usepackage[a4paper, total={6in, 9.7in}]{geometry}
\usepackage[utf8]{inputenc}
\usepackage[T1]{fontenc}
\usepackage{textcomp}
\usepackage{amsmath, amssymb, amsthm}
\theoremstyle{plain}
% \usepackage[outputdir=tmp]{minted}
% \usepackage{lmodern}
\usepackage{tgpagella}
\usepackage{fancyhdr}
\usepackage{lastpage}
\usepackage{enumitem}
\pagestyle{fancy}
\fancyhf{}
% \rhead{Page \thepage/\pageref{LastPage}}
\rhead{Page \thepage}
\lhead{\myname}
\chead{\mytitle}

\usepackage{tocloft}
\usepackage[thinc]{esdiff}
\renewcommand{\thesection}{Question \arabic{section}}
\renewcommand{\thesubsection}{Part \arabic{section}(\roman{subsection})}
\renewcommand{\thesubsubsection}{Solution}
\cftsetindents{subsection}{1.5em}{4.5em}
\cftsetindents{subsubsection}{3.8em}{5.5em}

\newcommand{\pmat}[1]{ \begin{pmatrix}#1\end{pmatrix} }
\newcommand{\seqn}[1]{(#1)^\infty_{n=1}}
\newcommand{\seqk}[1]{(#1)^\infty_{k=1}}
% (series term): returns a series with counter n=1 to \infty.
\newcommand{\infsrsn}[1]{\sum\limits^\infty_{n=1}#1}
\newcommand{\infsrsk}[1]{\sum\limits^\infty_{k=1}#1}
\newcommand{\R}{\mathbb{R}}
\newcommand{\N}{\mathbb{N}}
\newcommand{\Q}{\mathbb{Q}}
\newcommand{\Z}{\mathbb{Z}}
\newcommand{\C}{\mathbb{C}}
\newcommand{\F}{\mathbb{F}}
\newcommand{\cmm}{C(M_1,M_2)}
\newcommand{\met}[1]{\langle M_{#1},\rho_{#1}\rangle}
\newcommand{\ntoinf}{\limits_{n\to\infty}}
\newcommand{\ktoinf}{\limits_{k\to\infty}}
% \newcommand{\onetoinf}[]{^\infty_{n=1}}
\newcommand{\limn}[1]{\lim\ntoinf #1}
\newcommand{\limk}[1]{\lim\ktoinf #1}

\DeclareMathOperator{\spn}{span}
\DeclareMathOperator{\diam}{diam}

\begin{document}
\maketitle
% \tableofcontents
\section{}
\begin{proof}[Solution]
We use the diagonalization argument. Suppose there is a universal $\alpha:\N^{2}\to\N$,
such that for any total recursive function $f$, we must have some $n\in\N$ such
that \[\forall x\in\N, f(x) = \alpha(n, x).\]

Then let $g:\N\to\N$ be defined by \[g(x) = \alpha(x, x) + 1.\] It is easy to see that
$g$ is total recursive since $\alpha$ is total recursive. Suppose $\alpha$ is universal,
then $g(x) = \alpha(n, x)$ for some $n\in\N$ by the definition of universal function.
Then feeding $n$ as the input to $g$,
\begin{align*}
  g(n) = \alpha(n,n) \ne \alpha(n,n) + 1,
\end{align*}
a contradiction to the definition of $g$.
\end{proof}

\section{}
\subsubsection{}
Our solution has the following idea:
\begin{enumerate}
  \item The input has form $01^{x}01^{y}$.
  \item We replace the 1 at the beginning of both sections pair-by-pair.
  \item If any of them run out but the other have not, then output $01$, which
        represents \verb|false|.
  \item Otherwise, output $011$, which represents \verb|true|.
\end{enumerate}



\section{}
\subsubsection{}
We can encode the (ordered) alphabet using the binary digits, representing
$a_{n}$ as $n$ in binary. For the tape, treat every $\lceil\log_{2}(n)\rceil$ cells as a
unit, and operate on it.

\medskip It is possible to use $\{\square,1\}$ as alphabet. We use $1^{x}$ to
represent $a_{x}$, and use spaces to delimit. Two or more consecutive blank
cells indicate the either end of the tape.

\medskip But I read online that Claude Shannon proved that one-symbol Turing
machines are not universal.

\section{}
\subsection{}
We assume knowledge on calling another Turing Machine as a subprogram - simply
changing the initial and final states of the subprogram so it fits into the
current program.
\begin{enumerate}[label={\Roman*.}]
  \item {
    Primitive recursion: let the Turing Machines representing
    $g:\N^{m}\to\N, h:\N^{m+2}\to\N$ be $M_{g}, M_{h}$ respectively. To implement
    $f(\overrightarrow{x},y)$, we mark the number of times to the left of the
    input and build the result systematically:
    \begin{enumerate}
      \item First mark $1^{y}$ at the left of the input.
      \item Calculate $f(\overrightarrow{x}, 0)$ by calling $M_{g}$ on
            $\overrightarrow{x}$.
      \item If to the left of the input is already cleared, return the result.
            Otherwise, remove a mark from the left of the input, and calculate
            the next iteration using $M_{h}$.
      \item If the left to the input is cleared, return the result. Otherwise
            go to (c).
    \end{enumerate}
  }
  \item { Minimization: suppose the partial recursive function $f:\N^{m}\to\N$ is
        defined as a minimization of some Turing Computable $g:\N^{m+1}\to\N$. To
        code a Turing Machine for $f$, we just need to literally search from
        ${0,1,2,\ldots}$, and at the end of each step increment a ``counter''. If the
        result is found, just return the ``counter'' as an output.
    \begin{enumerate}
      \item First let the input $01^{x_{1}}01^{x_{2}}\ldots$ be given.
      \item Mark the counter (eg. to the left of the input) as $0$.
      \item Run $M_{g}$, the Turing machine representing $g$ on
            $\overrightarrow{x}$ and counter.
      \item If the result returned by $M_{g}$ is $0$, return this value of
            counter. Otherwise, go to (c).
    \end{enumerate}
  }
\end{enumerate}

% \bigskip
% \begin{center}
% End of Assignment Solutions
% \end{center}
\end{document}
