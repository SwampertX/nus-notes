\documentclass{article}

\newcommand{\myname}{Tan Yee Jian (A0190190L)}
\newcommand{\mytitle}{MA3201 Homework 4}
\title{\mytitle}
\author{\myname}
\date{\today}

\usepackage[a4paper, total={6in, 9.7in}]{geometry}
% \usepackage[utf8]{inputenc}
% \usepackage[T1]{fontenc}
\usepackage{textcomp}
\usepackage{amsmath, amssymb, amsthm}
\theoremstyle{plain}
% \usepackage[outputdir=tmp]{minted}
\usepackage{lmodern}
\usepackage{fancyhdr}
\usepackage{lastpage}
\pagestyle{fancy}
\fancyhf{}
% \rhead{Page \thepage/\pageref{LastPage}}
\rhead{Page \thepage}
\lhead{\myname}
\chead{\mytitle}

\newtheorem{thm}{Theorem}[section]
\newtheorem{crl}{Corollary}[thm]
\newtheorem{lemma}{Lemma}[thm]
\newtheorem*{lemma*}{Lemma}
\newtheorem{note}{Note}[thm]
\newtheorem{defn}{Definition}[section]
\newtheorem{ex}{Example}[section]
\newtheorem{prop}{Proposition}[section]
\newtheorem{obs}{Observation}
\newtheorem{claim}{Claim}

\newcommand{\pmat}[1]{ \begin{pmatrix}#1\end{pmatrix} }
\newcommand{\seqn}[1]{(#1)^\infty_{n=1}}
\newcommand{\seqk}[1]{(#1)^\infty_{k=1}}
% (series term): returns a series with counter n=1 to \infty.
\newcommand{\infsrsn}[1]{\sum\limits^\infty_{n=1}#1}
\newcommand{\infsrsk}[1]{\sum\limits^\infty_{k=1}#1}
\newcommand{\R}{\mathbb{R}}
\newcommand{\N}{\mathbb{N}}
\newcommand{\Q}{\mathbb{Q}}
\newcommand{\Z}{\mathbb{Z}}
\newcommand{\C}{\mathbb{C}}
\newcommand{\F}{\mathbb{F}}
\newcommand{\cmm}{C(M_1,M_2)}
\newcommand{\met}[1]{\langle M_{#1},\rho_{#1}\rangle}
\newcommand{\ntoinf}{\limits_{n\to\infty}}
\newcommand{\ktoinf}{\limits_{k\to\infty}}
% \newcommand{\onetoinf}[]{^\infty_{n=1}}
\newcommand{\limn}[1]{\lim\ntoinf #1}
\newcommand{\limk}[1]{\lim\ktoinf #1}

\DeclareMathOperator{\spn}{span}
\DeclareMathOperator{\diam}{diam}
\DeclareMathOperator{\Tor}{Tor}
\DeclareMathOperator{\Hom}{Hom}
\DeclareMathOperator{\End}{End}

\begin{document}
\maketitle
\section*{Problem 1}
Let $R$ be a ring with $1\ne0$. Let $M$ be a $R$-module. We define
\[\Tor(M)=\{m\in M|rm=0,\text{ for some }0\ne r\in R\}.\]
\subsection*{Problem 1.1}
Prove that $\Tor(M)$ is a submodule of $M$ if $R$ is an integral domain.
\subsubsection*{Solution}
\begin{proof}
  We show by the submodule criterion.
  $\Tor(M)\ni0=1\cdot0$ is not empty.

  For any $r\in R, m_{1},m_{2}\in M$,
  \begin{align*}
    r_{1}r_{2}(m_{1}+rm_{2}) &= r_{2}(r_{1}m_{1})+rr_{1}(r_{2}m_{2})
    &&\text{$R$ is commutative, $M$ is a $R$-module}\\
                             &=r_{2}\cdot0+rr_{1}\cdot0\\ &=0+0=0\\
                             &\implies (m_{1}+rm_{2})\in\Tor(M).
  \end{align*}
  $\therefore\Tor(M)$ satisties the submodule criteiron and is a submodule of $M$.
\end{proof}
\subsection*{Problem 1.2}
Give a counterexample of the statement above for general $R$.
\subsubsection*{Solution}
\begin{proof}
  Consider $R=\Z/10\Z$ as a $R$-module by left multiplication. Then
  $2\cdot5=5\cdot2=0$ implies $\{2,5\}\subseteq\Tor(M)$, but $2+5=7\notin\Tor(M)$.
\end{proof}

\section*{Problem 2}
Let $I$ be a right ideal of $R$. Let $M$ be a left $R$-module. We define
\[N=\{m\in M|rm=0\ \forall r\in I\}.\] Prove that $N$ is a $R$-submodule of $M$.

\subsubsection*{Solution}
\begin{proof}
  We show by the submodule criterion. Since $\forall r\in I\subset R, r\cdot0=0\implies 0\in N\ne\emptyset$.

  Also for any $r\in R, i\in I, m,m'\in N$,
  \begin{align*}
    i(m+rm')&=im+i(rm')\\
            &=0+i'm' &&I\text{ is an ideal}\implies ir=i'\text{ for some }i'\in R\\
            &=0+0=0\\
            &\implies (m+rm')\in N.
  \end{align*}
  Therefore $N$ satisfies the submodule criterion and hence is a submodule of $M$.
\end{proof}

\section*{Problem 3}
Let $R$ be a ring with $1\ne0$. Let $M,N$ be (left) $R$-modules. Consider the
abelian group $\Hom_{\Z}(M,N)$ of $\Z$-module homomorphisms from $M$ to $N$.
\subsection*{Problem 3.1}
For any $r\in R$, we define \[(rf)(x)=r(f(x))\text{, for }f\in\Hom_{\Z}(M,N),x\in M,\]
where the right hand side is the action of $r$ on $N$. Prove that
$\Hom_{\Z}(M, N)$ is a left $R$-module with the action defined above.
\subsubsection*{Solution}
\begin{proof}
          For any $x\in M, r\in R, f,g\in\Hom_{\Z}(M,N)$,
  \begin{enumerate}
    \item Show $r(f+g)=rf+rg$. $\forall x\in M$,
          \begin{align*}
            (r(f+g))x&=r((f+g)x)&&\text{definition}\\
                     &=r(fx+gx)&&\Hom_{\Z}(M,N)\text{ ring distribution }\\
                     &=(rf+rg)(x) &&\text{definition}.
        \end{align*}
    \item Show $(rs)f=r(sf)$. $\forall x\in M$,
          \begin{align*}
            ((rs)f)(x)&=(rs)f(x)&&\text{definition}\\
                      &=r(sf(x))&&N\text{ is a }R\text{-module}\\
                      &=(r(sf))x&&\text{definition}.\\
          \end{align*}
    \item Show $(r+s)f=rf+sf$. $\forall x\in M$,
          \begin{align*}
            ((r+s)f)(x)&=(r+s)(f(x))&&\text{definition}\\
                      &=rf(x)+sf(x)&&N\text{ is a }R\text{-module}\\
                      &=(rf+sf)x&&\text{definition}.
          \end{align*}
    \item Show $1\cdot f=f$. $\forall x\in M$,
          \begin{align*}
            (1\cdot f)(x)&=1(f(x))&&\text{definition}\\
                      &=f(x)&&\text{$N$ is a $R$-module}.
          \end{align*}
\end{enumerate}
Result follows.
\end{proof}
\subsection*{Problem 3.2}
For any $r\in R$, we define \[(fr)(x)=f(rx)\text{, for }f\in\Hom_{\Z}(M,N),x\in M,\]
where the right hand side is the action of $r$ on $N$. Prove that
$\Hom_{\Z}(M, N)$ is a right $R$-module with the action defined above.
\subsubsection*{Solution}
\begin{proof}
          For any $x\in M, r\in R, f,g\in\Hom_{\Z}(M,N)$,
  \begin{enumerate}
    \item Show $(f+g)r=fr+gr$. $\forall x\in M$,
          \begin{align*}
            ((f+g)r)x&=(f+g)(rx)&&\text{definition}\\
                     &=f(rx)+g(rx)&&\Hom_{\Z}(M,N)\text{ ring distribution }\\
                     &=(fr+gr)(x) &&\text{definition}.
        \end{align*}
    \item Show $f(rs)=(fr)s$. $\forall x\in M$,
          \begin{align*}
            (f(rs))(x)&=f((rs)x)&&\text{definition}\\
                      &=f(r(sx))&&M\text{ is a }R\text{-module}\\
                      &=(fr)(sx)&&\text{definition}.\\
          \end{align*}
    \item Show $f(r+s)=fr+fs$. $\forall x\in M$,
          \begin{align*}
            (f(r+s))(x)&=f((r+s)x)&&\text{definition}\\
                      &=f(rx+sx))&&M\text{ is a }R\text{-module}\\
                      &=(fr+fs)x&&\text{definition}.\\
          \end{align*}
    \item Show $f\cdot1=f$. $\forall x\in M$,
          \begin{align*}
            (f\cdot1)(x)&=f(1\cdot x)&&\text{definition}\\
                      &=f(x)&&M\text{ is a }R\text{-module}.\\
          \end{align*}
\end{enumerate}
Result follows.
\end{proof}
\subsection*{Problem 3.3}
Let $f\in\Hom_{\Z}(M,N)$. Prove that $f\in\Hom_{R}(M,N)$ if and only if $rf=fr$ for
any $r\in R$ with the actions defined above.
\subsubsection*{Solution}
\underline{($\implies$):} Let $f\in\Hom_{R}(M,N)$ be given. Then we know, in
general, for any $r\in R, m,m'\in M$, \[f(m+rm')=f(m)+rf(m')\] since $f$ is a
R-module homomorphism. In particular, let $m=0$, then
\begin{align*} f(rm')=rf(m')\ \forall m'\in M\end{align*} as desired.
\\
\underline{($\impliedby$):} We show that $\forall r\in R,f\in\Hom_{\Z}(M,N),\ m,m'\in M$ we
have the property that $f(m+rm')=f(m)+rf(m')$ which implies $f$ is a $R$-module map.
\begin{align*}
  f(m+rm')&=f(m)+f(rm')&&f\in\Hom_{\Z}(M, N)\\
  &=f(m)+rf(m')&&\text{assumption}.
\end{align*}
Result follows.

\section*{Problem 4}
Let $M$ be a $R$-module for a commutative ring $R$. Prove that the map
\[M\to\Hom_{R}(R,M),\quad m\mapsto(f:R\to M, r\mapsto rm)\]
is an isomorphism of $R$-modules, where the $R$-module structure of
$\Hom_{R}(R,M)$ is given by Q3(1).
\subsubsection*{Solution}
\begin{proof}
  We call the map above $\phi$, and show that $\phi$ is a $R$-module homomorphism that
  is both injective and surjective.

\underline{$R$-module Homomorphism:}

\begin{align*}
  \phi(m+rm')(s) &=s(m+rm')&&\text{definition}\\
              &=sm+s(rm')&&s\in R,\text{$M$ is $R$-module}\\
              &=sm+r(sm')&&\text{$M$ is $R$-module, $R$ commutative}\\
              &=\phi(m)s+r(\phi(m')s)\\
              &=\phi(m)s+(r\phi(m'))s&&\text{Q3(1)}.
\end{align*}

\underline{Injectivity:}
\begin{align*}
\ker(\phi)&=\{m\in M|(r\mapsto rm)\text{ is the zero map}\}\\
  &=\{m\in M|rm=0_{M}\ \forall r\in R\}
\end{align*}

I claim that $\ker(\phi)={0}$. Otherwise, suppose any
$0\ne x\ni M, x\in\ker(\phi)\implies r\cdot x=0$ for any $r\in R$. However,
$1_{R}\cdot x=x\ne 0_{M}$, a contradiction. Therefore the kernel is trivial and $\phi$ is
injective.\medskip

\underline{Surjectivity:}
For any $f\in\Hom_{R}(R,M)$, we have
\begin{align*}
f(r) = f(r\cdot1)=r\cdot f(1).
\end{align*}
Then let $f(1)=m\in M$, we have $f(r)=rm$, so $f = (r\mapsto rm)=\phi(m)$. Thus $\phi$ is surjective.
\end{proof}

\section*{Problem 5}
Let $R$ be commutative. Prove that $R\cong \End_{R}(R)$ as rings, where $R$ is a
$R$-module via left multiplication.
\subsubsection*{Solution}
From Q4, we already know that $R\cong\Hom_{R}(R,R)=\End_{R}(R)$ as modules, hence
they are isomorphic as abelian groups. We just need to show that multiplication
is preserved in this map.\medskip

Consider the same map $\phi:R\to\End_{R}(R),r\mapsto(r_{1}\mapsto rr_{1})$. We show that
$\phi(rs)x=[\phi(r)\phi(s)]x$ for any $r,s,x\in R$.
\begin{align*}
  \phi(rs)(x)&=rsx\\
          &=r(sx)\\
          &=\phi(r)(\phi(s)(x))\\
          &=[\phi(r)\phi(s)](x)&&\text{multiplication is composition in $\End_{R}(R)$}.
\end{align*}
Since this R-module isomorphism preserves multiplication, it is also a ring isomorphism.

Now we can conclude the stronger result that $R^{op}\cong\End_{R}(R)$, by noticing
that if $R$ is commutative, then $R^{op}\cong R$ as rings via the canonical map
$\varphi:r^{op}\mapsto r$. We already know this map is bijective (set-theoretically
isomorphic), and addition is automatically preserved by definition, now the
multiplication is preserved by commutativity of $R$:
\begin{align*}
a^{op}b^{op}=(ba)^{op}=(ab)^{op}\text{ (commutativity) }\implies \varphi(a^{op}b^{op})=ab
\end{align*}
for any $a, b\in R$. Therefore, we have $R^{op}\cong R\cong\End_{R}(R)$ by the isomorphism $\varphi\circ\phi$.
\end{document}
