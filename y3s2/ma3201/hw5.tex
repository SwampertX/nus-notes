\documentclass{article}

\newcommand{\myname}{Tan Yee Jian (A0190190L)}
\newcommand{\mytitle}{MA3201 Homework 5}
\title{\mytitle}
\author{\myname}
\date{\today}

\usepackage[a4paper, total={6in, 9.7in}]{geometry}
\usepackage[utf8]{inputenc}
\usepackage[T1]{fontenc}
\usepackage{textcomp}
\usepackage{amsmath, amssymb, amsthm}
\theoremstyle{plain}
% \usepackage[outputdir=tmp]{minted}
\usepackage{lmodern}
\usepackage{fancyhdr}
\usepackage{lastpage}
\pagestyle{fancy}
\fancyhf{}
% \rhead{Page \thepage/\pageref{LastPage}}
\rhead{Page \thepage}
\lhead{\myname}
\chead{\mytitle}

\usepackage{tocloft}
\usepackage[thinc]{esdiff}
\renewcommand{\thesection}{Question \arabic{section}}
\renewcommand{\thesubsection}{Part \arabic{section}(\roman{subsection})}
\renewcommand{\thesubsubsection}{Solution}
\cftsetindents{subsection}{1.5em}{4.5em}
\cftsetindents{subsubsection}{3.8em}{5.5em}

\newcommand{\pmat}[1]{ \begin{pmatrix}#1\end{pmatrix} }
\newcommand{\seqn}[1]{(#1)^\infty_{n=1}}
\newcommand{\seqk}[1]{(#1)^\infty_{k=1}}
% (series term): returns a series with counter n=1 to \infty.
\newcommand{\infsrsn}[1]{\sum\limits^\infty_{n=1}#1}
\newcommand{\infsrsk}[1]{\sum\limits^\infty_{k=1}#1}
\newcommand{\R}{\mathbb{R}}
\newcommand{\N}{\mathbb{N}}
\newcommand{\Q}{\mathbb{Q}}
\newcommand{\Z}{\mathbb{Z}}
\newcommand{\C}{\mathbb{C}}
\newcommand{\F}{\mathbb{F}}
\newcommand{\cmm}{C(M_1,M_2)}
\newcommand{\met}[1]{\langle M_{#1},\rho_{#1}\rangle}
\newcommand{\ntoinf}{\limits_{n\to\infty}}
\newcommand{\ktoinf}{\limits_{k\to\infty}}
% \newcommand{\onetoinf}[]{^\infty_{n=1}}
\newcommand{\limn}[1]{\lim\ntoinf #1}
\newcommand{\limk}[1]{\lim\ktoinf #1}

\DeclareMathOperator{\spn}{span}
\DeclareMathOperator{\diam}{diam}

\newtheorem{claim}{Claim}
\newtheorem*{claim*}{Claim}

\begin{document}
\maketitle
% \tableofcontents
\section{}
\subsection{}
\subsubsection{}
\begin{claim*}
If $e$ is central idempotent, so is $(1-e)$.
\end{claim*}
\begin{proof}[Proof of claim]
  \begin{align*}
    &{(1-e)}^{2}=1-e-e+e^{2}=1-e\\
    &(1-e)r=1r-er=r-re=r(1-e).
  \end{align*}
\end{proof}
\begin{proof}
With the claim above, it suffices to show for any central idempotent $e\in R$,
$eM$ is a submodule of $M$. Let $e$ be given, then we show the submodule
criterion: for any $m_{1},m_{2}\in M, r\in R$,
\begin{align*}
  em_{1}+r\cdot(em_{2})
  &=em_{1}+(re)m_{2}\\
  &=em_{1}+(er)m_{2}&&(er=re)\\
  &=em_{1}+e\cdot(rm_{2})\\
  &=e(m_{1}+rm_{2})\\
  &\in eM&&(RM\subset M).
\end{align*}
Since $e\cdot0=0\in eM\ne\emptyset$, $eM$ is a submodule of $M$.
\end{proof}
\subsection{}
\begin{proof}
  We first show $eM$ and $(1-e)M$ have trivial intersection. Suppose
  $x=em=(1-e)m'$ for some $m, m'\in M$, then
  \begin{align*}
    e(x)&=e(em)=em=x\\
    \text{but } e(x) &= e(1-e)m' = em' - em' = 0\\
  \end{align*}
  Combining, we have that any $x\in eM\cap(1-e)M$ must be $0$, hence the intersection
  is trivial.
  Consider the map
  \[\phi:M\to eM\oplus(1-e)M,\ m\mapsto(em, (1-e)m).\]
  We show this is an isomorphism. Let $e$ be central idempotent,
  $m_{1},m_{2}\in M, r\in R$.

  \underline{Homomorphism:}
  \begin{align*}
    \phi(m_{1}+rm_{2})
    &=(e(m_{1}+rm_{2}), (1-e)(m_{1}+rm_{2}))\\
    &=(em_{1}+e(rm_{2}), (1-e)m_{1}+(1-e)(rm_{2}))\\
    &=(em_{1}, (1-e)m_{1}) + ((re)m_{2}, [r(1-e)]m_{2})
    &&\text{direct sum, central idempotence}\\
    &=\phi(m_{1}) + r(em_{2}, (1-e)m_{2})&&\text{direct sum of modules}\\
    &=\phi(m_{1}) + r\phi(m_{2}).
  \end{align*}

  \underline{Injectivity:}
  \begin{align*}
    \ker\phi=\{m\in M|em=0=(1-e)m\in M\}
  \end{align*}

  For any $m\in \ker\phi$, $em=m-em\implies m=0$. Therefore the kernel is trivial.

  \underline{Surjectivity:}
  Let $(em,(1-e)m')\in em\oplus(1-e)M$ be given.
  Note that $e(1-e)=e-e^{2}=0=(1-e)e$. Then
  \begin{align*}
    &\phi(em+(1-e)m')\\
    =&(e^{2}m+e(1-e)m', (1-e)em+{(1-e)}^{2}m')\\
    =&(em, (1-e)m').
  \end{align*}

  Therefore $M\cong eM\oplus(1-e)M$.
\end{proof}
\section{}
\begin{proof}
  Denote the map as $\phi$. We check for homomorphism and its kernel.

\underline{R-module homomorphism:} We check $\phi(m+rn)=\phi(m)+r\phi(n)$ for any
$m,m'\in M, r\in R$.

For simplicity, we write
\[(m+A_{1}M,\ m+A_{2}M,\ldots,\ m+A_{n}M)\text{ as }(m+A_{i}M).\]

Then
\begin{align*}
  \phi(m+rm')
  &=(m+rm'+A_{i}M)\\
  &=(m+A_{i}M) + (rm'+A_{i}M)\\
  &=\phi(m) + r\cdot(m'+A_{i}M)&&(A_{i}\text{ is an ideal for all $i$ })\\
  &=\phi(m) + r\cdot\phi(m').
\end{align*}

On the other hand, the kernel:
\begin{align*}
  \ker\phi&=\{M|m+A_{i}M=A_{i}M, i=1,\ldots,n\}\\
  &=\{m\in M|\bigwedge_{i=1}^{n}(m\in A_{i}M)\}\\
  &=A_{1}M\cap A_{2}M\cap\cdots\cap A_{n}M.&&(\forall i,\ A_{i}M\subset M)
\end{align*}
\end{proof}

\section{}
Let $\phi:M\to M/A_{1}M\oplus M/A_{2}M\oplus\ldots\oplus M/A_{n}$ be the map as defined
in the previous question.
\\
We only show the $n=2$ case, and by properties of direct sum (in particular,
cartesian product), we can repeatedly apply this process till the $n$ desired
and show what the question asked for, as long as we have that
$A_{1}A_{2}\ldots A_{n-1}+A_{n}=R$.
\begin{claim} % Claim 1
  Given $A_{i}+A_{j}=R$ for any $i\ne j$, then $A_{1}A_{2}\ldots A_{n-1}+A_{n}=R$.
\end{claim}
\begin{proof}[Proof of Claim 1]
  It then sufficies to just show that $A_{1}\ldots A_{n-1}+A_{n}\ni1$. Let
  $x_{k}\in A_{k}$ be such that $x_{k} + x_{n}=1$. Then,
  \begin{align*}
    1&=(x_{1}+x_{n})(x_{2}+x_{n})\ldots(x_{n-1}+x_{n})\\
       &\in x_{1}x_{2}\ldots x_{n-1} + A_{n}\\
       &\in A_{1}A_{2}\ldots A_{n-1} + A_{i}
  \end{align*}
  as desired.
\end{proof}
\begin{claim} % Claim 2
  Given $A_{1}+A_{2}=R$, $(A_{1}A_{2})M=A_{1}M\cap A_{2}M$.
\end{claim}
\begin{proof}[Proof of Claim 2]
  Recall in Homework 1, we showed if two ideals $I,J$ of a ring $R$ such that
  $I+J=R$, then $IJ=I\cap J$. Therefore,
  \begin{align*}
    A_{1}A_{2}M &= (A_{1}\cap A_{2})M\\
    &=\{(\sum_{\text{finite}}rm) \in M|r\in A_{1}\wedge r\in A_{2}\}\\
    &=\{(\sum_{\text{finite}}rm) \in M|r\in A_{1}\}\cap\{(\sum_{\text{finite}}rm) \in M|r\in A_{2}\}\\
    &=A_{1}M\cap A_{2}M.
  \end{align*}
\end{proof}
\begin{claim} % Claim 3
  $\phi:M\to M/A_{1}M\oplus M/A_{2}M$ is surjective, given $A_{1},A_{2}$ are ideals such that $A_{1}+A_{2}=R$.
\end{claim}
\begin{proof}[Proof of Claim 3]
  It suffices to show that $(0,1),(1,0)$ are in the image of $\phi$. Since
  $A_{1}+A_{2}=R$, let $a_{1}+a_{2}=1\implies a_{1}m+a_{2}m=m$ for any $m\in M$,
  where $a_{1}\in A_{1},\ a_{2}\in A_{2}$. Then
  \begin{align*}
    \phi(a_{1}m)&=(a_{1}m+A_{1}M,\ a_{1}m+A_{2}M)\\
    &=(A_{1}M,\ 1\cdot m - a_{2}m+A_{2}M)&&(a_{1}m=m-a_{2}m)\\
    &=(0,1)\in M/A_{1}M \oplus M/A_{2}M\\
    \phi(a_{2}m)&=(a_{2}m+A_{1}M,\ a_{2}m+A_{2}M)\\
    &=(1\cdot m-a_{1}m + A_{1}M,\ A_{2}M)&&(a_{2}m=m-a_{1}m)\\
    &=(1,0)\in M/A_{1}M \oplus M/A_{2}M.
  \end{align*}
  Thus $\phi$ is surjective.
\end{proof}
Finally, by the first isomorphism theorem,
\begin{align*}
  M/A_{1}M\oplus M/A_{2}M &\cong M/\ker\phi&&(\ker\phi\text{ is surjective, Claim 3})\\
  &= M/(A_{1}M\cap A_{2}M)&&(\text{Q2})\\
  &= M/(A_{1}A_{2})M &&(\text{Claim 2}).
\end{align*}
Together with the condition to repeat in Claim 1, we have the results desired.

\section{}
\begin{proof}
\underline{$\implies$:}
Let $M$ be an Artininan $R$-module, $N$ a submodule of $M$. Let $N_{1}\supset N_{2}\supset\ldots$
be a descending chain of submodules of $N$. This is a descending chain of
submodules of $M$ too so it stabilizes, therefore $N$ is Artinian. By the 4th
Isomorphism Theorem, any descending chain of $M/N$ can be written as
$M_{1}/N\supset M_{2}/N\supset\ldots$, where $M_{1}\supset M_{2}\supset\ldots$ is a descending chain in $M$. Since
the descending chain $M_{i}$ stabilizes, so must $M_{i}/N$, and therefore $M/N$
is also Artinian.

\underline{$\impliedby$:}
\\
Take any submodule $N$ of $M$, and we know $N$ and $N/M$ are Artinian. Then given
any descending chain in $M$,
\begin{equation}
M_{1}\supset M_{2}\supset\ldots
\end{equation}
consider the chains \begin{equation}
M_{1}\cap N\supset M_{2}\cap N\supset\ldots
\end{equation} and
\begin{equation}
M_{1}+ N\supset M_{2}+ N\supset\ldots
\end{equation} in $N$ and $M/N$ respectively. We will show that (1) must
stabilize. Since $N$ and $M/N$ are Artinian, let $n$ be large enough such that
both (2) and (3) stabilize at the $n$-th term, ie. $M_{n}\cap N=M_{n+1}\cap N=\ldots$ and
$M_{n}+ N=M_{n+1}+ N=\ldots$. We want to show that (1) must stabilize at $n$ too, in
other words, $M_{n}=M_{n+1}=\ldots$. It suffices to just show $M_{n} \subset M_{n+1}$ since
we already have the other inclusion. \medskip

Let $m_{n}$ be from $M_{n}$. Then
\begin{align*}
  &m_{n}+N\in M_{n}+N = M_{n+1} + N\\
  \implies& m_{n}=m_{n+1} + n&&\text{for some $m_{n+1}\in M_{n+1}$ and $n\in N$}.
\end{align*}
  Now, $M_{n}\supset M_{n+1}\implies n=m_{n}-m_{n+1}\in M_{n}$. Therefore,
  \begin{align*}
    &m_{n}-m_{n+1}\in M_{n}\cap N=M_{n+1}\cap N\\
    \implies&m_{n}-m_{n+1}\in M_{n+1}\\
    \implies&m_{n}\in M_{n+1}
  \end{align*}
  Therefore, $M_{n}\subset M_{n+1}$ as desired. Since any descending chain of
  submodules in $M$ stablilizes, $M$ is Artinian.\end{proof}

\section{}
\begin{proof}
We consider the ring $F[G]$ as a \textbf{left}-$F[G]$ module. Then $F[G]$ is naturally a
$F$-module, hence a $F$-vector space, by considering the restricted action of
$F\cdot1_{G}$, where $1_{G}$ is the identity element in the group $G$.\medskip

Now since $G$ is finite, any element in $F[G]$ can be expressed by a n-tuple,
where $n=|G|$, hence it is a $n$-dimensional vector space. Any proper inclusion
of submodules (subspaces) of $F[G]$, say $M_{1}\subset M_{2}$ must have an increase of
dimension, and thus any ascending/descending chain of submodules must stabilize
due to finite dimensions. In particular, any ascending/descending chain of
\textbf{left} ideals in $F[G]$ (thus $F[G]$-submodules) must stabilize.\medskip

The proof is identical considering $F[G]$ as a \textbf{right }$F[G]$-module,
therefore we will have that any ascending/descending chain of \textbf{right}
ideals must stabilize as well. We can then conclude that any
ascending/descending chain of (\textbf{two-sided}) ideals of $F[G]$ must also
stabilize, thus $F[G]$ must be both Artinian and Noetherian as a ring.
\end{proof}
\end{document}
