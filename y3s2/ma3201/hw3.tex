\documentclass{article}

\newcommand{\myname}{Tan Yee Jian (A0190190L)}
\newcommand{\mytitle}{MA3201 Homework 3}
\title{\mytitle}
\author{\myname}
\date{\today}

\usepackage[a4paper, total={6in, 9.7in}]{geometry}
\usepackage[utf8]{inputenc}
\usepackage[T1]{fontenc}
\usepackage{textcomp}
\usepackage{amsmath, amssymb, amsthm}
\theoremstyle{plain}
% \usepackage[outputdir=tmp]{minted}
\usepackage{lmodern}
\usepackage{fancyhdr}
\usepackage{lastpage}
\pagestyle{fancy}
\fancyhf{}
% \rhead{Page \thepage/\pageref{LastPage}}
\rhead{Page \thepage}
\lhead{\myname}
\chead{\mytitle}

% figure support
\usepackage{import}
\usepackage{xifthen}
\pdfminorversion=7
\usepackage{pdfpages}
\usepackage{transparent}
\newcommand{\incfig}[1]{%
  \def\svgwidth{\columnwidth}
  \import{./figures/}{#1.pdf_tex}
}

\newtheorem{thm}{Theorem}[section]
\newtheorem{crl}{Corollary}[thm]
\newtheorem{lemma}{Lemma}[thm]
\newtheorem*{lemma*}{Lemma}
\newtheorem{note}{Note}[thm]
\newtheorem{defn}{Definition}[section]
\newtheorem{ex}{Example}[section]
\newtheorem{prop}{Proposition}[section]
\newtheorem{obs}{Observation}
\newtheorem{claim}{Claim}

\newcommand{\pmat}[1]{ \begin{pmatrix}#1\end{pmatrix} }
\newcommand{\seqn}[1]{(#1)^\infty_{n=1}}
\newcommand{\seqk}[1]{(#1)^\infty_{k=1}}
% (series term): returns a series with counter n=1 to \infty.
\newcommand{\infsrsn}[1]{\sum\limits^\infty_{n=1}#1}
\newcommand{\infsrsk}[1]{\sum\limits^\infty_{k=1}#1}
\newcommand{\R}{\mathbb{R}}
\newcommand{\N}{\mathbb{N}}
\newcommand{\Q}{\mathbb{Q}}
\newcommand{\Z}{\mathbb{Z}}
\newcommand{\C}{\mathbb{C}}
\newcommand{\F}{\mathbb{F}}
\newcommand{\cmm}{C(M_1,M_2)}
\newcommand{\met}[1]{\langle M_{#1},\rho_{#1}\rangle}
\newcommand{\ntoinf}{\limits_{n\to\infty}}
\newcommand{\ktoinf}{\limits_{k\to\infty}}
% \newcommand{\onetoinf}[]{^\infty_{n=1}}
\newcommand{\limn}[1]{\lim\ntoinf #1}
\newcommand{\limk}[1]{\lim\ktoinf #1}

\DeclareMathOperator{\spn}{span}
\DeclareMathOperator{\diam}{diam}
\DeclareMathOperator{\chr}{char}

% Question: section
% Solution: subsection, subsubsection
% Hence remove numberings
% \setcounter{secnumdepth}{0}


\begin{document}
\maketitle
\section*{Problem 1}
Let $P$ be a prime ideal of a commutative ring $R$ (with $1\ne0$). Let $I,J$ be
two ideals of $R$ such that $I\cap J\subset P$. Prove that either $I\subset P$ or $J\subset P$.
\subsection*{Solution}
\begin{proof}
  For the sake of contradiction, suppose there is some $i\in I$ and $j\in J$ such
  that both are not in $P$. Then since both are in ideals, $ij\in I\cap J\subset P$. Since $P$
  is a prime ideal, we have $ij\in P\implies i\in P \vee j\in P$, a contradiction.
\end{proof}

\section*{Problem 2}
Let $\Z[i]$ be the ring of Gaussian integers. Let $I\subset\Z[i]$ be a non-zero ideal.
Prove that the quotient ring $\Z[i]/I$ is a finite set.
\subsection*{Solution}
Recall that $\Z[i]$ is an Euclidean domain. Since it is in particular a
principal ideal domain, let $I=(r)$. Then we carry out Euclidean division on any
arbitrary element of $R$, say $\alpha$, we have
\begin{align*}
\alpha = qr + \beta &&N(\beta)<N(r).
\end{align*}
If $\beta=0$, then $\alpha\in(r)$ is in the kernel. Otherwise $\beta>0$, then there are only
finitely many $\beta$ such that $N(\beta)=a^{2}+b^{2}<N(r), \{a,b\}\subset\Z$. Therefore, its
image under the quotient map must also be finite.

\section*{Problem 3}
Let $p\in\Z$ be a positive prime. We define
\[\Phi_{p}(x)=\frac{x^{p}-1}{x-1}=x^{p-1}+x^{p-2}+\cdots+1\in\Z[x].\]
Prove that $\Phi_{p}(x)$ is irreducible.
\subsection*{Solution}
\begin{lemma*}
  Let $R$ be a commutative ring. If $f(x)\in R[x]$ is reducible, then $f(x+1)$ is
  reducible.
\end{lemma*}
\begin{proof}[Proof of Lemma]
  Let $f(x)=a(x)b(x)$ where $a(x), b(x)$ are not units in $R[x]$. Then
  $a(x+1), b(x+1)$ cannot be units since their highest powers are preserved,
  according to the Binomial Theorem. Therefore $f(x+1)=a(x+1)b(x+1)$ witnesses
  the reducibility of $f(x+1)$.
\end{proof}
Now we use the contrapositive of the lemma to show that $\Phi_{p}(x+1)$ is
irreducible, following the example on the textbook.
\begin{proof}[Proof of problem]
  \begin{align*}
    \Phi_{p}(x+1)&=\frac{(x+1)^{p}-1}{(x+1)-1}\\
              &=\frac{1}{x}(x^{p}+px^{p-1}+\frac{p(p-1)}{2}x^{p-2}+\ldots+px+1-1)\\
              &=x^{p-1}+px^{p-2}+\frac{p(p-1)}{2}x^{p-3}+\ldots+p.
  \end{align*}
  Since $p\in\Z$ is prime, then all binomial coefficients of the form
  $\binom{p}{k}, 1\le k<p$ must be a multiple of $p$. By Eisenstein's criteria,
  $a^{p-1}=0, a^{p-2},\ldots,a^{0}\in(p)$ but $a_{0}=p\notin(p^{2})$ gives us that
  $\Phi_{p}(x+1)$ is irreducible. By the contrapositive of the lemma above,
  $\Phi_{p}(x)$ must also be irreducible.
\end{proof}

\section*{Problem 4}
\subsection*{Problem 4.1}
Let $R$ be an integral domain. Prove that the characteristic of $R$ is a prime
number or $0$.
\subsubsection*{Solution}
\begin{proof}
  For the sake of contradiction, suppose the characteristic of $R,\ c$ is positive
  but not a prime.\medskip

  \underline{Case 1: $c=1$}

  Then $1=0\implies R$ is a trivial ring, $1\times1=0$ is a zero divisor, a contradiction.\medskip

  \underline{Case 2: $\chr R$ is composite}

  Then $c=ab$ for some $a, b\in\N$ where both $a, b$ are not $1$. Write
  $a_{R}=\sum^{a}1_{R}, b_{R}\sum^{b}1_{R}$, then we have
  \begin{align*}
    ab_{R} &= \underbrace{1_{R}+\ldots+1_{R}}_{ab\ \text{times}}\\
    &= \overbrace{(\underbrace{1_{R}+\ldots+1_{R}}_{a\ \text{times}})+\ldots+(\underbrace{1_{R}+\ldots+1_{R}}_{a\ \text{times}})}^{b\ \text{times}}\\
    &=b_{R}\cdot a_{R}=0_{R}.
  \end{align*}
  Both $a_{R}, b_{R}$ cannot be $0_{R}$ since otherwise, it would be the
  characteristic of $R$. This means both are nonzero and thus are zero divisors,
  contradicting the integral domain assumption.

\end{proof}
\subsection*{Problem 4.2}
Let $R$ be a field with $1\ne0$. Prove that the additive group $R$ and the
multiplicative group $R*$ are never isomorphic.
\subsubsection*{Solution}
\begin{proof}
  It is clear that $R$ can never be finite, since otherwise, $|R|=|R^{*}|+1$ cannot
  form a bijection.\medskip

  We just consider the case where $R$ is an infinite field. For the sake of
  contradiction, suppose $\phi:R\to R^{*}$ be an isomorphism. In particular, since it
  is a homomorphism, it must be that $\phi(0)=1$. Let the characteristic of $R$ be $c$,
  and we split by cases:\medskip

  \underline{Case 1: $c=2$}

  Then $1+1=0\implies 1=-1$. We have
  \begin{align*}
    \phi(1_{R})^{2}&=\phi(1_{R})\cdot\phi(1_{R})=\phi(1_{R}+1_{R})\\
            &=\phi(0_{R})&& (1_{R}=-1_{R}\implies2_{R}=0_{R})\\
              &=1_{R} && (\text{group homomorphisms preserve identity}).
  \end{align*}

  Solving the quadratic equation $x^{2}=1_{R}$ in $R$, we have $\phi(1_{R})=1_{R}$ uniquely (since the
  other solution $-1_{R}=1_{R}$). This means $\phi(0_{R})=\phi(1_{R})=1_{R}$, violating the bijectivity
  (in particular, injectivity) of $\phi$. A contradiction.\medskip

  \underline{Case 2: $c\ne2$}

  Since $\phi$ is surjective, there exists some $x\in R$ such that $\phi(x)=-1_{R}\in R^{*}$. Then
  \begin{align*}
    \phi(x)=-1_{R}&\implies\phi(x)^{2}=(-1_{R})^{2}\\
           &\implies\phi(2_{R}x)=1_{R}\\
           &\implies 2_{R}x=0_{R}&& (\text{since }\ker\phi=\{0_{R}\})\\
           &\implies x=0_{R} && (\text{characteristic}\ne2\implies2_{R}\ne0_{R})\\
           &\implies \phi(x)=1_{R} &&\text{but }\phi(0_{R})=1_{R}.
  \end{align*}
  Contradiction.
\end{proof}

\section*{Problem 5}
Let $R$ be an integral domain. Given $R$ is an Artinian ring (thus fulfills the
Descending Chain Condition), show that $R$ is a field.
\subsection*{Solution}
\begin{proof}
For any $0\ne a\in R$, we can have a chain of ideals,
\[(a)\supseteq(a^{2})\supseteq\cdots\supseteq(a^{n})\supseteq\cdots\] and there exists a $k\ge1$ such that
$(a^{k})=(a^{k+1})$. We thus have a unit $b\in R$ such that
\begin{align*}
  a^{k}=ba^{k+1}&\implies a^{k}\cdot1=a^{k}(ab)&& (commutative)\\
                  &\implies ab=1 && (a\ne0, \text{ integral domain}).
\end{align*}
Therefore every non-zero element $a\in R$ is invertible $\implies R$ is a field.
\end{proof}
\end{document}
