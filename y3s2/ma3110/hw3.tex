\documentclass{article}

\newcommand{\myname}{Tan Yee Jian (A0190190L)}
\newcommand{\mytitle}{MA3110 Homework 3}
\title{\mytitle}
\author{\myname}
\date{\today}

\usepackage[a4paper, total={6in, 9.7in}]{geometry}
\usepackage[utf8]{inputenc}
\usepackage[T1]{fontenc}
\usepackage{textcomp}
\usepackage{amsmath, amssymb, amsthm}
\theoremstyle{plain}
% \usepackage[outputdir=tmp]{minted}
\usepackage{lmodern}
\usepackage{fancyhdr}
\usepackage{lastpage}
\pagestyle{fancy}
\fancyhf{}
% \rhead{Page \thepage/\pageref{LastPage}}
\rhead{Page \thepage}
\lhead{\myname}
\chead{\mytitle}

\newcommand{\pmat}[1]{ \begin{pmatrix}#1\end{pmatrix} }
\newcommand{\seqn}[1]{(#1)^\infty_{n=1}}
\newcommand{\seqk}[1]{(#1)^\infty_{k=1}}
% (series term): returns a series with counter n=1 to \infty.
\newcommand{\infsrsn}[1]{\sum\limits^\infty_{n=1}#1}
\newcommand{\infsrsk}[1]{\sum\limits^\infty_{k=1}#1}
\newcommand{\R}{\mathbb{R}}
\newcommand{\N}{\mathbb{N}}
\newcommand{\Q}{\mathbb{Q}}
\newcommand{\Z}{\mathbb{Z}}
\newcommand{\C}{\mathbb{C}}
\newcommand{\F}{\mathbb{F}}
\newcommand{\cmm}{C(M_1,M_2)}
\newcommand{\met}[1]{\langle M_{#1},\rho_{#1}\rangle}
\newcommand{\ntoinf}{\limits_{n\to\infty}}
\newcommand{\ktoinf}{\limits_{k\to\infty}}
% \newcommand{\onetoinf}[]{^\infty_{n=1}}
\newcommand{\limn}[1]{\lim\ntoinf #1}
\newcommand{\limk}[1]{\lim\ktoinf #1}
\newcommand{\bigint}{\displaystyle\int}

\DeclareMathOperator{\spn}{span}
\DeclareMathOperator{\diam}{diam}

\begin{document}
\maketitle
\section*{Problem H1}
Let $F(x)=\bigint_{x}^{x^{3}}\frac{\cos(t^{2})}{t}dt$ for $x\ge 1$. Find $F'(x)$ for $x\ge 1$.
\subsection*{Solution}
\begin{proof}
Let $f(x)=\displaystyle\frac{\cos(x^{2})}{x}, G(x)=\bigint_{1}^{x}f(t)dt$. Then
by the Fundamental Theorem of Calculus I, since $f$ is continuous on $[1,\infty)$, we
have $G'(x)=f(x)$ for any $x\in[1,\infty)$. Then $\forall x\ge 1$,
\begin{align*}
F(x)&=\bigint_{x}^{x^{3}}\frac{\cos(t^{2})}{t}dt\\
&=\bigint_{1}^{x^{3}}\frac{\cos(t^{2})}{t}dt - \bigint_{1}^{x}\frac{\cos(t^{2})}{t}dt\\
&=G(x^{3})-G(x).\\
  \therefore F'(x)&=3x^{2}G'(x^{3})-G'(x) &&\text{chain rule}\\
  &=3x^{2}f(x^{3})-f(x) &&\text{FTC(I)}\\
  &=\frac{3}{x}\cos(x^{6})-\frac{\cos(x^{2})}{x}.
\end{align*}
\end{proof}

\section*{Problem H2}
Let $f$ and $g$ be continuous functions on $[a,b]$ and let $H:[a,b]\to\R$ be
defined
by \[H(x)=\left(\int_{a}^{x}f(t)dt\right)\left(\int_{x}^{b}g(t)dt\right)\quad\text{for all
  } x\in[a,b].\]
Prove that there exists $c\in(a,b)$ such that
\[g(c)\int_{a}^{c}f(x)dx=f(c)\int_{c}^{b}g(x)dx.\]
\subsection*{Solution}
\begin{proof}
We notice that
\begin{align*}
H(a)=\left(\int_{a}^{a}f(t)dt\right)\left(\int_{a}^{b}g(t)dt\right)=0=
\left(\int_{a}^{b}f(t)dt\right)\left(\int_{b}^{b}g(t)dt\right)=H(b).
\end{align*}
Let $F(x)=\bigint_{a}^{x}f$ and $G(x)=\bigint_{b}^{x}g$. We rewrite $H$ into
products of $F$ and $G$:
\begin{align*}
H(x)&=\left(\int_{a}^{b}f(t)dt\right)\left(\int_{x}^{b}g(t)dt\right)\\
  &=(F(x))(-G(x))
\end{align*}
Since $f$ and $g$ are continuous, $F$ and $G$ are differentiable on $[a,b]$ by
Tutorial 6 Question 3, and $F'=f, G'=g$ on $(a,b)$. Thus by Rolle's theorem,
there exists $c\in(a,b)$ such that $H'(c)=0$. Since $F$ and $G$ are differentiable
on $(a,b)$, we can use the product rule to differentiate $H$:
\begin{align*}
  H'(c)=0&=F'(c)(-G(c))+F(c)(-G'(c))\\
  0&=f(c)\int_{c}^{b}g(t)dt+\left(\int_{a}^{c}f(t)dt\right)(-g(c))
\end{align*}
Therefore
\[f(c)\int_{c}^{b}g(t)dt=g(c)\int_{a}^{c}f(t)dt,\quad \text{for some }c\in(a,b).\]
\end{proof}

\section*{Problem H3}
Suppose that $f$ is continuous on $[a,b]$. Prove that there exists $c\in(a,b)$
such that \[\int_{a}^{b}f=f(c)(b-a).\]
\subsection*{Solution}
\begin{proof}
  Define \[F(x)=\int_{a}^{x}f,\quad x\in[a,b].\]We note that $f$ is continous on $[a,b]$
  implies that $F$ is differentiable on $[a,b]$. Therefore we can apply Mean
  Value Theorem to $F$ on the interval $[a,b]$: there exists some $c\in(a,b)$ such
  that
  \begin{align*}
    F'(c)&=\frac{F(b)-F(a)}{b-a}&&\text{Mean Value Theorem}\\
    \therefore f(c)&=\frac{1}{b-a}\left(\int_{a}^{b}f-\int_{a}^{a}f\right) &&\text{FTC(I), definiton}\\
    &=\frac{1}{b-a}\left(\int_{a}^{b}f\right)
  \end{align*}
  Moving terms, we have \[\int_{a}^{b}f=f(c)(b-a).\]
\end{proof}

\section*{Problem H4}
Using the Riemann integral of a suitable chosen function, find the limit
\[\lim_{n\to\infty}\frac{1}{n^{2}}\sum_{k=1}^{n}k\sin\left(\frac{\pi k^{2}}{n^{2}}\right).\]
\subsection*{Solution}
\begin{proof}
Let \[f(x)=x\sin(\pi x^{2}),\ P_{n}=\{0,\frac{1}{n},\ldots,\frac{n}{n}\},\
\xi^{(n)}=P_{n}\setminus\{0\}.\]
Then we can rewrite the summation into a Riemann sum:
\begin{align*}
  &\lim\ntoinf\sum_{k=1}^{n}\frac{1}{n}\cdot\frac{k}{n}\sin\left(\pi\left(\frac{k}{n}\right)^{2}\right)\\
  =&\lim\ntoinf S(f, P_{n})(\xi^{(n)})\\
  =&\int_{0}^{1}f&&n\to\infty\implies\|P_{n}\|\to0,\text{ Corollary 7.4.3}\\
  =&\frac{1}{2\pi}\int_{0}^{1}2\pi x\sin(\pi x^{2})dx\\
  =&\frac{1}{2\pi}[-\cos(\pi\cdot1^{2})+\cos(\pi\cdot0^{2})]=\frac{1+1}{2\pi}=\frac{1}{\pi}.
\end{align*}
\end{proof}

\section*{Problem H5}
In each part, determine if the improper integral converges. Justify your answers.
\subsection*{Part H5(i)}
\[\int_{1}^{\infty}\frac{\sin^{2}x}{x^{2}}dx.\]
\subsubsection*{Solution}
\begin{proof}
  Recall that if \[f(x)=\frac{\sin^{2}x}{x^{2}}\] is non-negative over $[1,\infty)$,
  then the improper integral converges $\iff$ $F(x)=\int_{1}^{x}f$ is bounded for all
  $x\ge 1$. First notice that for any $x\ge1$, we have
  \begin{align*}
    0\le (\sin x)^{2}\le 1 \implies 0\le f(x)=\frac{\sin^{2}x}{x^{2}}\le \frac{1}{x^{2}}.
  \end{align*}
  Which shows $f$ is nonnegative on $[1,\infty)$. Now we only need to show that
  $F(x)$ is bounded above. By Theorem 7.2.6(iii), we have
  \begin{align*}
    \frac{\sin^{2}x}{x^{2}}\le \frac{1}{x^{2}}
    \implies
    \int_{1}^{t}\frac{\sin^{2}x}{x^{2}}&\le \int_{1}^{t}\frac{1}{x^{2}}\\
                                    &= (-\frac{1}{t}+\frac{1}{1})\\
                                    &=1-\frac{1}{t}\\
                                    &<1&&\forall t\in[1,\infty).
  \end{align*}
  Hence the improper integral converges.
\end{proof}
\subsection*{Part H5(ii)}
\[\int_{0}^{1}\frac{x}{1-x^{2}}dx.\]
\subsubsection*{Solution}
\begin{proof}
  Note that
  \[f(x)=\frac{x}{1-x^{2}}\]
  is unbounded as $x\to1$. Let us rewrite the improper integral as
  \begin{align*}
    \int_{0}^{1}\frac{x}{1-x^{2}}dx
    &=\lim_{t\to1^{-}}-\frac{1}{2}\int_{0}^{t}\frac{-2x}{1-x^{2}}dx\\
    &=-\frac{1}{2}\lim_{t\to1^{-}}\left[\ln(1-x^{2})\right]^{x=t}_{x=0}\\
    &=-\frac{1}{2}\lim_{t\to1^{-}}(\ln(1-t^{2})-\ln(1))&&\text{ then let }t'=(1-t^{2})\to0,\\
    &=-\frac{1}{2}\lim_{t'\to0^{+}}\ln(t')\\
    &=\infty.
  \end{align*}
  Therefore the improper integral diverges.
\end{proof}
\end{document}
