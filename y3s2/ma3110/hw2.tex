\documentclass{article}

\newcommand{\myname}{Tan Yee Jian (A0190190L)}
\newcommand{\mytitle}{MA3110 Homework 2}
\title{\mytitle}
\author{\myname}
\date{\today}

\usepackage[a4paper, total={6in, 9.7in}]{geometry}
\usepackage[utf8]{inputenc}
\usepackage[T1]{fontenc}
\usepackage{textcomp}
\usepackage{amsmath, amssymb, amsthm}
\theoremstyle{plain}
% \usepackage[outputdir=tmp]{minted}
\usepackage{lmodern}
\usepackage{fancyhdr}
\usepackage{lastpage}
\pagestyle{fancy}
\fancyhf{}
% \rhead{Page \thepage/\pageref{LastPage}}
\rhead{Page \thepage}
\lhead{\myname}
\chead{\mytitle}

% figure support
\usepackage{import}
\usepackage{xifthen}
\pdfminorversion=7
\usepackage{pdfpages}
\usepackage{transparent}
\newcommand{\incfig}[1]{%
  \def\svgwidth{\columnwidth}
  \import{./figures/}{#1.pdf_tex}
}

\newtheorem{thm}{Theorem}[section]
\newtheorem{crl}{Corollary}[thm]
\newtheorem{lemma}{Lemma}[thm]
\newtheorem*{lemma*}{Lemma}
\newtheorem{note}{Note}[thm]
\newtheorem{defn}{Definition}[section]
\newtheorem{ex}{Example}[section]
\newtheorem{prop}{Proposition}[section]
\newtheorem{obs}{Observation}
\newtheorem{claim}{Claim}

\newcommand{\pmat}[1]{ \begin{pmatrix}#1\end{pmatrix} }
\newcommand{\seqn}[1]{(#1)^\infty_{n=1}}
\newcommand{\seqk}[1]{(#1)^\infty_{k=1}}
% (series term): returns a series with counter n=1 to \infty.
\newcommand{\infsrsn}[1]{\sum\limits^\infty_{n=1}#1}
\newcommand{\infsrsk}[1]{\sum\limits^\infty_{k=1}#1}
\newcommand{\R}{\mathbb{R}}
\newcommand{\N}{\mathbb{N}}
\newcommand{\Q}{\mathbb{Q}}
\newcommand{\Z}{\mathbb{Z}}
\newcommand{\C}{\mathbb{C}}
\newcommand{\F}{\mathbb{F}}
\newcommand{\cmm}{C(M_1,M_2)}
\newcommand{\met}[1]{\langle M_{#1},\rho_{#1}\rangle}
\newcommand{\ntoinf}{\limits_{n\to\infty}}
\newcommand{\ktoinf}{\limits_{k\to\infty}}
% \newcommand{\onetoinf}[]{^\infty_{n=1}}
\newcommand{\limn}[1]{\lim\ntoinf #1}
\newcommand{\limk}[1]{\lim\ktoinf #1}

\DeclareMathOperator{\spn}{span}
\DeclareMathOperator{\diam}{diam}

% Question: section
% Solution: subsection, subsubsection
% Hence remove numberings
% \setcounter{secnumdepth}{0}


\begin{document}
\maketitle
\section*{Problem 1}
Let $f:(a,b)\to\R$ be differentiable on $(a,b)$ and let $c\in(a,b)$. Prove that if
the limit $\lim_{x\to c}f'(x)=L$ exists, then $f'(c)=L$.
\subsection*{Solution}
As per the hint, we note that the difference quotient has numerator and
denominator both tend to $0$ as $x$ approaches $c$ and $f$ is differentiable on $(a,b)$:
\begin{align*}
  f'(c)&=\lim_{x\to c}\frac{f(x)-f(c)}{x-c}\\
  &=\lim_{x\to c}\frac{f'(x)-0}{1-0} && \text{(L'hopital's Rule)}\\
  &=\lim_{x\to c}f'(x)=L && \text{(assumption)}.
\end{align*}

\section*{Problem 2}
The fucntion $g:[-1,1]\to\R$ is such that $g'''$ exists on
$[-1,1], g(0)=g(1)=0,g(1)=1$ and $g'(0)=0$.
\begin{enumerate}
  \item Show that there exists $c\in(0,1)$ such that
        $\frac{g''(0)}{2!}+\frac{g'''(c)}{3!}=1$.
        \begin{proof}
          Apply Taylor's Theorem to $g$ on $[0,1]$, and let $x=1, x_{0}=0$, we
          have
          \begin{align}
            g(1)&=g(0)+g'(0)x+\frac{1}{2}g''(0)x^{2}+\frac{1}{6}g'''(c)x^{3}\\
1&=0+0x+\frac{1}{2}g''(0)(1)^{2}+\frac{1}{6}g'''(c)(1)^{3}\\
            \therefore \frac{1}{2}g''(0)+\frac{1}{6}g'''(c)=1
          \end{align}
          for some $c\in(0,1)$ as desired.
        \end{proof}
  \item Show that there exists $d\in(-1,1)$ such that $g'''(d)\geq 3$.
        \begin{proof}
          Apply Taylor's Theorem to $g$ on $[-1,0]$, and let $x=-1, x_{0}=0$, we
          have
          \begin{align}
            g(-1)&=g(0)+g'(0)x+\frac{1}{2}g''(0)x^{2}+\frac{1}{6}g'''(c')x^{3}\\
0&=0+0x+\frac{1}{2}g''(0)(-1)^{2}+\frac{1}{6}g'''(c')(-1)^{3}\\
            \therefore \frac{1}{2}g''(0)-\frac{1}{6}g'''(c')=0
          \end{align}
          for some $c'\in(-1,0)$. Taking (3)-(6), we have
          \begin{align*}
            1 = g'''(c') + g'''(c)\\
            6 = g'''(c') + g'''(c)
          \end{align*}
          Without loss of generality, we must have $g'''(c)\geq g'''(c')$ or
          $g'''(c)\leq g'''(c')$. In any case, one of them must be at least $3$.
          Therefore, let it be $d\in(-1, 1)$ and we are done.
        \end{proof}
\end{enumerate}

\section*{Problem 3}
Let $f(x)=(1+3x)^{2/3},\ x>-1/3$.
\begin{enumerate}
\item Find the values of $f'(0),f''(0)$ and $f'''(0)$.
        \begin{proof}
          \begin{align*}
            f'(x)=(2/3)(1+3x)^{-1/3}(3)=2(1+3x)^{-1/3}, &&f'(0)=2(1)=2.\\
            f''(x)=(2)(-1/3)(1+3x)^{-4/3}(3)=-2(1+3x)^{-4/3}, &&f''(0)=-2(1)=-2.\\
            f'''(x)=(-2)(-4/3)(1+3x)^{-7/3}(3)=8(1+3x)^{-7/3}, &&f'''(0)=8(1)=8.
          \end{align*}
        \end{proof}

  \item Use Taylor's Theorem to prove that for $x>-1/3,\ f(x)\le1+2x-x^{2}+4x^{3}/3$.
        \begin{proof}
          Applying Taylor's theorem to $f$ with $x_{0}=0$,
          \begin{align*}
            f(x) &=f(0)+f'(0)x+\frac{1}{2}f''(0)x^{2}+\frac{1}{6}f''(0)x^{3}+R_{3}(x)\\
            &=1+2x+\frac{1}{2}(-2)x^{2}+\frac{1}{6}8x^{3}+R_{3}(x)\\
            &=1+2x-x^{2}+\frac{4}{3}x^{3}+R_{3}(x)\\
          \end{align*}
          Where $R_{3}(x)=\frac{1}{24}f^{(4)}(c)x^{4}$ for some $c>-1/3$, and
          \[f^{(4)}(x)=(8)(-7/3)(1+3x)^{-10/3}(3)=-56(1+3x)^{-10/3}. \]
          Since $x>-1/3\implies(1+3x)^{-10/3}>0$, and $x^{4}$ is always
          non-negative, therefore the only negative coefficient forces $R_{3}(x)\le0$. We thus have
          \begin{align*}
            f(x) &=1+2x-x^{2}+\frac{4}{3}x^{3}+R_{3}(x)\\
            &\le1+2x-x^{2}+\frac{4}{3}x^{3} && (R_{3}\le0).
          \end{align*}
        \end{proof}
\end{enumerate}

\section*{Problem 4}
Let $h:[0,2]\to\R$ be defined by
\[h(x)=\begin{cases}
    4x, & x\text{ is rational}\\
    4, & x\text{ is irrational}
  \end{cases}
\] and let $P=\{0,1/2,1,3/2,2\}$. Find the upper sum $U(h,P)$ of $h$ with
respect to the partition $P$.
\subsection*{Solution}
\begin{proof}
  \begin{align*}
U(H,P)=&\sum^{4}_{i=1}(x_{i}-x_{i-1})M_{i-1} &&\text{where }M_{i-1}=\sup{f(x)|x\in[x_{i-1},x_{i}]}\\
    =&(\frac{1}{2}-0)(\sup\{4,4(\frac{1}{2}),4(0)\})\\
    &+(1-\frac{1}{2})(\sup\{4,4(1),4(\frac{1}{2})\})\\
    &+(\frac{3}{2}-1)(\sup\{4,4(\frac{3}{2}),4(1)\})\\
       &+(2-\frac{3}{2})(\sup\{4,4(2),4(\frac{3}{2})\})\\
    =&\frac{1}{2}[4+4+6+8]=2+2+3+4=11.
  \end{align*}
\end{proof}
\end{document}
