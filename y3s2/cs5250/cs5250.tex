% Created 2021-05-05 Wed 15:25
% Intended LaTeX compiler: pdflatex
\documentclass[11pt]{article}
\usepackage[utf8]{inputenc}
\usepackage[T1]{fontenc}
\usepackage{graphicx}
\usepackage{grffile}
\usepackage{longtable}
\usepackage{wrapfig}
\usepackage{rotating}
\usepackage[normalem]{ulem}
\usepackage{amsmath}
\usepackage{textcomp}
\usepackage{amssymb}
\usepackage{capt-of}
\usepackage{hyperref}
\usepackage{amsthm}
\usepackage{minted}
\author{Tan Yee Jian}
\date{\today}
\title{Cs5250}
\hypersetup{
 pdfauthor={Tan Yee Jian},
 pdftitle={Cs5250},
 pdfkeywords={},
 pdfsubject={},
 pdfcreator={Emacs 27.2 (Org mode 9.4.4)}, 
 pdflang={English}}
\begin{document}

\maketitle
\tableofcontents


\section{Week 6:}
\label{sec:org45f5a91}
\subsection{System calls}
\label{sec:org875a619}
Two ways to enter kernel: \texttt{SYSCALL/SYSENTER} or an interrupt
\subsubsection{How to do system calls?}
\label{sec:org533d333}
\begin{itemize}
\item Model Specific Registers (MSR) in x86 allows configuration of OS-specific
things like \texttt{sysenter/exit}
\begin{itemize}
\item \texttt{rdmsr, wrmsr} instructions used to move data from(to) \texttt{MSR[ECX]} to(from) \texttt{EDX:EAX}
\item Existence can be confirmed through the \texttt{cpuid} instruction
\end{itemize}
\end{itemize}
\begin{enumerate}
\item SYSCALL/SYSENTER instruction and security
\label{sec:orge44bf73}
\begin{itemize}
\item was implemented as an interrupt \texttt{int 0x80}, now using
\texttt{SYSENTER/SYSEXIT/SYSCALL} instruction
\item having an instruction prevents attacker from calling it before their code (can
only be called by kernel)
\end{itemize}
\item Difference between SYSENTER and interrupt
\label{sec:org743bd28}
Interrupts can happen even at ring 0, however SYSENTER is used to transit from
other rings to 0
\end{enumerate}
\subsubsection{Stacks}
\label{sec:org84ecd9d}
\begin{enumerate}
\item Kernel stack
\label{sec:org59b3f2f}
\begin{itemize}
\item Every thread has one
\item Stored in \texttt{task\_struct}
\item Grows toward the \texttt{thread\_info} structure
\end{itemize}
\item Intel's hardware support for stacks: Task State Segment
\label{sec:org966b933}
\url{https://wiki.osdev.org/Task\_State\_Segment}
\end{enumerate}
\subsubsection{Page Table Isolation}
\label{sec:orgad46452}
\begin{itemize}
\item Full set of kernel page tables used to coexist with user page tables, presents
a security loophole
\item Now each process's virtual memory space only save partial kernel PT
\end{itemize}
\subsubsection{What happens when we do a syscall?}
\label{sec:orgbad6e10}
More info: \texttt{arch/x86/entry/entry\_64.s:entry\_SYSCALL\_64}
\begin{enumerate}
\item save the current stack location and load the full kernel stack (pointed to by
\texttt{CR3})
\item Uses assembly to construct the \texttt{pt\_regs} struct, which stores registers
needed by \texttt{syscall}
\item Calls the service dispatcher (call the correct syscall given the sycall
number), in \texttt{arch/x86/entry/common.c:do\_syscall\_64()}
\end{enumerate}
\subsubsection{How to speed up syscalls?}
\label{sec:orgf1642fc}
\begin{enumerate}
\item Old idea: call them in userspace
\label{sec:orgb836ded}
\begin{itemize}
\item some system calls (eg. read, write) are called very often. How do we speed up?
\item Answer: execute \textbf{read-only} syscalls in userspace, eg. \texttt{gettimeofday()},
\texttt{time()}, \texttt{getcpu()} etc
\end{itemize}
\item New idea: virtual dynamic shared object (vDSO) + Address Space Randomization (ASR)
\label{sec:orged6323b}
\begin{itemize}
\item Still allows system calls (whose object file is mapped in userspace) in user space
\item However there is Address Space Randomization (ASR) to change the address of
this vDSO, so attacker will not know where it is.
\end{itemize}
\end{enumerate}
\subsection{Interrupts}
\label{sec:org5d50020}
Motivating question: how to get input from devices?
\begin{enumerate}
\item Poll from each device regularly
Waste resources
\item Use interrupts - devices take the lead
\end{enumerate}
\subsubsection{When are interrupts checked?}
\label{sec:org45089c0}
\begin{itemize}
\item At EVERY pipeline round, Fetch - Decode - Execute - Check for Interrupts,
Fetch etc
\item If there is an interrupt:
\begin{enumerate}
\item Save context
\item Get interrupt number
\item Execute and clean up Instruction Service Routine (ISR)
\end{enumerate}
\end{itemize}
\subsubsection{Two types of interrupts}
\label{sec:org54b37e4}
\begin{enumerate}
\item Classification 1
\label{sec:orga731c2b}
\begin{enumerate}
\item Async
\label{sec:orgeeba96f}
Not related to current instruction - from external sources such as mouse click.
\item Synchronous (aka exceptions)
\label{sec:org7b1fbbc}
\begin{enumerate}
\item Faults (eg page faults)
It is retried. Intel since Pentium 4 has a \texttt{replay queue} to retry
instructions in case of fault.
\item Traps, Not retried.
\begin{itemize}
\item Eg. debuggers like gdb stop at a certain breakpoint.
This is done by injecting the \texttt{INT 3 (0xCC)} one-byte instruction at the
instruction location where you want to stop. CPU will hit this one-byte
instruction and the parent debugger process will gain control.
\end{itemize}
\item Abort
Hardware failure
\item Programmed, eg syscalls
\end{enumerate}
\end{enumerate}
\item Classification 2
\label{sec:org5c7e976}
\begin{enumerate}
\item I/O interrupts
\label{sec:org42cd3cb}
\begin{enumerate}
\item Critical
\label{sec:orgf7c3268}
\item Non-Critical
\label{sec:orgc2864c1}
\item Non-Critical Deferrable
\label{sec:org337a3e1}
\end{enumerate}
\item Timer
\label{sec:orgcdaa577}
\item Interprocessor
\label{sec:org1b3449e}
\end{enumerate}
\end{enumerate}
\subsubsection{Hardware to cause interrupts}
\label{sec:orga377372}
\begin{enumerate}
\item History: Intel 8259 Programmable Interrupt Controller (PIC)
\label{sec:orgd08e02b}
\begin{itemize}
\item A hardware chip that has multiple physical pins.
\item When the pins receive signals (by ethernet, whatever), it sends interrupts to
the CPU
\item Every interrupt has a mask (an interrupt is masked if CPU does not want to
service this interrupt) and a priority.
\end{itemize}
\item Extension: Advanced PIC (APIC)
\label{sec:orgc401bae}
\begin{itemize}
\item PIC can only service interrupts per-CPU. For multithreaded systems, APIC can
service inter-CPU interrupts.
\end{itemize}
\item Interrupt vectors (Intel)
\label{sec:org373647f}
\begin{itemize}
\item Each interrupt is mapped to a number (called vectors) in software.
\item Interrupt vector \texttt{32} and above are masked. Vector \texttt{2} is non-maskable
interrupt (NMI), which is the \textbf{hardware reset button}.
\item Which pin maps to which vectore is entirely programmable (by the hardware designer)
\end{itemize}
\item Assigning IRQs to Devices
\label{sec:org20b0803}
\begin{itemize}
\item PCI assigns at boot
\item Usually vector = IRQ + 32
\item The mapping is saved at the Interrupt Descriptor Table. Location is indicated
by the IDTR (IDT Register).
\end{itemize}
\end{enumerate}
\subsubsection{How are interrupts handled by x86?}
\label{sec:orgd265134}
\begin{enumerate}
\item Interrupt Descriptor Table Registor (IDTR)
\label{sec:orgb4f840b}
\begin{itemize}
\item Points to a table (IDT) consisting of either
\begin{enumerate}
\item Task-gate descriptor
\item Interrupt-gate descriptor
No more execute
\item Trap-gate descriptor
Can still execute
\end{enumerate}
\end{itemize}
\item Interrupts and stacks
\label{sec:org7f5709a}
\begin{itemize}
\item Interrupts are handled in the kernel, thus there is a switch of stack from
higher to lower ring.
\item If a ring change happens, the stack switches to that pointed to the current
task's Task State Segment (TSS), storing info for all registers.
\end{itemize}
\item Things that can happen during interrupt
\label{sec:org9b6c5ed}
\begin{enumerate}
\item Double Fault
\label{sec:org3f43c6f}
The exception-handling code induces another exception. Panics and dies.
\item Nested interrupt
\label{sec:org5a810c8}
Not to be confused with above - While handling eg. mouse interrupt, a mic
interrupt comes, and while handling mic interrupt, another interrupt comes etc\ldots{}
\begin{itemize}
\item This happens on a CPU. We can mask all other interrupts while handling one,
but it is not preferrable.
\item We can utilize other cores.
\item To handle an interrupt, we check if it is nested by checking whether the
privilege level is changed. If it is nested, ring 0 -> ring 0 then no change.
\end{itemize}
\end{enumerate}
\end{enumerate}
\subsubsection{How does OS handle interrupts?}
\label{sec:orgcfaa57a}
\begin{enumerate}
\item Overall strategy
\label{sec:orgb026f7a}
Splits the handling into \textbf{top} and \textbf{bottom} half.
\begin{enumerate}
\item Top half
\label{sec:org4a317d4}
bare minimum needed, save register, unmask etc. Very fast to return
\end{enumerate}
\item Bottom half
\label{sec:orgeef5d03}
\begin{enumerate}
\item Software IRQ (ksoftirq)
\label{sec:orga5fb464}
\begin{itemize}
\item Interrupt Queue as a software
\item Each processor runs a daemon, \texttt{ksoftirqd} which infinitely polls for softIRQs
to service.
\item softIRQ is re-entrant, means can be re-executed as needed
\end{itemize}
\item Tasklet
\label{sec:orgaa36fa9}
\begin{itemize}
\item Can be statically/dynamically allocated, whereas softIRQ is fixed
\item Is non-re-entrant.
\end{itemize}
\item Work queue
\label{sec:org40f3a7a}
\item Kernel Thread
\label{sec:org7094734}
\end{enumerate}
\end{enumerate}
\subsection{Signals}
\label{sec:org7105543}
\subsubsection{POSIX Signals}
\label{sec:org78686e9}
\begin{itemize}
\item 1990 and 2001, like \texttt{SIGINT, SIGKILL, SIGTERM} etc
\item Sent using syscall (to others) \texttt{kill}, or to self using \texttt{raise}
\end{itemize}
\subsubsection{Data structures for signal handling}
\label{sec:orge49b6e2}
\begin{itemize}
\item \texttt{task\_struct} is the process descriptor, saves signal handlers
\end{itemize}
\subsubsection{Handling flow}
\label{sec:org2870c71}
\begin{itemize}
\item Check slides 78 of Chapter 6
\item Kernel points the instruction pointer (RIP) to the signal handler, as well as
back up the stack
\end{itemize}
\section{Lecture 8: Memory}
\label{sec:org36ff607}
\subsection{The memory hierachy}
\label{sec:org9c4482e}
\begin{center}
\begin{tabular}{lllll}
 & L1 Cache & L2 Cache & Main Memory & Secondary Memory\\
\hline
clock cycle & a few & tens & hundreds & millions\\
size & word (4-8B) & block(8-32B) & 1-4blocks & very huge\\
\end{tabular}
\end{center}
\subsection{Principle of Caching: Locality}
\label{sec:orgc3af8e3}
\begin{itemize}
\item Temporal locality: repeatedly use the same data
\item Spatial locality: use nearby data
\end{itemize}
\subsection{Virtual vs Physical addressing}
\label{sec:org7d7912f}
\begin{itemize}
\item Every process has illusion that it owns all \(2^{64}\) bits of memory in its
virtual memory
\begin{itemize}
\item What if runs out of a certain limit? Out-Of-Memory (OOM) error
\end{itemize}
\item Supports ``Modern OS Features'':
\begin{itemize}
\item Protection: don't use other process's memory
\item Translation: ``use disk swap''
\item Sharing: Map multiple virtual pages to a sams physical page
\end{itemize}
\end{itemize}
\subsection{Address Translation}
\label{sec:orgead3566}
\begin{itemize}
\item Given page size = 4k, we have offset = 12 bits (2\textsuperscript{12}=4k)
\item Address = virtual page \# + offset
\begin{itemize}
\item Virtual page \# leads to an entry in page table, with
\begin{enumerate}
\item A valid bit
\item A next level virtual page \#
\end{enumerate}
\end{itemize}
\item Page fault: page is not in main memory
\end{itemize}
\subsubsection{Caching address traslation}
\label{sec:org565e307}
\begin{itemize}
\item Problem with caching:
\begin{itemize}
\item Cache that contains memory (physical page) is only relevant only when
address is \textbf{translated} which is expensive
\item Enter \textbf{TLB}, which caches virtual address translation
\begin{itemize}
\item Smaller, 128-256 entries max
\end{itemize}
\end{itemize}
\end{itemize}
\subsection{Paging in Linux x86}
\label{sec:orgd043e1b}
\begin{itemize}
\item 4KB a page
\item \texttt{PG} bit in register \texttt{CR0} toggles paging
\item Root page table given by (40bits) ``page directory base register'' in \texttt{CR3}
\end{itemize}
\subsubsection{x86-64 paging}
\label{sec:org1f3caee}
\begin{itemize}
\item only \textbf{48} bits are used. 48 = 9 + 9 + 9 + 9 + 12
\item \begin{center}
\begin{tabular}{rl}
Page Level & Name\\
\hline
4 & PLM4\\
3 & Directory Ptr\\
2 & Directory\\
1 & Table\\
\end{tabular}
\end{center}
\item Each process has its own CR3 hence root page table value.
\item Run through:
\begin{enumerate}
\item Get CR3 hence RPT address.
\item RPT + first 9 bits = addr of Level 2 page table.
\item Addr from 2 + next 9 bits = addr of Level 3 Page table.
\end{enumerate}
\end{itemize}
\begin{enumerate}
\item The CR3 Register
\label{sec:org3936b1d}
\begin{center}
\begin{tabular}{llr}
63:M & M-12 & 11:0\\
\hline
reserved & RPT addr & PCID\\
\end{tabular}
\end{center}
\item Reserve bits must be sign-extended
\label{sec:org2a13412}
IE if the most significant (non-reserved) bit is 1, then the reserved is 1 and vv.
\item Process Context ID = Address Space ID
\label{sec:org74f5b2d}
\begin{itemize}
\item ASID is different across processes (even though RPT might b the same)
\end{itemize}
\end{enumerate}
\subsubsection{Linux memory map}
\label{sec:org72fe617}
\begin{itemize}
\item If the 48th bit is 0 -> user space
\item Otherwise -> kernel space (half the Virtual Address Space)
\item Note that all are sign-extended
\end{itemize}
\subsubsection{Master Kernel Page Table}
\label{sec:org35d546a}
\begin{itemize}
\item Every process has own page table
\item Of which contains the same MKPT
\item MKPT maps the whole physical memory, unlike user process
\end{itemize}
\subsubsection{TLB}
\label{sec:org1b043ec}
\begin{itemize}
\item Q: When is it flushed?
A: Always by kernel. For eg when CR3 is changed
\end{itemize}
\subsubsection{Page Table Isolation (previously KAISER)}
\label{sec:orgf48b64c}
Only kernel process has the full kernel page table.
User process has only enough kernel page table to enter kernel mode (to prevent attacks).
\subsubsection{Virtual Memory Area (\texttt{vm\_area\_struct})}
\label{sec:orga664f87}
Stores the stack, heap, data, bss segments
\begin{enumerate}
\item Lazy expansion
\label{sec:org00885c2}
\begin{enumerate}
\item \texttt{brk()} requests for more pages when you run out of memory (eg. via
repeatedly calling \texttt{malloc} until oom)
\item New \textbf{virtual} pages are allocated. But not linked to physical pages.
\item UNTIL new virtual pages are used by eg. malloc.
\end{enumerate}
\end{enumerate}
\subsubsection{Physical Memory Zone}
\label{sec:orgd543790}
\begin{enumerate}
\item Direct Memory Access
\label{sec:org4d3d9cc}
Write directly into memory of drivers, disk etc
\begin{enumerate}
\item Modes
\label{sec:orgcb70114}
\end{enumerate}
\end{enumerate}
\subsection{Memory Allocator}
\label{sec:org3915196}
\subsubsection{Physical memory allocator (buddy allocator)}
\label{sec:orgee7ef84}
Eg. buddy system.
\begin{enumerate}
\item Buddy System
\label{sec:org62bcaf6}
Ask for k page frames. Upgrade to the nearest 2\textsuperscript{n} power. If there is such a
frame, give it. Otherwise, go up one level to 2\textsuperscript{(n+1)}, give it half, and put
half below.
\item API
\label{sec:orgde3c005}
\begin{itemize}
\item \texttt{kmalloc()} for physical
\begin{itemize}
\item Either \texttt{GFP\_ATOMIC} (not allowed to sleep, crucial)
\item or \texttt{GFP\_KERNEL}
\end{itemize}
\item \texttt{vmalloc} for virtual. kernel's \texttt{malloc()}
\end{itemize}
\end{enumerate}
\subsubsection{Virtual memory allocator (object allocator)}
\label{sec:org0bc6cf8}
\begin{enumerate}
\item Simple List of Blocks (SLOB)
\label{sec:org600c3ff}
\begin{itemize}
\item A kernel object is like the project we did in 2106, contains some bookeeping
data (pointing to the next free frame), some payload etc
\item SLOB uses first-fit, optimized (?) by best-fit
\end{itemize}
\item SLAB
\label{sec:org70f733c}
\begin{itemize}
\item A slab is a few contiguous pages.
\item Objects are aligned in to the size of cache lines.
\end{itemize}
\end{enumerate}
\subsection{Page Fault}
\label{sec:org00b58bc}
\subsubsection{Page Fault or SIGSEGV?}
\label{sec:orgf64b912}
\begin{itemize}
\item If the address trying to access is in the process address space:
\begin{center}
\begin{tabular}{ll}
Have rights to access? & Result\\
\hline
Yes & Access, or Page fault\\
No & SIGSEGV\\
\end{tabular}
\end{center}
\item Otherwise (wrong address space)
\begin{center}
\begin{tabular}{ll}
Mode? & Result\\
\hline
User & SIGSEGV\\
Kernel & Kernel bug, kill the process\\
\end{tabular}
\end{center}
\end{itemize}
\subsubsection{The Linux Kernel is not pageable}
\label{sec:org44deeb9}
This is to prevent page fault in page fault handling etc.
\subsubsection{Process}
\label{sec:org5edb051}
\begin{enumerate}
\item Entry point: \texttt{arch/x86/mm/fault.c:do\_page\_fault()}
\item Captures the linear address causing the fault in \texttt{CR2} contro register.
\item Block the process while setting if the fault is big.
\item Run the flowchart on slide page 99
\end{enumerate}
\subsection{Reclamation of Page Frame (deallocation of frame)}
\label{sec:orgdec4eab}
\subsubsection{Page Frame Reclamation Algorithm (PFRA)}
\label{sec:org0171cca}
\begin{itemize}
\item essentially removes/evict page frames to the swap to keep RAM free
\item must run before all frames are used up (otherwise no more process can run and
will crash)
\item 4 kinds of frame:
\begin{center}
\begin{tabular}{ll}
Type & Reclaim action\\
\hline
Unreclaimable & -\\
Swappable (anonymous) & evict to swap\\
Syncable (mapped to file) & sync with hard disk\\
Discardable & just discard\\
\end{tabular}
\end{center}
\end{itemize}
\subsubsection{Reverse mapping}
\label{sec:orgd52d353}
Motivation: given a frame, which page table entries (PTE) point to it?
\begin{enumerate}
\item Object-based Reverse Mapping
\label{sec:org7272aa9}
\begin{itemize}
\item Done by two fields:
\item \texttt{\_mapcount} counts the number of PTEs pointing to it. 0-indexed (ie. -1 when
no one points to it)
\item \texttt{mapping}:
\begin{center}
\begin{tabular}{ll}
state & meaning\\
\hline
null & belongs to the swap cache\\
non-null, lsb=1 & anonymous, points to \texttt{anon\_vma}\\
non-null, lsb=0 & mapped, points to the \texttt{address\_space} obj in page cache\\
\end{tabular}
\end{center}
\end{itemize}
\begin{enumerate}
\item \texttt{anon\_vma}
\label{sec:org8e5a7b1}
\begin{itemize}
\item A doubly linked list collecting all the pointers to the same frame
\end{itemize}
\item \texttt{address\_space}
\label{sec:org487ae23}
\begin{itemize}
\item Similar as \texttt{anon\_vma}
\end{itemize}
\end{enumerate}
\end{enumerate}
\subsubsection{Page Reclamation Process}
\label{sec:org898d280}
\begin{enumerate}
\item For mapped page frames:
\label{sec:org87859ee}
\begin{itemize}
\item Two clock lists (Least Recently Used), fixed size, one is the \texttt{active} list,
the other is \texttt{inactive} list
\item Original algorithm:
\begin{enumerate}
\item If freshly faulted (allocated) you start at the inactive list
\item If used for the second time, promoted to active list
\item In active list, membership cannot be refreshed. Once sufficient active
pages enter the active list, it is demoted to inactive
\item In inactive long enough then reclaimed, else step (2)
\end{enumerate}
\item Some observations
\begin{itemize}
\item eviction: only happens when a new fault is introduced. kicks the head of
\texttt{inactive} list away.
\item activation: second time it is read, promoted from \texttt{inactive} to \texttt{active}.
\item min \# of inactive page access  = eviction + activation.
\item refault distance: R - E where E is the reading of the sum above when it is
evicted, R is the reading above when it is refaulted back into memory.
\begin{itemize}
\item Intuition: If the list was made R-E longer, it wont be evicted.
\end{itemize}
\item Minimum access distance: length of inactive + (R-E)
\end{itemize}
\end{itemize}
\end{enumerate}
\subsubsection{Page Cache (for opened files)}
\label{sec:org5098570}
\begin{itemize}
\item Write-back caching: changes are not written immediately but only marked dirty
on cache. Periodically, it is written into the disk (when evicted by algorithm)
\end{itemize}
\begin{enumerate}
\item Important fields in \texttt{address\_space}
\label{sec:org6bcdd45}
\begin{enumerate}
\item \texttt{page\_tree} a radix tree of all pages
\label{sec:org17d4ec8}
\begin{itemize}
\item Radix trees represent a map from keys (strings) to values.
\item All pages have their memory addresses as keys, pointed to the frames. Every 6
bits as a unit for branching.
\end{itemize}
\end{enumerate}
\end{enumerate}
\subsection{Swap}
\label{sec:org8eae2a0}
\begin{itemize}
\item A disk partition (or a file), maximally \texttt{MAX\_SWAPFILES=32}
\item Swap area consists of swap slots, each slot is 4KB.
\item Pages can be \textbf{swapped in} to swap or \textbf{swapped out} into RAM
\item Race conditions are resolved using the swap cache
\end{itemize}
\subsubsection{Swap Cache}
\label{sec:org2d243eb}
\begin{enumerate}
\item Suppose A, B are processes using a same page P.
\item Page P needs to be swapped.
\item Swap cache caches P. It provides a reference in case when swapping in P,
either process needs to read P.
\item Once done, both A and B point to the swap slot.
\end{enumerate}
\section{Lecture 9: Synchronization}
\label{sec:org1b6265b}
\subsection{Concurrency in the Kernel}
\label{sec:orge45f899}
\begin{itemize}
\item Motivation: SMP OS Kernels are pre-emptible
\item True concurrency: due to multiple processors
\item Pseudo concurrency: due to interleaving of instructions
\begin{itemize}
\item Software based: voluntary/involuntary preemption (avoidable by disabling
preemption)
\item Hardware based: interrupt/trap/fault/exception handlers (can be disabled)
\end{itemize}
\end{itemize}
\subsection{Synchronization techniques in Linux}
\label{sec:org273aad3}
\subsubsection{Per-CPU Variables}
\label{sec:org59e40b7}
\begin{itemize}
\item Removes the need to synchronize between cores
\end{itemize}
\begin{enumerate}
\item How is this implemented?
\label{sec:orgf7fc9c5}
\begin{itemize}
\item Code, Data, Extra, Stack Segments (CS,DS,ES,SS) have segment base = 0,
creating a flat address space.
\item FS, GS are exceptions.
\item \texttt{SWAPGS} instruction swaps in correct value of the GS register (kept to 0 in
user mode) from the MSR
\item Which points to an array of pointers, maintaining per-cpu variables
\end{itemize}
\end{enumerate}
\subsubsection{Atomic Operations}
\label{sec:orgfb95a59}
\begin{itemize}
\item Atomic instructions, implemented by the CPU
\end{itemize}
\subsubsection{Memory Barriers}
\label{sec:org8f4b66d}
\begin{itemize}
\item Optimization barrier (user library): \texttt{barrier()} to prevent reordering of
instructions by the compiler
\item Memory barrier (instructions): to prevent read-write reordering
\end{itemize}
\subsubsection{Spin Locks (Active waiting)}
\label{sec:org9f5a8e9}
\begin{enumerate}
\item API
\label{sec:org7a9ee21}
\texttt{spin\_lock()} and \texttt{spin\_unlock()}
\item Ticket Locks
\label{sec:org5bd43f0}
Assumption: there is an atomic \texttt{fetch\_and\_inc(var)} function.
\begin{itemize}
\item entering cs: take a ticket number with fetch and inc. Active waiting when is
not my number
\item exiting cs: increment the now\textsubscript{serving} with \texttt{fetch\_and\_inc(var)}.
\end{itemize}

\item Problem with interrupts
\label{sec:orge51bfdb}
\begin{itemize}
\item What if an interrupt happens when a process holds the spin lock?
\item What if the interrupt handler needs to acess data held by the spin lock holder?
\end{itemize}
\item Solution: disable interrupts while holding spin lock
\label{sec:orgc4e89b0}
\end{enumerate}
\subsubsection{Semaphore}
\label{sec:orgfc70ad1}
\begin{enumerate}
\item Etymology
\label{sec:org3a50cae}
Flags used on ships to communicate
\item Difference compared to spin locks
\label{sec:org98d0113}
\begin{itemize}
\item Contention for the lock causes \textbf{blocking} not \textbf{spinning}.
\item Can allow concurrency of more than 1 process if necessary
\end{itemize}
\item API
\label{sec:org01b51c1}
\begin{itemize}
\item \texttt{down()} or \texttt{P()}: can be blocked, although can either down or try\textsubscript{down}
\item \texttt{up()} or \texttt{V()}
\end{itemize}
\end{enumerate}
\subsubsection{Reader/writer locks}
\label{sec:org7e642f2}
Allows multiple reader or a single writer. Increase efficiency of semaphore
\begin{enumerate}
\item Big reader locks (\texttt{br\_lock})
\label{sec:orgab5daa9}
Specialized form, very fast to acquire for reading but slow for writing. Think
possible starvation for writer?
\end{enumerate}
\subsubsection{Seqlock}
\label{sec:orga26343c}
A reader-writer lock with high priority on writers
\begin{enumerate}
\item How it works
\label{sec:orge825d17}
\begin{enumerate}
\item On top of a spin lock, there is a sequence field
\item The sequence value is incremented when writer uses
\item Readers might need to re-read
\end{enumerate}
\end{enumerate}
\subsubsection{Read-copy Update (RCU)}
\label{sec:org3d9ae39}
\begin{itemize}
\item Lock-free data structure
\item Another R-W lock. Writes are written to a new copy, and writer changes the
pointer to the new copy atomically.
\item Before reading, some time is given (swap to user space, idle loop) to ensure
the new copy is pointed to, hence read.
\end{itemize}
\subsubsection{Completion}
\label{sec:orgd3f1484}
\subsection{Lock-free data structures}
\label{sec:org49dda6e}
\subsubsection{Problem with locks}
\label{sec:orgf101066}
\begin{enumerate}
\item Deadlock
\item Priority inversion
\item Convoying - long thread get the lock, stalling short ones
\item Signal safety
\item Kill-tolerance - what if the holder is killed?
\item Pre-emption tolerance - what if the holder is preempted?
\item Performance - forced sequential operations
\end{enumerate}
\subsubsection{The heart of lock-free data structures - CAS}
\label{sec:org046119f}
\ldots{} is the \texttt{compare\_and\_swap(*ptr, old\_val,  new\_val)} atomic operation.
\subsubsection{An example: lock-free stack}
\label{sec:org988ef53}
\begin{enumerate}
\item Structure
\label{sec:org1f45070}
A (singly) linked list, where head is the top of the stack
\item API
\label{sec:org7052cd2}
\begin{enumerate}
\item push
\label{sec:org01e2afb}
\begin{enumerate}
\item create a linked list node with the new value
\item while \texttt{compare\_and\_swap(head, curr\_node->next, curr\_node)} is false, make the
new node point to the (new) linked list head
\end{enumerate}
\item pop
\label{sec:orgd67da1b}
\end{enumerate}
\item The ABA Problem
\label{sec:orgdefad77}
\begin{enumerate}
\item Say the stack is A -> B -> X
\item P1 reads A from the stack
\item P2 pops A
\item P2 pops B
\item P2 pushes A
\item P1 comes back and compares with the head, which is still A, but the value is
changed (they use the same pointer?)
\end{enumerate}
\begin{enumerate}
\item Solution
\label{sec:org3240b06}
\begin{itemize}
\item Keep an update count for each element. That way the new generation will not be
confused as the previous.
\item Uses the doubleword compare and swap (CAS)
\end{itemize}
\end{enumerate}
\end{enumerate}
\subsubsection{CAS is Insufficient}
\label{sec:org89ac0a8}
Imagine a linked list, P1 tries to delete the 2nd element, P2 tries to insert
between the 2nd and 3rd element. The new list does not the 1st element point to
P2 in one execution (verify this).
\section{Lecture 10: Filesystem}
\label{sec:orgfb22055}
A bookeeping attempt for the array of bits on hard drive.
\subsection{Virtual Filesystem (VFS)}
\label{sec:org21e301c}
\begin{itemize}
\item Known though its APIs: \texttt{open, close, read, write} syscalls
\item Abstracts through different fs: \texttt{ext2, ms-dos} etc
\end{itemize}
\subsubsection{Common File Model}
\label{sec:orgfd35be5}
\begin{itemize}
\item The data structure to represent files
\end{itemize}
\begin{enumerate}
\item Superblock object
\label{sec:org30d6e51}
\begin{itemize}
\item Stores info about a mounted system
\item All superblocks are stored in a doubly linked list
\item corr. to (filesystem control block)
\end{itemize}
\item Inode object
\label{sec:orge293454}
\begin{itemize}
\item metadata for a file, directory etc
\item \texttt{ls -i}
\item corr. to (file control block)
\end{itemize}
\item File object
\label{sec:orgced6044}
\begin{itemize}
\item Only exists in kernel memory, when it is opened.
\item Transactonal data between a process (reading/writing it) and the file
\item hierachy \texttt{task\_struct->files\_struct->struct file}
\end{itemize}
\item Dentry object
\label{sec:org4765319}
\begin{itemize}
\item Directory entry: links dirs to files
\item Is cached since takes long to construct
\item Exists because a directory may not exist in the disk, need a common API
\end{itemize}
\end{enumerate}
\subsubsection{Everything is a file}
\label{sec:org3d6d4c9}
filesystems include \texttt{usbfs} etc
\subsubsection{API}
\label{sec:orgbbbf9a0}
\begin{itemize}
\item 0th layer: system calls
\item 1st layer: \texttt{read, write} etc in libc
\item 2nd layer: \texttt{fread, fwrite} etc work with file objects and have buffering
\end{itemize}
\subsection{Disk filesystems}
\label{sec:org31c577b}
\subsubsection{Purpose}
\label{sec:org4e217d1}
\begin{itemize}
\item provides structure to data
\item provides a logical namespace
\item abstract user interface
\item security
\end{itemize}
\subsubsection{Journaling}
\label{sec:org8f6bdec}
\begin{itemize}
\item Keeps a ``journal'' for not-yet-committed changes to files, especially for
complex filesystems like deleting a file (remove dir entry, release inode,
move blocks into free pool, bookeeping\ldots{})
\item Provides points for recovery in case the chain above breaks due to a crash
\end{itemize}
\subsubsection{Examples}
\label{sec:orgf7afe80}
\begin{enumerate}
\item File Allocation Table (FAT)
\label{sec:org8a9610e}
\begin{enumerate}
\item Core Idea
\label{sec:orgc0f5e10}
the FAT (the table) points to sectors, which are then linked lists of sectors
forming the file.
\end{enumerate}
\item New Technology File System (NTFS)
\label{sec:org9cc3da6}
\end{enumerate}
\subsection{Sysfs - in-memory filesystem}
\label{sec:org67bbd35}
\end{document}
