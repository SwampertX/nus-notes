% Created 2021-03-18 Thu 11:33
% Intended LaTeX compiler: pdflatex
\documentclass[11pt]{article}
\usepackage[utf8]{inputenc}
\usepackage[T1]{fontenc}
\usepackage{graphicx}
\usepackage{grffile}
\usepackage{longtable}
\usepackage{wrapfig}
\usepackage{rotating}
\usepackage[normalem]{ulem}
\usepackage{amsmath}
\usepackage{textcomp}
\usepackage{amssymb}
\usepackage{capt-of}
\usepackage{hyperref}
\usepackage{amsthm}
\usepackage{minted}
\author{Tan Yee Jian}
\date{\today}
\title{Week 8: Memory}
\hypersetup{
 pdfauthor={Tan Yee Jian},
 pdftitle={Week 8: Memory},
 pdfkeywords={},
 pdfsubject={},
 pdfcreator={Emacs 27.1 (Org mode 9.4.4)}, 
 pdflang={English}}
\begin{document}

\maketitle
\tableofcontents

\section{The memory hierachy}
\label{sec:org869e480}
\begin{center}
\begin{tabular}{lllll}
 & L1 Cache & L2 Cache & Main Memory & Secondary Memory\\
\hline
clock cycle & a few & tens & hundreds & millions\\
size & word (4-8B) & block(8-32B) & 1-4blocks & very huge\\
\end{tabular}
\end{center}
\section{Principle of Caching: Locality}
\label{sec:org215dd61}
\begin{itemize}
\item Temporal locality: repeatedly use the same data
\item Spatial locality: use nearby data
\end{itemize}
\section{Virtual vs Physical addressing}
\label{sec:orgd91745f}
\begin{itemize}
\item Every process has illusion that it owns all \(2^{64}\) bits of memory in its
virtual memory
\begin{itemize}
\item What if runs out of a certain limit? Out-Of-Memory (OOM) error
\end{itemize}
\item Supports ``Modern OS Features'':
\begin{itemize}
\item Protection: don't use other process's memory
\item Translation: ``use disk swap''
\item Sharing: Map multiple virtual pages to a sams physical page
\end{itemize}
\end{itemize}
\section{Address Translation}
\label{sec:org0b479d6}
\begin{itemize}
\item Given page size = 4k, we have offset = 12 bits (2\textsuperscript{12}=4k)
\item Address = virtual page \# + offset
\begin{itemize}
\item Virtual page \# leads to an entry in page table, with
\begin{enumerate}
\item A valid bit
\item A next level virtual page \#
\end{enumerate}
\end{itemize}
\item Page fault: page is not in main memory
\end{itemize}
\subsection{Caching address traslation}
\label{sec:org8aebf55}
\begin{itemize}
\item Problem with caching:
\begin{itemize}
\item Cache that contains memory (physical page) is only relevant only when
address is \textbf{translated} which is expensive
\item Enter \textbf{TLB}, which caches virtual address translation
\begin{itemize}
\item Smaller, 128-256 entries max
\end{itemize}
\end{itemize}
\end{itemize}
\section{Paging in Linux x86}
\label{sec:orgac2ba53}
\begin{itemize}
\item 4KB a page
\item \texttt{PG} bit in register \texttt{CR0} toggles paging
\item Root page table given by (40bits) ``page directory base register'' in \texttt{CR3}
\end{itemize}
\subsection{x86-64 paging}
\label{sec:org543a478}
\begin{itemize}
\item only \textbf{48} bits are used. 48 = 9 + 9 + 9 + 9 + 12
\item \begin{center}
\begin{tabular}{rl}
Page Level & Name\\
\hline
4 & PLM4\\
3 & Directory Ptr\\
2 & Directory\\
1 & Table\\
\end{tabular}
\end{center}
\item Each process has its own CR3 hence root page table value.
\item Run through:
\begin{enumerate}
\item Get CR3 hence RPT address.
\item RPT + first 9 bits = addr of Level 2 page table.
\item Addr from 2 + next 9 bits = addr of Level 3 Page table.
\end{enumerate}
\end{itemize}
\subsubsection{The CR3 Register}
\label{sec:orga3b70f7}
\begin{center}
\begin{tabular}{llr}
63:M & M-12 & 11:0\\
\hline
reserved & RPT addr & PCID\\
\end{tabular}
\end{center}
\subsubsection{Reserve bits must be sign-extended}
\label{sec:org872a8f6}
IE if the most significant (non-reserved) bit is 1, then the reserved is 1 and vv.
\subsubsection{Process Context ID = Address Space ID}
\label{sec:orgf6190e4}
\begin{itemize}
\item ASID is different across processes (even though RPT might b the same)
\end{itemize}
\subsection{Linux memory map}
\label{sec:orgc6e86fe}
\begin{itemize}
\item If the 48th bit is 0 -> user space
\item Otherwise -> kernel space (half the Virtual Address Space)
\item Note that all are sign-extended
\end{itemize}
\subsection{Master Kernel Page Table}
\label{sec:orgc7d6a6c}
\begin{itemize}
\item Every process has own page table
\item Of which contains the same MKPT
\item MKPT maps the whole physical memory, unlike user process
\end{itemize}
\subsection{TLB}
\label{sec:org14412de}
\begin{itemize}
\item Q: When is it flushed?
A: Always by kernel. For eg when CR3 is changed
\end{itemize}
\subsection{Page Table Isolation (previously KAISER)}
\label{sec:orgdfa3377}
Only kernel process has the full kernel page table.
User process has only enough kernel page table to enter kernel mode (to prevent attacks).
\subsection{Virtual memory area (\texttt{vm\_area\_struct})}
\label{sec:org679efed}
\end{document}
