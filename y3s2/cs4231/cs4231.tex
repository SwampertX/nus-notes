% Created 2021-04-30 Fri 16:57
% Intended LaTeX compiler: pdflatex
\documentclass[11pt]{article}
\usepackage[utf8]{inputenc}
\usepackage[T1]{fontenc}
\usepackage{graphicx}
\usepackage{grffile}
\usepackage{longtable}
\usepackage{wrapfig}
\usepackage{rotating}
\usepackage[normalem]{ulem}
\usepackage{amsmath}
\usepackage{textcomp}
\usepackage{amssymb}
\usepackage{capt-of}
\usepackage{hyperref}
\usepackage{amsthm}
\usepackage{minted}
\author{Tan Yee Jian}
\date{\today}
\title{Cs4231}
\hypersetup{
  pdfauthor={Tan Yee Jian},
  pdftitle={Cs4231},
  pdfkeywords={},
  pdfsubject={},
  pdfcreator={Emacs 27.2 (Org mode 9.4.4)},
  pdflang={English}}
\begin{document}

\maketitle
\tableofcontents


\section{Arrows}
\label{sec:org1881088}
\begin{itemize}
  \item \(a\prec b\) is process order.
  \item \(a\to b\) is happen-before order: the logical clock value is less than, iff
        there is a directed path. Clearly \(a\prec b\implies a\to b\).
  \item \(a\leadsto b\) is the send-receive order.
\end{itemize}
\section{Lecture 5 Global Snapshot}
\label{sec:orgeba08ca}
Assumption: channels are FIFO. If not, it is discussed in Message Ordering
lecture
\subsection{Definitions}
\label{sec:orgcdef94f}
\subsubsection{Global snapshot}
\label{sec:orgb685326}
A set of events \texttt{E} such that for any event \texttt{e} in \texttt{E}, if \(f\prec e\), then
\(f\in E\).
\subsubsection{Consistent global snapshot}
\label{sec:orgea00abd}
A global snapshot where if \(e_1\) send, \(e_2\) receive, \(e_2\) in then \(e_1\) must be in.
\subsection{Chandy \& Lamport's protocal for taking snapshots}
\label{sec:orgf0defc2}
\begin{itemize}
  \item Key idea:
        \begin{enumerate}
          \item after a process takes a snapshot, it orders others to stop via a message
          \item If they already took a snapshot before the order arrives, then capture all
                message till the order (these are the on-the-fly messages)
          \item Otherwise once they receive the order, stop. (otherwise will receive stuff
                which sender does not include)
        \end{enumerate}
\end{itemize}
\section{Lecture 6 Message Ordering}
\label{sec:org8d1d9f7}
We still assume FIFO (except for fully async order) for inter-process channels.
\subsection{Definitions}
\label{sec:org7ce1ebf}
\subsubsection{Fully Asynchronous}
\label{sec:org1ca21bd}
No restriction.
\subsubsection{FIFO}
\label{sec:org0baa245}
Messages \(s_1,s_2\) are sent from i to j. Then \(\neg(s_2\prec s_1)\).
\subsubsection{Causally Ordered}
\label{sec:org20a9c79}
If \(s_1\) happen before \(s_2\), \(r_1, r_2\) on the same process, then \(r_1\prec
r_2\).
\subsubsection{Synchronous Ordered}
\label{sec:org87de76a}
Diagram can be redrawn into vertical, respecting happen-before, process and
send-receive order.
\subsection{Algorithm to ensure Causal Ordering}
\label{sec:orgf896f74}
\subsubsection{Core Idea}
\label{sec:org8ed3d63}
When a process sends a message, attach also every message you have seen or sent
so far.
\subsubsection{Why does it work?}
\label{sec:org0453d7c}
\begin{itemize}
  \item Recall causal ordering: if s1 happens before s2 (there is a directed path),
        then r1 must before r2
  \item If s1 and s2 are on the same process, then attaching all messages it has sent
        will include s1 for the target process to see before s2
  \item If not on the same process, then attaching all messages it has received will
        include s1 for the target process to see before s2.
\end{itemize}
\subsubsection{Can we improve it?}
\label{sec:org7ccdaa7}
\begin{itemize}
  \item Just include messages that are targeted at the receiving process.
  \item Or number each message locally on the process. Instead of attaching all
        messages, make the target wait until all processes' intended message to it has
        been delivered.
\end{itemize}
\subsection{Skeen's Algorithm for Broadcast Ordering}
\label{sec:orge4e7905}
\begin{itemize}
  \item An example is a chat group, sending messages to everyone
  \item Everyone needs the same message number for the same msg
\end{itemize}
\subsubsection{Core Idea}
\label{sec:org11c5dae}
\begin{enumerate}
  \item First round - send message to everyone to save in the buffer. After saving,
        each process replies the coordinator their current logical clock number.
  \item Wait for all them to come back, and assign the message the largest logical
        clock number.
  \item Let everyone know the decided message number.
\end{enumerate}
\subsubsection{Summary}
\label{sec:orgab337c5}
It orders messages, not logical clock numbers. Each logical clock value run as
usual.
\section{Lecture 7 Leader Election}
\label{sec:org808ec1e}
\subsection{Chang-Roberts Algorithm for Leader Election}
\label{sec:org0b3baae}
\subsubsection{Setting}
\label{sec:org7567e1b}
Given a ring of nodes that can only send messages clockwise, select a leader
\subsubsection{Algorithm}
\label{sec:org80a8bb4}
\begin{enumerate}
  \item Sending: Every node send its number clockwise
  \item Receiving: Every node relay the message (clockwise) if the value is bigger
        than self.
  \item If receive own id, then it is the leader
\end{enumerate}
\subsubsection{Complexity}
\label{sec:org86f55de}
\begin{enumerate}
  \item Message Complexity
        \label{sec:org2e23e03}
        Number of messages sent
  \item Best Case
        \label{sec:org4232628}
        Condition: sorted in clockwise ascending
        1+\ldots{}+1+n = 2n-1
  \item Worst case
        \label{sec:orgcfafe84}
        Condition: sorted in clockwise descending
        1 + 2 + \ldots{} + n = n(n+1)/2
  \item Average case?
        \label{sec:orgaeec55c}
\end{enumerate}
\section{Lecture 8 Consensus}
\label{sec:org9287394}
\subsection{Timing Models}
\label{sec:org9be41b7}
\subsubsection{Synchronous}
\label{sec:org7fd73e7}
\begin{itemize}
  \item Bounded amount of time to do processing (generate output msg, process input
        msg)
  \item and bounded amount of time to send messages
\end{itemize}
\subsection{Goals/Conditions for consensus}
\label{sec:org14e5ae4}
\subsubsection{Termination}
\label{sec:org288ea4c}
\subsubsection{Agreement}
\label{sec:org1574847}
\subsubsection{Validity}
\label{sec:org0e32ccf}
\subsection{Version 0: No failure (regardless of timing model)}
\label{sec:org1e318af}
\subsubsection{Algorithm}
\label{sec:orgb907115}
For any process:
\begin{enumerate}
  \item Keep broadcasting own value
  \item Once confident have all messages (due to known ub), run a deterministic
        algorithm on all values (eg. max/min)
\end{enumerate}
\subsubsection{When does it work?}
\label{sec:org9d9ccb8}
\begin{enumerate}
  \item {\bfseries\sffamily TODO} Synchronous
        \label{sec:orgccd72eb}
  \item {\bfseries\sffamily TODO} Async
        \label{sec:orgbe157ed}
\end{enumerate}
\subsection{Version 1: (Synchronous) Node crash failures}
\label{sec:org67b2133}
\subsubsection{Setup}
\label{sec:org0f51323}
\begin{itemize}
  \item Synchronous, all nodes have bounded processing time, all message has bounded
        delay
  \item Crash failure - crashes forever. On a ``round'' that it crashes, broadcast might
        not happen/not be complete.
\end{itemize}
\subsubsection{Intuition}
\label{sec:org3ff8c98}
\begin{itemize}
  \item We use rounds to gain information on crash: if no reply, then crashed.
  \item How to delineate rounds: let t1 be delay to generate output, t2 for message
        propagation, t3 to process input, then each round is simply \texttt{t1+t2+t3} long.
\end{itemize}
\begin{enumerate}
  \item Suppose we don't have accurate clock
        \label{sec:org9620598}
        \begin{itemize}
          \item Claim: every clock should be some constant multiple of other clocks.
          \item Consider a case where clock = 2x accurate clock.
                \begin{itemize}
                  \item Attempt 1: everyone set round duration to \texttt{2(t1+t2+t3)}. The process with
                        faster clock will advance rounds faster. Then every process that receives a
                        new message starts a new round. But what if a round starts before a process
                        has time to process its received message?
                  \item Attempt 2: We provide the offset, making the round duration
                        \texttt{2(t1+t2+t3+(t1+t2))}, the new offset tolerates the message delay(process
                        output, send msg) to the ``latest'' node.
                \end{itemize}
        \end{itemize}
\end{enumerate}
\subsubsection{Protocol}
\label{sec:orgeebc586}
\begin{enumerate}
  \item Intuition
        \label{sec:org8c4a5d5}
        \begin{itemize}
          \item suppose there is no failure, we can just broadcast our message to everyone and conclude.
          \item Since there can be a failure, we can \textbf{``forward''} all messages received in our broadcast.
          \item What is left is to decide how many rounds are needed.
        \end{itemize}
  \item Details
        \label{sec:org84abdb8}
        \begin{enumerate}
          \item Intialize the set \texttt{S:=\{my\_input\}}.
          \item Broadcast S to everyone for \texttt{f+1} rounds (where f is number of node crashes
                to tolerate)
          \item Union S with all the sets received in this round, and repeat 2.
        \end{enumerate}
  \item Correctness
        \label{sec:org3f8a457}
        \begin{enumerate}
          \item Termination is obvious (f+1 rounds, bounded waiting)
                \label{sec:orgc0c8b1c}
          \item Validity is obvious (if everyone same input, S = \{s\} is singleton)
                \label{sec:org2012c0f}
          \item Agreement
                \label{sec:orge6e4d16}
                \begin{enumerate}
                  \item Given f+1 rounds and f failure, there must be at least 1 round with no
                        failure. Call that the good round.
                  \item See that the good round must end with all nodes (that haven't crashed) with
                        the same set \texttt{S}.
                  \item After the good round, each nodes's \texttt{S} do not change anymore.
                  \item Then the deterministic function will choose the same value for all surviving
                        nodes.
                \end{enumerate}
        \end{enumerate}
\end{enumerate}
\subsubsection{Lower bound is (Omega(f))}
\label{sec:org779cd2c}
\begin{itemize}
  \item Statement: any deterministic algorithm that works (fulfills termination,
        agreement, validity) must take at least f+1 rounds.
  \item Proof is too hard (not in scope).
\end{itemize}
\subsection{Version 2: (Synchronous) Link Failure}
\label{sec:orgff2c823}
\subsubsection{Setup}
\label{sec:orgf83acf7}
\begin{itemize}
  \item nodes do not fail
  \item but the message channels (between any pair of processes) can fail arbitrarily
        long (drop unbounded \# of messages)
\end{itemize}
\subsubsection{Goal 1: Termination, Agreement, Validity}
\label{sec:org74df533}
\begin{itemize}
  \item Termination: all nodes eventually decide
  \item Agreement: all nodes settle on one same value
  \item Validity: if all started as same value, must settle on that value.
\end{itemize}
\begin{enumerate}
  \item There is no deterministic algorithm
        \label{sec:org4ae2aad}
        \begin{enumerate}
          \item Suppose there is one. Two processes with a eternally failing channel. Assume
                both have input 1, then has to conclude with 1.
                \begin{center}
                  \begin{tabular}{lrr}
                    process & input & conclusion\\
                    \hline
                    A & 1 & 1\\
                    B & 1 & 1\\
                  \end{tabular}
                \end{center}
          \item Suppose process B has input 0. To A, this is indistinguishable from B having
                input 1 since they cannot communicate. Thus B will still conclude 1.
                \begin{center}
                  \begin{tabular}{lrl}
                    process & input & conclusion\\
                    \hline
                    A & 1 & 1 (indistinguishable)\\
                    B & 0 & 1\\
                  \end{tabular}
                \end{center}
          \item By Agreement, B will conclude as 1 too, force by A's ignorant conclusion.
                \begin{center}
                  \begin{tabular}{lrl}
                    process & input & conclusion\\
                    \hline
                    A & 1 & 1\\
                    B & 0 & 1 (Agreement)\\
                  \end{tabular}
                \end{center}
          \item Now suppose A has input 0 as well. To B, this is indistinguishable from A
                having 1 (remember by 2, 3, A having 1 forces B to conclude 1) so it will
                still conclude with 1.
                \begin{center}
                  \begin{tabular}{lrl}
                    process & input & conclusion\\
                    \hline
                    A & 0 & 1\\
                    B & 0 & 1 (indistinguishable)\\
                  \end{tabular}
                \end{center}
          \item Then A will be forced to conclude with 1, violating Validity. Contradiction.
                \begin{center}
                  \begin{tabular}{lrl}
                    process & input & conclusion\\
                    \hline
                    A & 0 & 1 (validity)\\
                    B & 0 & 1\\
                  \end{tabular}
                \end{center}
        \end{enumerate}
\end{enumerate}
\subsubsection{Goal 2: T, A, Weakened Validity}
\label{sec:orgf15b4ba}
\begin{itemize}
  \item Since validity was violated just now, we try to weaken it. Is consensus
        possible now?
  \item If all start from 0, should settle on 0
  \item If all start from 1, then must settle on 1 only if no message is lost
\end{itemize}
\begin{enumerate}
  \item Still no deterministic algorithm.
        \label{sec:orgc435ea8}
        \begin{itemize}
          \item Lemma: if two processes start with 1, and one process's last message is lost,
                none detects it. Then the one who lost, B, didn't know it lost, hence is
                \textbf{indistinguishable} from nothing is lost and must settle on 1. By agreement, A
                should settle on 1 as well.
          \item Using the lemma above, we can still chain indistinguishability from 0, 0 to 1,
                1 and force a contradiction (settle on 0, 0 given 1, 1, when all messages are
                lost)
        \end{itemize}
        \begin{enumerate}
          \item Proof of no deterministic algorithm
                \label{sec:org4fc6e07}
                Essentially the same as in the previous Goal, but start with 0,0. Achieve
                contradiction not by validity (we weakened it thus would not be contradicted),
                but consider an undetected loss of last message, which has to conclude with 1,1.
        \end{enumerate}
\end{enumerate}
\subsubsection{Goal 3: T, Limited Agreement, Weakened Validity}
\label{sec:org3463446}
\begin{itemize}
  \item Limited Agreement: all nodes decide on the same value with probability =
        (1-\(\epsilon\))
\end{itemize}
\begin{enumerate}
  \item The concept of adversary
        \label{sec:orgcd0d58d}
        We imagine the message losses are not just random, but as bad as possible, as if
        designed by an adversary to destroy our plans for consensus. Then we do a worst
        case analysis against this adversary.
  \item Randomized Algorithm that works with error (1/r)
        \label{sec:org9350328}
        \begin{enumerate}
          \item Setup: We have processes P1, P2, etc.
                \begin{itemize}
                  \item All random choices not known by adversary beforehand.
                  \item Let the total number of rounds to be \texttt{r}.
                  \item Each process has an \textbf{input}, 0 or 1. Each process has a value called
                        \textbf{level}, initialized to 0.
                  \item P1 sets the a \textbf{bar} randomly between 1 and r, inclusive. Every process
                        tries to increment \textbf{level} so that they reach the \textbf{bar} over the \texttt{r}
                        rounds.
                \end{itemize}
          \item Broadcast messages and receive messages for r rounds. Each message contain
                \textbf{bar} (if you know it), \textbf{input}, and own \textbf{level}. Upon receive, update own
                \textbf{level} to \texttt{l+1} if everyone is at least level \texttt{l}.
          \item We can inductively show that every pair of levels must differ by at most one
                at every round.
          \item Once r rounds are over, decide on 1 IFF you know that everyone has input 1
                AND you know bar (trivial for P1) AND level >= bar.
        \end{enumerate}
        \begin{enumerate}
          \item Analysis
                \label{sec:org4d2258e}
                \begin{enumerate}
                  \item Termination (obvious by \texttt{r} rounds)
                        \label{sec:org9880bd7}
                  \item Agreement with probability (1-1/r)
                        \label{sec:orgab96424}
                        \begin{itemize}
                          \item Since not every process's \textbf{level} will reach \textbf{bar}, not everyone can decide
                        \end{itemize}
                        correctly (eg, should decide 1 but limited by insufficient level). We then
                        analyze what is the probability for these events.
                        \begin{itemize}
                          \item Error only happens if P1 decide on 1 while P2 decide on 0 (WLOG).
                          \item Denote P1's level as L1, P2's level as L2.
                        \end{itemize}
                        \begin{enumerate}
                          \item Scenario: One process decides 1, the other process decides 0
                                \label{sec:orgd225902}
                                \begin{enumerate}
                                  \item Case 1: P1 hears P2, but P2 never heard from P1 at all
                                        \label{sec:org5ad3aea}
                                        \begin{itemize}
                                          \item Then P2 must have level 0 (L2 = 0). It never heard from P1 and does not know
                                                bar, hence will conclude 0.
                                                \begin{center}
                                                  \begin{tabular}{lll}
                                                    process & level & decision\\
                                                    \hline
                                                    P1 &  & \\
                                                    P2 & 0 & 0\\
                                                  \end{tabular}
                                                \end{center}
                                          \item P1 hears P2 so must have level 1 (L1 = 1). To be an error case, it must
                                                decide 1.
                                                \begin{center}
                                                  \begin{tabular}{lrr}
                                                    process & level & decision\\
                                                    \hline
                                                    P1 & 1 & 1\\
                                                    P2 & 0 & 0\\
                                                  \end{tabular}
                                                \end{center}
                                          \item What do we know, now P1 decides 1? All conditions below must be fulfilled:
                                                \begin{enumerate}
                                                  \item Knows bar (no new info, P1 set the bar)
                                                  \item All inputs are 1
                                                  \item 1 >= bar. Since bar is at least 1, bar = 1.
                                                \end{enumerate}
                                        \end{itemize}
                                        \begin{enumerate}
                                          \item Conclusion: only happens when bar = 1, probability = 1/r
                                                \label{sec:orgd30a0df}
                                        \end{enumerate}
                                  \item Case 2: P2 hears P1, but P1 never heard from P2 at all
                                        \label{sec:org2a37117}
                                        \begin{itemize}
                                          \item Then P1 must have level 0 (L1 = 0). It never heard from P2 and does not know
                                                P2's input, hence will conclude 0.
                                                \begin{center}
                                                  \begin{tabular}{lrr}
                                                    process & level & decision\\
                                                    \hline
                                                    P1 & 0 & 0\\
                                                    P2 &  & \\
                                                  \end{tabular}
                                                \end{center}
                                          \item P2 hears P1 so must have level 1 (L2 = 1). To be an error case, it must
                                                decide 1.
                                                \begin{center}
                                                  \begin{tabular}{lrr}
                                                    process & level & decision\\
                                                    \hline
                                                    P1 & 0 & 0\\
                                                    P2 & 1 & 1\\
                                                  \end{tabular}
                                                \end{center}
                                          \item What do we know, now P2 decides 1? All conditions below must be fulfilled:
                                                \begin{enumerate}
                                                  \item Knows bar (no new info, P2 heard from P1)
                                                  \item All inputs are 1
                                                  \item 1 = L2 >= bar. Since bar is at least 1, bar = 1.
                                                \end{enumerate}
                                        \end{itemize}
                                        \begin{enumerate}
                                          \item Conclusion: only happens when bar = 1, probability = 1/r
                                                \label{sec:org6eb6f98}
                                        \end{enumerate}
                                  \item Case 3: P1 did not hear from P2 and vice versa
                                        \label{sec:orgb0c8981}
                                        This case would not cause error, since they will both conclude 0.
                                  \item Case 4: P1 heard from P2 and vice versa
                                        \label{sec:org50ae750}
                                        \begin{itemize}
                                          \item The only violatable condition before concluding 1, is that \texttt{level<bar}.
                                          \item Their \textbf{level} difference is at most one (by lemma).
                                          \item Probability is \texttt{1/r}, when the \textbf{bar} is set to the maximum of the two.
                                        \end{itemize}
                                \end{enumerate}
                          \item Conclusion: error probability is 1/r.
                                \label{sec:org631d363}
                        \end{enumerate}
                  \item Weakened Validity
                        \label{sec:orgecacf0a}
                        \begin{itemize}
                          \item suppose everyone starts with 0, no one can conclude 1 therefore everyone will
                                conclude 0.
                          \item suppose everyone starts with 1.
                                \begin{itemize}
                                  \item suppose no message lost, everyone will reach \textbf{bar} in \texttt{r} (\(r\ge bar\))
                                        rounds. In fact, everyone will have \textbf{level} = \texttt{r}. Everyone will conclude 1.
                                  \item suppose message lost, then can decide on anything.
                                \end{itemize}
                        \end{itemize}
                \end{enumerate}
        \end{enumerate}
  \item Error probability of 1/\#rounds is a lower bound.
        \label{sec:orge6193b0}
        \begin{itemize}
          \item proof can be found in Lynch.
        \end{itemize}
\end{enumerate}
\subsection{Version 3: (Asynchronous) Node crash failures}
\label{sec:orgdf19b89}
\subsubsection{Setup}
\label{sec:org78afd07}
\begin{itemize}
  \item Nodes can crash (indefinitely), but channels are reliable
  \item Asynchronous: message delay is unbounded
  \item Impact: can no longer define a round
\end{itemize}
\subsubsection{Fischer-Lynch-Paterson (FLP) Impossibility Theorem}
\label{sec:org3e9aec6}
Statement: the distributed consensus problem under the asynchronous timing model
is impossible to solve, even with a \textbf{single} node crash failure.
\begin{itemize}
  \item Fundamental reason: the protocol is unable to accurately detect node failure.
\end{itemize}
\subsection{Version 4: (Synchronous) Byzantine Failures}
\label{sec:org7026a71}
\subsubsection{Failure Model}
\label{sec:orgb5ec6a5}
Each node can ``lie'' about its input.
\subsubsection{Modified goals - everything only applies to non-faulty nodes}
\label{sec:org9b180d8}
\subsubsection{A simple (unsuccessful) attempt}
\label{sec:orgd0f876e}
\begin{enumerate}
  \item (Byzantine fault) A tells B 1, A tells C 0.
        B tells everyone 1.
        C tells everyone 0.
        \begin{center}
          \begin{tabular}{ll}
            node & received\\
            \hline
            A & 1, 1, 0\\
            B & 1, 1, 0\\
            C & 0, 1, 0\\
          \end{tabular}
        \end{center}
  \item Anyone could be lying. C decides B is lying -> it decides on 0. Similarly, B
        will decide on 1, violating Agreement.
  \item Furthermore, if you cross-check, you are not sure if the person you
        cross-check with is reliable.
\end{enumerate}
\subsubsection{Theorem: Byzantine Consensus Threshold}
\label{sec:orgdbbb288}
\begin{itemize}
  \item Theorem: if number of processes \texttt{n}, number of byzantine failures \texttt{f}, if
        \texttt{n<=3f}, then any protocol cannot reach consensus.
  \item Proof: omitted.
\end{itemize}
\subsubsection{A protocol for \# nodes n >= 4f+1}
\label{sec:orgaa3c626}
\begin{enumerate}
  \item Definitions
        \label{sec:orgd8bddc9}
        \begin{itemize}
          \item We have \emph{phases}.
          \item Each phase has a \emph{coordinator} which sends a proposal to all processes.
          \item If everyone decides, we have an agreement.
          \item A phase is a \emph{deciding phase} if the coordinator is nonfaulty.
          \item Each phase has a 3 rounds: broadcast, coordinator, decision.
        \end{itemize}
  \item Protocol
        \label{sec:orgb486711}
        \begin{enumerate}
          \item Setup: maintain everyone's value initialized to 0, my value = my input.
          \item Repeat the following steps (called a phase) for \texttt{f+1} times:
                \begin{enumerate}
                  \item Round 1: all-to-all broadcast. Send my input to everyone (including
                        myself). Record all msg received (if msg is non-binary or null, set it to
                        0). If there is a majority (occurrence > n/2) value, let \texttt{proposal} be that
                        value, otherwise \texttt{proposal} is 0.
                  \item Round 2: coordinator. if I am the coordinator (the ith phase has the ith node as
                        coord), send proposal to all, otherwise receive a proposal.
                  \item Round 3: if there is an overwhelming majority in V (already determined in
                        R1), ie > n/2 + f, set my value to that majority value, otherwise set it
                        to the proposal.
                \end{enumerate}
          \item After phases end, conclude with my value.
        \end{enumerate}
  \item Correctness - invariance for every round
        \label{sec:orge902a7a}
        \begin{enumerate}
          \item Lemma 1: all non-faulty processes retain \texttt{y} if all their values are \texttt{y}
                \label{sec:orgc4d10a8}
                Since \(n \ge 4f+1\), we must have \(n-f \ge 3f+1 \ge (4f + 1)/2 + f = n/2 + f\).
                This means the non-faulty nodes must be a overwhelming majority. Suppose they
                all have the same value at the start of phase \texttt{k}, they must retain these values
                at end of phase \texttt{k} forced by round 3.
          \item Lemma 2: if coord is nonfaulty, then all nonfaulty processes have the same value after a round k.
                \label{sec:orgec8b153}
                The coordinator can propose two values:
                \begin{enumerate}
                  \item Proposes some \texttt{x}. \texttt{x} occurs MORE THAN \texttt{n/2} times on coordinator process.
                        Worst case, \texttt{f} out of \texttt{n/2} are faulty. These MORE THAN \texttt{n/2-f} non-faulty
                        nodes must broadcast exactly these values to others, thus on any other nodes,
                        they cannot have another value \texttt{y} having the overwhelming majority (\texttt{y} can
                        only have LESS THAN \texttt{n/2+f} votes) in R3 and can thus only accept the
                        proposal.
                  \item Proposes 0. This means out of all the values coord sees, none exceed \texttt{n/2}.
                        It means none of these values will reach overwhelming majority MORE THAN
                        \texttt{n/2+f} since the amount can differ by atmost f.
                \end{enumerate}
          \item Termination - we let there be f+1 phases. No wait => termination.
                \label{sec:orgf61872d}
          \item Validity - by lemma 1, the invariance make them decide its input.
                \label{sec:org9adb32d}
          \item Agreement
                \label{sec:org3ac6d38}
                \begin{itemize}
                  \item \texttt{f+1} phases means at least one good phase.
                  \item The one good phase triggers lemma 2.
                  \item Lemma 2 triggers lemma 1, agreeing on the value as per lemma 1.
                \end{itemize}
        \end{enumerate}
\end{enumerate}
\end{document}
