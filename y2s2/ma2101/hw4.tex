\documentclass[12pt, a4paper]{article}

\usepackage[a4paper, total={170mm,257mm}, left=20mm, top=20mm]{geometry}
\usepackage[utf8]{inputenc}
\usepackage[T1]{fontenc}
\usepackage{textcomp}
\usepackage{amsmath, amssymb, amsthm}
\usepackage[outputdir=tmp]{minted}
\usepackage{lmodern}
\usepackage{fancyhdr}
\pagestyle{fancy}
\lhead{Tan Yee Jian (A0190190L)}
\chead{\title}
\rhead{Tutorial Group 03}


% figure support
\usepackage{import}
\usepackage{xifthen}
\pdfminorversion=7
\usepackage{pdfpages}
\usepackage{transparent}
\newcommand{\incfig}[1]{%
  \def\svgwidth{\columnwidth}
  \import{./figures/}{#1.pdf_tex}
}

\pdfsuppresswarningpagegroup=1

\newtheorem{thm}{Theorem}[section]
\newtheorem{crl}{Corollary}[thm]
\newtheorem{lemma}{Lemma}[thm]
\newtheorem{note}{Note}[thm]
\newtheorem{defn}{Definition}[section]
\newtheorem{ex}{Example}[section]
\newtheorem{prop}{Proposition}[section]
\newtheorem{obs}{Observation}
\newtheorem{claim}{Claim}

\newcommand{\pmat}[1]{ \begin{pmatrix}#1\end{pmatrix} }

\DeclareMathOperator{\spn}{span}
\DeclareMathOperator{\diam}{diam}

\title{MA2101 Homework 4}
\author{Tan Yee Jian}
\date{29 March 2020}

\begin{document}
\maketitle
\begin{enumerate}
  \item
    \begin{enumerate} %q1
      \item
        \begin{proof}[Solution]
          \begin{align*}
            [T]_S &= \bigg([T(1)]_S \bigg| [T(x)]_S \bigg| [T(x^2)]_S \bigg)\\
            &=\begin{pmatrix}
              1&0&1\\
              0&1&0\\
              1&2&1
            \end{pmatrix}
          \end{align*}
          .\end{proof}

      \item
        \begin{proof}[Solution]
          Let $A=[T]_S$. Eigenvalues of $T=$ eigenvalues of $[T]_S$, which
          satisfy

          \[ \det(xI-A)=0 \]
          \[ \det\begin{pmatrix}
              x-1&0&-1\\
              0&x-1&0\\
              -1&-2&x-1
            \end{pmatrix}=0\\ \]
          \[ (x-1)(x-1)^2-(x-1)=0 \]
          \[ x(x-1)(x^2-2x+1-1)=0\]
          \[ x(x-1)(x-2)=0 \]

          Therefore the eigenvalues are \(0,1,2\).
        \end{proof}

      \item
        \begin{proof}[Solution]
          Since $\dim P_2(\mathbb{R}) =3=|\{0,1,2\}|$, $T$ is
          diagonalizable.

          The basis consisting entirely of eigenvectors of T would satisfy


          \begin{align*}
            E_0 &= \ker(A-\mathbf{0})=\ker(A)\\
            &=\ker\pmat{1&0&-1\\0&1&0\\1&2&1}\\
            &=\ker\pmat{1&0&-1\\0&1&0\\0&0&0}\\
            &=\spn\big\{\pmat{1\\0\\-1}\big\}
            .\end{align*}

          For eigenvalue 1,

          \begin{align*}
            E_1 &= \ker(A-I)\\
            &=\ker\pmat{0&0&1\\0&0&0\\1&2&0}\\
            &=\spn\big\{\pmat{2\\-1\\0}\big\}
            .\end{align*}

          And eigenvalue 2,

          \begin{align*}
            E_2 &= \ker(A-2I)\\
            &=\ker\pmat{-1&0&1\\0&-1&0\\1&2&-1}\\
            &=\ker\pmat{-1&0&1\\0&-1&0\\0&0&0}\\
            &=\spn\big\{\pmat{1\\0\\1}\big\}
            .\end{align*}

          Therefore, $\mathcal{B}=\{1-x^2,2-x,1+x^2\}$, and $[T]_\mathcal{B}=$
          \[ \pmat{0&0&0\\0&1&0\\0&0&2} \]
          .\end{proof}
    \end{enumerate}

  \item
    \begin{enumerate} %q2
      \item
        \begin{proof}[Solution]
          Since $m_A(x)$ and $c_A(x)$ have the same roots, either
          $m_A(x)=(x+1)^2(x-2)$ or $m_A(x)=(x+1)(x-2)$.

          Since
          \[ (A+I)(A-2I)=\pmat{1&0&-4\\-1&-2&-1\\-1&-1&-2}
            \pmat{0&1&-3\\0&-3&0\\0&0&-3}=\mathbf{0}\]
          \[ \therefore m_A(x)=(x+1)(x-2) \].
        \end{proof}

      \item
        \begin{proof}[Solution]
          I claim that $\dim\{I,A,A^2,A^3,\dots\}=2=\text{ degree of }m_A(x)$. I
          will first show that $\{I,A\}$ is linearly independent, and then that
          $\{A^n:n=2,3,\dots\}$ can each be expressed as a linear combinations of
          $\{I,A\}$.

          Consider $\alpha_0I+\alpha1A=0$. Then if $\alpha_1\neq0$, then
          $\frac{\alpha_0}{\alpha_1}I+A=0$. Consider
          $f(x)=\frac{\alpha_0}{\alpha_1}+x$, which satisfies $f(A)=0$, but that
          contradicts with the fact that the minimal polynomial has degree 2,
          thus $\alpha_0=0$. Similarly, if $\alpha_0\neq0$, then$=a_0$, which is
          a contradiction too. Therefore, $\alpha_0=\alpha_1=0$ which means
          $\{I,A\}$ is linearly independent.

          Now, I show by induction that $\forall A^n, n=2,3,\dots$ can be
          expressed as a linear combination of ${I,A}$. From the minimal
          polynomial \( m_A(x)=(x+1)(x-2) \), we have base case of $n=2$ below:

          \begin{align*}
            m_A(A)&=(A-I)(A-2I)=A^2-A-2I=0\\
            \therefore A^2&=A+2I
          \end{align*}

          By induction, assuming that $A^k=a_0I+a_1A$. Then
          \begin{align*}
            A^{k+1}&=A(A^k)\\&=a_0A+a_1A^2\\&=a_0A+a_1A+2a_1I\\&=2a_1I+(a_0+a_1)A
            .\end{align*}

          Which is a linear combination of $\{I,A\}$. Therefore
          $\dim\{I,A,A^2,A^3,\dots\}=2=\text{ degree of }m_A(x)$.
          .\end{proof}
    \end{enumerate}

  \item
    \begin{enumerate} %q3
      \item
        \begin{proof}[Solution]
          \[
            \begin{pmatrix}
              J_2(1)&\mathbf{0}&\mathbf{0}&\mathbf{0}&\mathbf{0}\\
              \mathbf{0}&J_2(2)&\mathbf{0}&\mathbf{0}&\mathbf{0}\\
              \mathbf{0}&\mathbf{0}&J_1(1)&\mathbf{0}&\mathbf{0}\\
              \mathbf{0}&\mathbf{0}&\mathbf{0}&J_2(2)&\mathbf{0}\\
              \mathbf{0}&\mathbf{0}&\mathbf{0}&\mathbf{0}&J_1(2)
            \end{pmatrix},
            \begin{pmatrix}
              J_2(1)&\mathbf{0}&\mathbf{0}&\mathbf{0}&\mathbf{0}&\mathbf{0}\\
              \mathbf{0}&J_2(2)&\mathbf{0}&\mathbf{0}&\mathbf{0}&\mathbf{0}\\
              \mathbf{0}&\mathbf{0}&J_1(1)&\mathbf{0}&\mathbf{0}&\mathbf{0}\\
              \mathbf{0}&\mathbf{0}&\mathbf{0}&J_1(2)&\mathbf{0}&\mathbf{0}\\
              \mathbf{0}&\mathbf{0}&\mathbf{0}&\mathbf{0}&J_1(2)&\mathbf{0}\\
              \mathbf{0}&\mathbf{0}&\mathbf{0}&\mathbf{0}&\mathbf{0}&J_1(2)
            \end{pmatrix}
          \]
          where the $\mathbf{0}$ represent blocks of zeroes.
        \end{proof}

      \item 2. since there can only be 2 jordan blocks of value 1, of sizes 1
        and 2 respectively, in a jordan canonical form of the matrix, say these
        two blocks, $j_1(1),j_2(1)$ occupy the jth and kth position, i.e., the
        $(1,1)$ entry of these blocks are at the $(j,j)$ and $(k,k)$ position
        respectively (suitably apart), then $v_j,v_k$ are two (linearly
        independent) eigenvectors for the eigenvalue 1.
    \end{enumerate}
  \item
    \begin{enumerate} %q4
      \item
        \[ [T]_\mathcal{B} =
          \begin{pmatrix}
            J_3(-5)&\mathbf{0}&\mathbf{0}&\mathbf{0}&\mathbf{0}\\
            \mathbf{0}&J_2(-5)&\mathbf{0}&\mathbf{0}&\mathbf{0}\\
            \mathbf{0}&\mathbf{0}&J_3(0)&\mathbf{0}&\mathbf{0}\\
            \mathbf{0}&\mathbf{0}&\mathbf{0}&J_2(4)&\mathbf{0}\\
            \mathbf{0}&\mathbf{0}&\mathbf{0}&\mathbf{0}&J_1(4)
          \end{pmatrix},
        \]
      \item \[ \begin{aligned}
            T(v_1)&=-5v_1 &T(v_2)&=v_1-5v_2\\
            T(v_3)&=v_2-5v_3&T(v_4)&=-5v_4\\
            T(v_5)&=v_4-5v_5&T(v_6)&=0\\
            T(v_7)&=v_6&T(v_8)&=v_7\\
            T(v_9)&=4v_9&T(v_{10})&=v_9+4v_{10}\\
            T(v_{11})&=4v_{11}
          \end{aligned} \]

      \item Eigenvalues: $\{-5,0,4\}$.
        $E_{-5}=\spn\{v_1,v_4\},E_0=\spn\{v_6\},E_4=\spn\{v_9\}\quad\square$.
    \end{enumerate}
\end{enumerate}
\end{document}

