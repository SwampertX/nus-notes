\documentclass[12pt, a4paper]{article}

\usepackage[a4paper, total={170mm,257mm}, left=20mm, top=20mm]{geometry}
\usepackage[utf8]{inputenc}
\usepackage[T1]{fontenc}
\usepackage{textcomp}
\usepackage{amsmath, amssymb, amsthm}
\usepackage[outputdir=tmp]{minted}
\usepackage{lmodern}
\usepackage{fancyhdr}
\pagestyle{fancy}
\lhead{Tan Yee Jian (A0190190L)}
\chead{MA2108S Homework 6}


% figure support
\usepackage{import}
\usepackage{xifthen}
\pdfminorversion=7
\usepackage{pdfpages}
\usepackage{transparent}
\newcommand{\incfig}[1]{%
  \def\svgwidth{\columnwidth}
  \import{./figures/}{#1.pdf_tex}
}

\pdfsuppresswarningpagegroup=1

\newtheorem{thm}{Theorem}[section]
\newtheorem{crl}{Corollary}[thm]
\newtheorem{lemma}{Lemma}[thm]
\newtheorem{note}{Note}[thm]
\newtheorem{defn}{Definition}[section]
\newtheorem{ex}{Example}[section]
\newtheorem{prop}{Proposition}[section]
\newtheorem{obs}{Observation}
\newtheorem{claim}{Claim}

\newcommand{\pmat}[1]{ \begin{pmatrix}#1\end{pmatrix} }
\newcommand{\seqn}[1]{(#1)^\infty_{n=1}}
\newcommand{\seqk}[1]{(#1)^\infty_{k=1}}
% \newcommand{\txt}[1]{\quad\text{#1}\quad}

\DeclareMathOperator{\spn}{span}
\DeclareMathOperator{\diam}{diam}


\title{MA2108S Homework 6}
\author{Tan Yee Jian}
\date{\today}

\begin{document}
\maketitle
\begin{enumerate}
  \item \begin{proof} % q1
      For the sake of contradiction, assume that
      \[\forall d>0, \rho(x,y)<d\text{ for all }x\in E\text{ and all }y\in
        F\].

      Then let
      \begin{gather*}
        \rho(x_1,y_1) < 1\\
        \rho(x_2,y_2) < \frac{1}{2}\\
        \dots\\
        \rho(x_n,y_n) < \frac{1}{n}\\
      \end{gather*}

      Since $E$ is compact, then the sequence $\seqn{x_n}$  has a convergent
      subsequence, $\seqk{x_{n_k}}$, converging to a certain $x\in E$. We
      claim that $\seqk{y_{n_k}}$ is also convergent to $x$. If this is true,
      then since $F$ is closed, we have also $x\in F$. This then implies that
      \[ x\in (E\cap F) \iff (E\cap F)\neq\emptyset \], a contradiction.

      To prove that $\seqk{y_{n_k}}$ is convergent: let $\epsilon$ be given.
      Then since $\seqk{x_{n_k}}$ is convergent,

      \[ \exists N_1\in\mathbb{N} \text{ such that }\rho(x_{n_k}, x)
        <\frac{\epsilon}{2}\text{ for any } n_k>N_1\].

      Also, we can choose $N_2$ such that $ \frac{1}{N_2} <\frac{\epsilon}{2}
      \iff N_2 > \frac{2}{\epsilon}$, which implies
      \[ \rho(x_{n_k},y_{n_k}) < \frac{1}{n_k} \leq \frac{1}{N_2} <
        \frac{\epsilon}{2} \].

      Taking $N=\max\{N_1,N_2\}$, we have for any $n\geq N$,
      \[ \rho(y_{n_k},x)\leq\rho(y_{n_k},x_{n_k}) + \rho(x_{n_k},x) <
        \frac{\epsilon}{2}+\frac{\epsilon}{2}=\epsilon \]
      which shows that $\lim_{k\rightarrow\infty}{y_{n_k}} = x$.
    \end{proof}
  \item \begin{enumerate}
      \item \begin{proof} %q_2a
          For any $\epsilon>0$, we can choose $\delta<\frac{\epsilon}{C}$, and
          thus for any $x,y\in M_1$,
          \begin{align*}
            \rho_1(x,y)&<\delta<\frac{\epsilon}{C}\implies\\
            \rho_2(f(x),f(y))&\leq C\rho_1(x,y)\\
            &<C(\frac{\epsilon}{C})\\
            &=\epsilon
          \end{align*}
          F is uniformly continuous.
        \end{proof}
      \item \begin{proof}
          We use the fact that $g$ is uniformly continuous: take any
          $\epsilon>0$, (for example $\epsilon=1$), then there is a $\delta>0$
          such that
          \[ |x-y|<\delta \implies |g(x)-g(y)|<\epsilon \].

          I choose $r=\delta$, then we try to find $C$. If $|x-y|\geq r$, then
          we can select $(n-1)$ "jumping points" from $x$ to $y$ in intervals of
          $\delta/2$, such that between two consecutive points, it can satisfy
          the uniform continuity. That is, without loss of generality, assume
          $x<y$, then we select $x, (x+\frac{\delta}{2}), \dots,
          (x+(n-1)\frac{\delta}{2}),y$, where $(n-1)\frac{\delta}{2}<|x-y|$ and
          $\frac{n\delta}{2}\geq|x-y|$. By triangle inequality, we have

          \begin{align*}
            |g(x)-g(y)|&\leq|g(x)-g(x+\frac{\delta}{2})|
            +|g(x+\frac{\delta}{2})-g(x+\frac{2\delta}{2})|+\dots\\
            &\quad +|g(x+\frac{(n-1)\delta}{2})-g(y)|\\
          \end{align*}

          Since each of the terms are at most $\frac{\delta}{2}$ apart, and we
          have
          \[(n-1)\frac{\delta}{2}<|x-y| \implies
            n<\frac{2|x-y|+\delta}{\delta}\]
          Thus
          \begin{align*}
            |g(x)-g(y)|&<n\epsilon\\
            &<(\frac{2|x-y|+\delta}{\delta})\epsilon
          .\end{align*}

        Since $\delta=r\leq|x-y|$,

        \begin{align*}
          |g(x)-g(y)|
          &<(\frac{2|x-y|+\delta}{\delta})\epsilon\\
          &\leq(\frac{2|x-y|+|x-y|}{\delta})\epsilon\\
          &=(\frac{3\epsilon}{\delta})|x-y|
        \end{align*}

        and there we have it, $C=\frac{3\epsilon}{\delta}$.
      \end{proof}
    \end{enumerate}
  \item \begin{proof}
      Yes, a uniformly convergent sequence of uniformly continuous function
      would converge to a uniformly continuous function, and I will show that
      using the "famous" $\epsilon/3$ argument.

      Let $\epsilon>0$ be given. Then we want to show that
      \begin{align*}
        \rho_2(f(x),f(y))&\leq\rho_2(f(x), f_n(x))+\rho_2(f_n(x),f_n(y))+
        \rho(f_n(y),f(y))\\&<\epsilon
      \end{align*}

      Using the given $\epsilon$, the uniform convergence of $\seqn{f_n}$ gives
      us $N\in\mathbb{N}$ where
      \[ \forall x\in M_1,\rho_2(f(x),f_n(x)) < \epsilon/3.\]

      And similarly, using the given $\epsilon$ again, the uniform continuity of
      $f:M_1\rightarrow M_2$ gives us $\delta>0$ such that
      \[ \rho_1(x,y)<\delta \implies \rho_2(f_n(x),f_n(y)) < \epsilon/3.\]

      Therefore,
      \begin{align*}
        \rho_2(f(x),f(y))&\leq\rho_2(f(x), f_n(x))+\rho_2(f_n(x),f_n(y))+
        \rho(f_n(y),f(y))\\
        &< \epsilon/3 + \epsilon/3 + \epsilon/3\\
        &=\epsilon
      \end{align*}
      $\therefore f$ is uniformly continuous on $M_1$.
  \end{proof}
\end{enumerate}
\end{document}
