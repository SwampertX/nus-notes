% Created 2020-02-19 Wed 11:24
% Intended LaTeX compiler: pdflatex
\documentclass[presentation]{beamer}
\usepackage[utf8]{inputenc}
\usepackage[T1]{fontenc}
\usepackage{graphicx}
\usepackage{grffile}
\usepackage{longtable}
\usepackage{wrapfig}
\usepackage{rotating}
\usepackage[normalem]{ulem}
\usepackage{amsmath}
\usepackage{textcomp}
\usepackage{amssymb}
\usepackage{capt-of}
\usepackage{hyperref}
\usetheme{default}
\author{Tan Yee Jian}
\date{\today}
\title{CS2106 Operating Systems}
\hypersetup{
 pdfauthor={Tan Yee Jian},
 pdftitle={CS2106 Operating Systems},
 pdfkeywords={},
 pdfsubject={},
 pdfcreator={Emacs 26.3 (Org mode 9.4)}, 
 pdflang={English}}
\begin{document}

\maketitle
\begin{frame}{Outline}
\tableofcontents
\end{frame}

\begin{frame}[label={sec:orgb061a4c},fragile]{Lecture 6 IPC, Threads}
 \begin{block}{Archipelago Q\&A}
\begin{enumerate}
\item Why not set the Time Quantum to the Interval of Timer Interrupt, since that
is the only important time to run the scheduler?
A: ITI is hardware specific, and QT is set by kernel.
\item If there is only one job in RR, then will there be context switch every TQ
before the job ends?
A: No, but some stack will still be saved upon the running of the Interrupt Routine.
\end{enumerate}
\end{block}
\begin{block}{Inter-process Communication}
Mechanisms:
\begin{block}{Shared memory}
\end{block}
\begin{block}{Message passing}
Can be either \alert{Blocking} or \alert{Non-blocking}.
\begin{block}{Receiver}
Blocking receive is the most common
\end{block}
\begin{block}{Sender}
\begin{block}{Asynchronous}
Even asynchronous can be blocking, since the mailbox(buffer) can be full.
If sent before receiver calls \texttt{receive}, is blocked until the receiver receives
the message using the \texttt{receive} function call.
\end{block}
\begin{block}{Synchronous}
Also known as \alert{Rendezvous}, since both process have to meet at a known point in time.
There is no immediate buffering, the sender have to be blocked until the
receiver is ready.
\end{block}
\end{block}
\begin{block}{Pros and Cons}
\begin{block}{Pros}
\begin{itemize}
\item Can send across different machines
\item Cross-platform, as long as message is standard
\item Easier synchronization
\end{itemize}
\end{block}
\begin{block}{Cons}
\begin{itemize}
\item Inefficient
\item Messages are limited in size and format
\end{itemize}
\end{block}
\end{block}
\end{block}
\begin{block}{Pipes (Unix)}
One of the earliest UPC mechanism
\begin{itemize}
\item FIFO
\item Blocking
\begin{itemize}
\item Writers wait when buffer is full
\item Readers wait when buffer is empty
\end{itemize}
\end{itemize}
\end{block}
\begin{block}{Signal (Unix)}
\alert{SIGH$\backslash$*}, programs which receive that either handle it with it's own handlers, or
 use the default handlers given by the system.
\end{block}
\end{block}
\begin{block}{Threads}
\alert{Motivation}: Processes are
\begin{itemize}
\item heavy/expensive
\item hard to context switch
\item IPC is hard, since each process have independent memory space
\end{itemize}
and threads can help us with
\begin{itemize}
\item achieving multi-core programming
\item even on 1 core, we can move I/O intensive tasks to a thread to prevent
blocking, and enable doing some other CPU tasks at the main process
\end{itemize}

\alert{Brain teaser: would we want to run two threads with exact same code?}
Answer: Yes, probably on different data. Egg. searching for a number in an
array - data-centric parallelization

What can Threads share?
\begin{itemize}
\item GPR - No. They will interfere with one another
\item Special registers - No
\item Text segment - Yes
\item Data segment - Yes
\item Heap - Yes
\item Stack - No. Different fn calls
\item PID - Yes
\item Files - Yes
\item Instruction cache - not relevant
\item Data cache - not relevant
\end{itemize}

In short, anything other than \alert{Stack and Registers} are shared.
\end{block}
\end{frame}
\end{document}
