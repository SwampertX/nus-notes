\documentclass[12pt, a4paper]{article}

\usepackage[utf8]{inputenc}
\usepackage[T1]{fontenc}
\usepackage{textcomp}
\usepackage{amsmath, amssymb}
\usepackage[outputdir=tmp]{minted}

% figure support
\usepackage{import}
\usepackage{xifthen}
\pdfminorversion=7
\usepackage{pdfpages}
\usepackage{transparent}
\newcommand{\incfig}[1]{%
  \def\svgwidth{\columnwidth}
  \import{./figures/}{#1.pdf_tex}
}

\pdfsuppresswarningpagegroup=1

\title{UPI2211 Week 5 Thursday}
\author{Tan Yee Jian}
\date{13 February 2020}

\begin{document}
  \maketitle
  \section{Colonialism, slavery, and moral\\inconsistency}%
  \begin{itemize}
    \item 1517, the new world order in full swing. Spanish fleet full of silver
    \item It is justified for Utopia to occupy others' land if they are not
      using it well. Termed as agrarian policy.
    \item Making slaves do the killing might further desensitize them from
      violent behaviors
    \item Ambiguities: why is Utopia partially good and partially bad?
    \item Is utopia a blueprint for an ideal society, or merely something
      to reflect on?
  \end{itemize}
  \section{Discussion}
  \begin{itemize}
    \item Colonialism: More did not care about the periphery societies.
      You can't create a society without letting someone down.
    \item It might be impossible to create the perfect society - eg. in Utopia,
      some people (eg elderly) gets the best treatment.
    \item \textbf{Prof Bart} : A utopian society need not be egalitarian.
      For example if we think about a communist society, we try to make everyone
      to be treated equally.
    \item Communist vs Socialist vs Anarchist societies - what is the difference?
  \end{itemize}
\end{document}
