% Created 2020-03-06 Fri 23:57
% Intended LaTeX compiler: pdflatex
\documentclass[11pt]{article}
\usepackage[utf8]{inputenc}
\usepackage[T1]{fontenc}
\usepackage{graphicx}
\usepackage{grffile}
\usepackage{longtable}
\usepackage{wrapfig}
\usepackage{rotating}
\usepackage[normalem]{ulem}
\usepackage{amsmath}
\usepackage{textcomp}
\usepackage{amssymb}
\usepackage{capt-of}
\usepackage{hyperref}
\author{Tan Yee Jian}
\date{\today}
\title{MA2108S Week 7 Assignment}
\hypersetup{
 pdfauthor={Tan Yee Jian},
 pdftitle={MA2108S Week 7 Assignment},
 pdfkeywords={},
 pdfsubject={},
 pdfcreator={Emacs 27.0.90 (Org mode 9.4)}, 
 pdflang={English}}
\begin{document}

\maketitle
\begin{enumerate}
\item Since \textbf{(b)} would imply the forward direction in \textbf{(a)}, I will first prove \textbf{(b)}.

\textbf{(b)} We want to show that \(U\) is open and \(U \subseteq E\).

Since, by definition, \(\overline{(E')}\) is closed, \(U = \overline{(E')}'\)
is open.

Now we are left to show that \(U \subseteq E\). By contradiction, for any
\(x \in U\), we assume that \(x \notin E\). Then
\[
   x \in E' \implies x \in \overline{(E')} = U'
   \]
A contradiction, as \(x\) cannot be in both \(U\) and \(U'\). Therefore \(U =
   \overline{(E')}'\). \(\square\)

(a) Let us first prove the forward direction.

(\(\implies\)) Since \(U\) is an open set, there must exist an \(r > 0\) such
that \(B[x,r] \subseteq U \subseteq \E\). \(\square\)

(\(\impliedby\)) Now we are done with the forward direction, we shall prove that for any
\(x\),
\[
   \exists r, B[x,r] \subseteq E \implies x \in U
   \].

We show the contrapositive, \(x \notin U \implies \forall r, B[x,r]
   \nsubseteq E\).

Then \(x \in U'\). Since \(U \subseteq E\) by the previous part, either
\(x \in (E - U)\) or \(x \in E'\). The latter case is obvious since if that
is true,
\[ x \in B[x,r] \notin E \] and clearly the ball is not fully contained by E.

Otherwise, we must have \(x \in E - U\). Since \(x \in U' = \overline{(E')}\) which is a
closed set, then either \(x \in E'\) (which is impossible), or \(x\) is a
cluster point of \(E'\).

By the definition of cluster point,
\[
   \forall r > 0, \exists y \in B[x,r] \quad \text{such that} \quad y \in E'
   \].

Thus some part of the ball must always be in \(E'\), which gives the result.
\(\square\)

(c) Given \(O\) is open, then by definition,
\[
   \forall x \in O, \exists r, B[x,r] \subseteq O \subseteq E
   \].

by 1(a), we have \(x\) must be in \(U\) as well, since every element of O
must be contained by \(U\), \(O \subseteq U\). \(\square\)

\item We first show a lemma:

\textbf{Lemma}: \(S, T\) are subsets of metric space \(<M, \rho>\). If
\(S \subseteq \overline{T}\), then \(\overline{S} \subseteq \overline{T}\).

\uline{Proof of Lemma:}

Clearly, if \(S\) is closed, then \(\overline{S} = \overline{T}\).
Otherwise, then we just consider whether the cluster points of \(S\) are in
\(\overline{T}\).
Suppose \(x\) is a cluster point of S, then by definition,
\[
   B[x, r] \cap S \neq \emptyset
   \]
and thus,
\[
   B[x, r] \cap T \supseteq S \neq \emptyset
   \]
\(x\) is a cluster point of T too, and thus is in \(\overline{T}\). \(\square\)

Now we shall proceed with the proof.

(\(\implies\)) We first have \(f(A) \subseteq \overline{f(A)}\). Then looking
at preimages of both sets,
\[
   A \subseteq f^{-1}(f(A)) \subseteq f^{-1}(\overline{f(A)})
   \]
And since \(f\) is continuous, \(\overline{f(A)}\) is closed \(\implies
   f^{-1}(\overline{f(A)})\) is closed.

Since we have \(A \subseteq f^{-1}(\overline{f(A)})\), where RHS is a closed set,
by the lemma, this implies
\[
   \overline{A} \subseteq f^{-1}(\overline{f(A)})
   \]
And taking the images of both sets, we have
 \[f(\overline{A}) \subseteq \overline{f(A)}\] \(\square\)

(\(\impliedby\)) Given that \[f(\overline{A}) \subseteq \overline{f(A)}\], we
wish to show that if we have a closed set \(V \subseteq M_2\), then
\(f^{-1}(V)\) is also closed.

By the assumption, we have
\[
   f(\overline{f^{-1}(V)}) \subseteq \overline{f(f^{-1}(V))}
   \]
and since \(f\) need not be injective,
\[
   f(\overline{f^{-1}(V)}) \subseteq \overline{f(f^{-1}(V))} \subseteq
   \overline{V} = V
   \]
since V is defined as closed.

We then have
\[
   f(\overline{f^{-1}(V)}) \subseteq V
   \]
which, considering their preimages,
\[
   \overline{f^{-1}(V)} \subseteq f^{-1}(V)
   \]

By definition, \(\overline{f^{-1}(V)} = (\{\text{cluster points of V}\} \cup f^{-1}(V))
   \subseteq f^{-1}(V)\),
which implies \(f^{-1}(V)\) contains all its cluster points. Thus
\(f^{-1}(V)\) is closed. \(\square\)

\item Set \(a\) such that \(a > f(x) + \epsilon\) for some \(\epsilon > 0\) and \(x
   \in f^{-1}(-\infty, a)\) naturally.

Then since \(f^{-1}(-\infty, a)\) is open,
\[
   \exists r > 0 \quad \text{such that} \quad B[x, r] \subseteq f^{-1}(-\infty, a)
   \]
Since it is known that \((x_n)^\infty_{n=1}\) converges, then given \(r\) as
defined above, there is an \(N \in \mathbb{N}\) such that for all \(n \geq
   N\),
\[
   \rho(x_n, x) < r \quad \forall n \geq \mathbb{N}
   \].

Since we now have
\[
   x_N, x_{N+1}, \dots \in B[x,r] \subseteq f^{-1}(-\infty, a)
   \]

and since \(f\) is upper-semicontinuous,
\[
   f(x_N), f(x_{N+1}), \dots < a
   \]

Consider \(M_k = sup\{f(x_N), f(x_{N+1}), \dots \} < a\). Thus,
\[
   \lim_{k \rightarrow \infty} M_k < a \implies
   \limsup_{k \rightarrow \infty} f(x_n) < a = f(x) + \epsilon
   \]

Now, for every epsilon given, we can find an \(a\), and thus \(N\) such that
the previous statement is true. Given by Exercise 2.2 Question 2 in
\emph{Goldberg},
\[
   \limsup_{k \rightarrow \infty} f(x_n) < f(x) + \epsilon
   \implies
   \limsup_{k \rightarrow \infty} f(x_n) < f(x)
   \]
\(\square\)
\end{enumerate}
\end{document}
