\documentclass[a4paper, 12pt, fleqn]{report}

\usepackage{hyperref}
\usepackage{titlesec}
\usepackage{amssymb}
\usepackage{relsize}
\usepackage{mathtools}
\usepackage{graphicx}
\usepackage{enumitem}
\usepackage{scrextend}
\usepackage{hyperref}
\usepackage[super]{nth}
\usepackage{setspace}
\usepackage{mathptmx}
\usepackage[T1]{fontenc}
\usepackage{titlesec}
\usepackage[style=authoryear]{biblatex}
\addbibresource{sip-report.bib}

% \titleformat{\chapter}
% [hang] % shape
% {\bf\Huge} % format
% {\thechapter} % label
% {1ex} % sep
% {}{} % before-code, after-code
% \titlespacing{\chapter}{0pt} % left
% {0pt}{2em} % above, below

\setlength{\mathindent}{2em}
\setlength{\parindent}{0pt}

\renewcommand{\contentsname}{Table of Contents}

\begin{document}

\hypersetup{
linktoc=all     %set to all if you want both sections and subsections linked
}

\pagenumbering{roman}
\begin{titlepage}
    \begin{center}
        \vspace*{1cm}

        \Large
        \textbf{Student Internship Programme (SIP) \\
        Final Project Report}

        \vspace{2cm}
        \large
        \textbf{at \\
        \ \\ Go-Jek Singapore Pte Ltd}
        
        \vspace{1cm}
        \normalsize
        \textbf{Reporting Period: \\
        \ \\ 05 2020 to 08 2020}

        \vspace{1cm}
        by \\
        \ \\ TAN YEE JIAN

        \vfill

        Department of Computer Science \\
        \ \\ School of Computing \\
        \ \\ National University of Singapore \\
        \ \\ 2019/2020
        \vspace{1cm}
    \end{center}
\end{titlepage}

\section*{}
\addcontentsline{toc}{chapter}{Title}
\thispagestyle{plain}
    \begin{center}

        \Large
        \textbf{Student Internship Programme (SIP) \\
        Final Project Report}

        \vspace{2cm}
        \large
        \textbf{at \\
        \ \\ Go-Jek Singapore Pte Ltd}

        \vspace{1cm}
        \normalsize
        \textbf{Reporting Period: \\
        \ \\ 05 2020 to 08 2020}

        \vspace{1cm}
        by \\
        \ \\ TAN YEE JIAN

        \vfill

        Department of Computer Science \\
        \ \\ School of Computing \\
        \ \\ National University of Singapore \\
        \ \\ 2019/2020

        \vspace{1cm}
        Project Title: \textbf{Intern, Machine Learning Engineer}\\
        Project ID: \textbf{SY192497610}\\
        Project Supervisor: \textbf{Dr. PRABAWA Adi Yoga Sidi}
        \vspace{1cm}
    \end{center}


\chapter*{Summary}
\addcontentsline{toc}{chapter}{Summary}


Data and features are the lifeblood of Machine Learning systems. However,
productionized Machine Learning systems are often independent and spend an
dispropotionate amount of effort on selecting features and cleaning data which
are impossible to port over to another project. Feast \footnote{Feast: an open
  source feature store for machine learning. \url{https://feast.dev}} (FEAture STore) is a
production feature store that allows teams to define, store, validate, discover
and serve features to models during both training and inference.
\newline

 Subject Descriptors:
\\ \begin{addmargin}[2em]{2em}
  Machine Learning \\
  Distributed Storage \\
  RESTful web services
\end{addmargin}
\ \\

 Keywords:
\\ \begin{addmargin}[2em]{2em}
  Machine Learning, Data Science, Feature Selection, Distributed Data Storage
\end{addmargin}
\ \\ 

 Implementation Software and Hardware:
\\ \begin{addmargin}[2em]{2em}
  Java (OpenJDK 11)\footnote{JDK 11 Documentation \url{https://docs.oracle.com/en/java/javase/11/}},
  Python 3\footnote{Python \url{https://www.python.org/}},
  Apache KAFKA\footnote{Apache Kafka \url{https://kafka.apache.org/}},
  Apache SPARK\footnote{Apache Spark \url{https://spark.apache.org/}},
  Google Cloud\footnote{Google Cloud \url{https://cloud.google.com/}}
\end{addmargin}


\chapter*{Acknowledgements}
\addcontentsline{toc}{chapter}{Acknowledgements}
\doublespacing


I would like to first and foremost thank \textbf{Go-Jek Singapore Pte Ltd} for
this internship opportunity, and the Engineering Leads of the \textbf{Data
  Science Platform} team: my interviewer \textbf{Yu-Xi Lim} and my manager
\textbf{Willem Pienaar} for this wonderful 3 months working in this inspiring
team. I would also like to thank \textbf{ShuShu Luo} and \textbf{Felicia Wu} for
the smooth and enjoyable HR journey.
\newline


In addition, I want to extend my gratitude to my teammates in the \textbf{Feast}
team for their inspiring work and nurturing presence for this new intern in the
team. Having the opportunity to work on the development of an open-source,
impactful system for Machine Learning engineers was eye-opening and really
satisfying. \newline


Last but not least, I would like to thank the the whole \textbf{Data Science
  Platform} team for the great 3 months of work-from-home internship, who
despite working remotely, made me feel that I am a part of the team.

\singlespacing

\tableofcontents


\newpage

\pagenumbering{arabic}

\chapter{Introduction}

\section{Background and Organisational Structure of Host Organisation}
Gojek\footnote{Gojek \url{https://gojek.com}} is Southeast Asia’s leading
on-demand, multi-service tech platform providing access to a wide range of
services including transport, payments, food delivery, logistics, and many more.
Founded in 2010 with providing solutions to Jakarta’s ever-present traffic
problems in mind, Gojek started as a call center with a fleet of only 20
motorcycle-taxi drivers (ojek). \newline

With the principle of using technology to improve the lives of users, the Gojek
app was launched in January 2015 for users in Indonesia to provide motorbike
ride-sharing (GoRide), delivery (GoSend), and shopping (GoMart) services. Today,
Gojek has transformed into a “Super App”: a one-stop platform with more than 20
services, connecting users with over 2 million registered driver-partners, and
500,000 GoFood merchants – with a total of more than 170 million total downloads
across the region. By providing them access to products and services across
multiple sectors, Gojek has helped – and continues to help – create more value
for society, improving efficiency and productivity, as well as boosting
financial inclusion (\cite{gojekhome}).
\newline

\section{Principal Activities of Host Organisation}
Gojek provides a variety of services through it's super app, including Ride
Hailing, Food Delivery, Digital Payments, Logistics and Livestyle Services. It's
reach of more than 100 million bookings every month across 4 countries presents
many interesting challenges and problems in the field of machine learning.
\newline

Machine learning at Gojek is used for many purposes, from Matchmaking to
Fraud Prevention, many day-to-day services are run on Machine Learning powered
systems.

\section{Training Programme within Host Organisation}
As an intern in the DSP team, I was immediately allocated to one of the few
project teams from the start of the internship, namely the Feast team. The
training I received during work do not have a fixed format, but take the form of
the following activities (non-exhaustive):

\begin{itemize}
  \item Working independently to investigate and fix a software bug
  \item Working in pairs for a new feature
  \item Working independently for a new feature
  \item Take part in meet-the-customer feedback sessions
  \item Respond to and resolve issues raised by users
  \item Take part in team milestone discussions
  \item Knowledge transfer/sharing by other teammates
\end{itemize}

\section{Position of Host Unit within Host Organisation}
I was an Intern, Machine Learning Engineer within the Data Science Platform
(DSP) team. DSP team is a team in Gojek that holds the mission of ``make machine
learning fast and simple for everyone''. The DSP team collaborates with data
scientists and product teams to develop innovative AI solutions, developing
end-to-end platforms to support production systems deployed within Gojek. As an
intern, I was treated as an equal in the team and took on tasks that were
challenging and rewarding to work on, and also worked on solving the technical
challenges on software systems that the team was maintaining. \newline

\chapter{Training Schedule and Assignments}

\emph{Clarification}: this section was added in retrospect, the tasks assigned
(thus the training) was related only to the team's release schedule and
determined at the start of an Agile sprint. No schedule was pre-determined in
the beginning of the internship.

\section{Training Schedule by Month for the Entire Training Period}
\begin{enumerate}
  \item Familiarization of Feast, and manage its deployment within Gojek
  \item New feature: Google BigQuery\footnote{BigQuery
    \url{https://cloud.google.com/bigquery/}} as a data source for features
  \item Maintainence of the BigQuery source feature, and development of Feast
    User Interface (UI)
\end{enumerate}

\section{Training Assignments Completed in \nth{1} Month}
The first month was spend familiarizing with Feast's deployment within Gojek. I
maintained the \texttt{feature-specifications} repository, which was a
Git\footnote{Git \url{https://git-scm.com/}} repository of
\texttt{YAML}\footnote{YAML: YAML Ain't Markup Language \url{https://yaml.org/}}
files describing the Features\footnote{Feast: Features
  \url{https://docs.feast.dev/user-guide/features}} in a certain Feature Set
\footnote{Feast: Feature Sets
  \url{https://docs.feast.dev/user-guide/feature-sets}}, which are some of the
most important concepts in Feast. These files will dictate the name of the
features (similar to columns in a Pandas Dataframe\footnote{pandas.DataFrame API
  documentation
  \url{https://pandas.pydata.org/pandas-docs/stable/reference/api/pandas.DataFrame.html}}),
the type of the data (for example \texttt{INT64, STRING}), and the source of the
data, usually from an Apache KAFKA Stream\footnote{Kafka Streams
  \url{https://kafka.apache.org/documentation/streams/}}. \newline

My tasks for the first month include: improving on the documentation Jupyter
notebook for Feast, writing Bash and Python scripts to validate the YAML files
in the \texttt{feature-specifications} repository during the Continuous
Integration stages, as well as adding static type checking via
Mypy\footnote{mypy \url{http://mypy-lang.org/}} to the Python Software
Development Kit (SDK) for Feast.

\section{Training Assignments Completed in \nth{2} Month}
Feast as a Feature Store accepts data source that provides data in stream (such
as an Apache KAFKA topic) or in batch (such as a Pandas Dataframe). Prior to the
new feature I worked on, batch data was usually loaded into Feast via the Python
SDK, which does not scale when the amount of data increases to the order of
millions of rows in a table. To improve on this, I packaged an Apache Beam tool
that creates data pipelines that read from Google's BigQuery data store into
Feast directly as a Docker\footnote{Docker \url{https://www.docker.com/}} image,
and deployed in Gojek's internal Apache Airflow\footnote{Apache Airflow
  \url{https://airflow.apache.org/}} instance. This task solves the scaling
problem as it leverages Google Cloud's Dataflow\footnote{Dataflow
  \url{https://cloud.google.com/dataflow/}} pipelines, which can spawn as many
worker as needed, thus horizontally scalable. \newline

This was a challenging task as I am not only responsible for a small part of a
tool, but in charge of the end-to-end user experience of the BigQuery tool, from
a user-provided BigQuery SQL query string to scheduling the loading of the
specified data into Feast automatically. I had the chance to use Docker as a
containerization solution to package the tool into an executable image, as well
as designing user and machine interface to the BigQuery tool, and liasing with
various Google Cloud products such as BigQuery, Dataflow, Storage\footnote{Cloud
  Storage \url{https://cloud.google.com/storage}}, and IAM\footnote{Cloud
  Identity and Access Management \url{https://cloud.google.com/iam}}. In the
end, I delivered a system that is well documented, optimized for quick
iterations, also transparent but does not overwhelm the users by exposing too
many internal details.

\section{Training Assignments Completed in \nth{3} Month}
The end of July 2020 sees the release of Feast 0.6, which includes the internal
release of the BigQuery Tool to scale batch data sources. As the last minor
release (Feast 0.5) was during the first weeks of my internship, I did not
participate much on that minor release, but as one of the feature contributors
in Feast 0.6, I was involved in the production release of the new Feast version
and the various quality assurance tasks that follow the minor release, alongside
with an internal showcase on the new features of Feast 0.6 that is well received
by over a hundred employees and managers from the various data science teams.
\newline

Shortly after the release, I was involved the milestone planning for Feast v0.7,
after having understood more of the system and its intricacies. Our team held
meetings with users of Feast (other data science teams within Gojek) that worked
on different type of Machine Learning problems, and took in their suggestions to
select the features and improvements for our next minor release. \newline

As Feast starts to take shape in its core functionality, our team decides that
it is time for us to have a Web-based Graphical User Interface for a more
intuitive interaction and visualization of the Feast system. This feature was
also requested by various data science teams to achieve the goal of feature
sharing across teams. I was tasked to write the REST\footnote{Representational
  state transfer.
  \url{https://en.wikipedia.org/wiki/Representational_state_transfer}} HTTP
endpoints for the Feast UI, as the current Feast system uses gRPC\footnote{gRPC
  \url{https://grpc.io/}} for interaction. I worked with Java Spring
Boot\footnote{Spring Boot \url{https://spring.io/projects/spring-boot}} for the
HTTP endpoints, as well as writing holistic integration tests for it. \newline

At the meantime, I also had to maintain the BigQuery source for Feast by writing
guides and answering users' questions on the system, as well as solving
authorization issues and format violations on the use of the system that arises
with the increasing adoption of the tool, as it provides a centralized, one-off
way to query from Google BigQuery and carry out scheduled data ingestion into
Feast.

\chapter{Knowledge and Experience Gained}

\section{Technical Knowledge Gained from Assignments}
\begin{itemize}
  \item Gitlab CI, Python and Bash scripting for CI processes. Writing
    command-line programs with that are short, self-contained, extensible and
    readable.
  \item Introductory Google Cloud products (BigQuery, Storage, Compute,
    Dataflow, IAM) that are related to Data Engineering processes.
  \item Java Spring Boot as a MVC \footnote{Model-view-controller} framework as
    well as related libraries/frameworks such as gRPC, RestAssured testing
    library.
  \item Containerization of tools using Docker for reproducibility, and its
    interaction with workflow management platforms such as Apache Airflow.
  \item Data pipeline tools such as Apache KAFKA, Apache Beam and Google Cloud
    Dataflow, and how they interact with each other to form a data pipeline.
\end{itemize}

\section{Organisational/Industry Experience Gained from Assignments}
Full software development lifecycle using the Agile methodology. Even though I
was an intern, I participated from planning to release and maintainence of the
software systems, putting theoretical software engineering knowledge into
practice.

Production-level machine learning systems that are more than just Jupyter
notebooks and Pandas dataframe. I also experienced first hand the job of
solving problems for data scientists and data engineers, by creating and
maintaining scalable systems that are extensible.

In terms of soft skills, I had the chance to experience a work-from-home
setting, having to document decisions and instructions in detail as there are no
more shoulders to tap on for answers. Preparing for internal presentations,
liasing with open source contributors and putting their contribution to good use
was eye-opening.

\section{Areas of Applicability of Knowledge and Experience Gained}
Almost every modern software system today would benefit from knowledge on Continuous
Integration, scripting and cloud deployment workflows. Many mature companies
that look for Machine Learning solutions could benefit from experience with the
Machine Learning lifecycle, having written software on parts of it and
implementing solutions with the team.

\chapter{Conclusions}

\section{Summary of Work Completed and Training Received}
I maintained the source of truth for Feast deployment within Gojek, and wrote CI scripts to
validate specifications and manage the repository of features, as well as
generating pipeline specification for Apache Airflow for scheduling jobs
automatically with each merged Merge Request.
\newline

I also packaged and shipped a tool that brings features from Google BigQuery to
Feast with just a query SQL, automating everything in between with scripts, and
readied the tool for productionized usage.
\newline

Last but not least, I wrote HTTP REST endpoints that was originally gRPC
endpoints in preparation for a Web-based Graphical User Interface.

\section{Problems Faced}
Machine learning at the industry level is very different from that of academic
or research purpose, and the tools related to it (Apache KAFKA, Apache Beam,
Google BigQuery, Redis) are vastly different from the typical Jupter Notebook,
Scikit-Learn \footnote{scikit-learn
  \url{https://scikit-learn.org/stable/index.html}} systems that we develop for
projects. As such, there were many new frameworks and concepts to learn that can
be very challenging. \newline

Since hhe whole internship was work-from-home, it posed a serious requirement in
communicating ideas with the rest of the team, in an asynchronous and detailed
manner, which is new. With everyone focusing on their own work, it is easy to
be left behind if someone does not voice up any difficulties or ask for help
because of the lack of convenience to do so, as compared to in a face-to-face
setting.

\section{Assessment of Training Experience and Concluding Remarks}
I am glad to have the chance to work in this project that is highly impactful,
novel and relevant to the machine learning community, especially that Feast is
open-source, which makes me feel more proud of the team's contribution to the
betterment of the society through technology. I am very grateful for the
multitude of opportunities I have received during this internship and I would
really love to work for DSP again.

\printbibliography
\end{document}
