% Created 2020-03-11 Wed 09:46
% Intended LaTeX compiler: pdflatex
\documentclass[11pt]{article}
\usepackage[utf8]{inputenc}
\usepackage[T1]{fontenc}
\usepackage{graphicx}
\usepackage{grffile}
\usepackage{longtable}
\usepackage{wrapfig}
\usepackage{rotating}
\usepackage[normalem]{ulem}
\usepackage{amsmath}
\usepackage{textcomp}
\usepackage{amssymb}
\usepackage{capt-of}
\usepackage{hyperref}
\newtheorem{theorem}{Theorem}[section]
\newtheorem{corollary}{Corollary}[theorem]
\newtheorem{lemma}{Lemma}[theorem]
\newtheorem{note}{Note}[theorem]
\newtheorem{definition}{Definition}[section]
\newtheorem{ex}{Example}[section]
\newtheorem{observation}{Observation}[section]
\author{Tan Yee Jian}
\date{\today}
\title{ST2131 Mathematical Statistics}
\hypersetup{
 pdfauthor={Tan Yee Jian},
 pdftitle={ST2131 Mathematical Statistics},
 pdfkeywords={},
 pdfsubject={},
 pdfcreator={Emacs 27.0.90 (Org mode 9.4)}, 
 pdflang={English}}
\begin{document}

\maketitle
\tableofcontents


\section{Chapter 4 Parameter Estimation}
\label{sec:org961607a}
\subsection{Standard Error}
\label{sec:org56fe284}
SE = SD(sample mean)
\section{Chapter 6 Hypothesis Testing}
\label{sec:orgced21f8}


\subsection{The Neymann-Pearson Paradigm}
\label{sec:org65f1871}
\begin{definition}[Statistical Hypothesis]

\uline{Statistical hypothesis} is an assertion/conjecture about the distribution of
one or more random RVs.

\uline{Simple hypothesis}: a SH that completely specifies the distribution

\uline{Complex hypothesis}: otherwise
\end{definition}

\begin{definition}[Null & Alternative Hypotheses]
When deciding which of two hypothesis is true, the \textbf{first} is called the \uline{null
hypothesis \(H_0\)}, and the \textbf{other}, \uline{alternative hypothesis \(H_A\) or \(H_1\)}

The decision rule is based on a \uline{test statistic}.
\end{definition}

\begin{definition}[Type I & Type II errors]
The decision rule has typically 2 possible conclusions: \textbf{reject, or do not
reject} \(H_0\).

\uline{Type I error:} rejecting \(H_0\) when it is true. The probability of this is
called \uline{significance level} of the test, \(\alpha\).

\uline{Type II error:} \textbf{accepting} (do not reject) \(H_0\) when it is false. Probability
of this is \(\beta\). \uline{Power of the test} is probability of \textbf{rejecting} \(H_0\)
when it is false, \(1-\beta\).
\end{definition}

\begin{ex}[Egg Tarts - Normal]
Egg tarts weigh, in grams \(N(40, 2^2)\) when made by chefs, but \(N(43, 2^2)\)
when made by a trainee. Given the weight of tarts, is the trainee working today?
\end{ex}
\end{document}
