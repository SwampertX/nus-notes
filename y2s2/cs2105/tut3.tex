% Created 2020-02-19 Wed 14:20
% Intended LaTeX compiler: pdflatex
\documentclass[11pt]{article}
\usepackage[utf8]{inputenc}
\usepackage[T1]{fontenc}
\usepackage{graphicx}
\usepackage{grffile}
\usepackage{longtable}
\usepackage{wrapfig}
\usepackage{rotating}
\usepackage[normalem]{ulem}
\usepackage{amsmath}
\usepackage{textcomp}
\usepackage{amssymb}
\usepackage{capt-of}
\usepackage{hyperref}
\author{Tan Yee Jian}
\date{\today}
\title{CS2105 Tutorial 3}
\hypersetup{
 pdfauthor={Tan Yee Jian},
 pdftitle={CS2105 Tutorial 3},
 pdfkeywords={},
 pdfsubject={},
 pdfcreator={Emacs 26.3 (Org mode 9.4)}, 
 pdflang={English}}
\begin{document}

\maketitle
\tableofcontents

\begin{enumerate}
\item Network diagnostic tool
\begin{enumerate}
\item \texttt{POST}
\item \texttt{POST} sends data to server, while \texttt{GET} fetches data from server
\end{enumerate}
\item Not sure
\item TCP needs to establish connection with server before sending. \textbf{TCPEchoClient}
will end because it cannot create a socket without a receiving server.

On the other hand, \textbf{UDPEchoClient} just fires datagrams mindlessly and when
the server comes online, response will be received from the server.
\item The Datagram headers contain the source IP and ports, so the server can distinguish
the datagrams from different hosts that way.
\item \texttt{01011100} +
\texttt{01100101} =
\texttt{11000001} has 1s complement \texttt{00111110}

\texttt{11011010} +
\texttt{01100101} =
\texttt{100111111} overflows to \texttt{01000000} has 1s complement \texttt{10111111}
\item No. It can correspond to overflown values too, where \texttt{01000000} can be a sum
of the 2 numbers as in question 5 part 2, it can also be the sum of
\texttt{00000000} and \texttt{01000000} which are corrupted inputs.
\item We can accidentally ack the wrong package due to delay in receiving the message.
\end{enumerate}
\end{document}
