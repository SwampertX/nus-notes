% Created 2020-05-01 Fri 20:09
% Intended LaTeX compiler: pdflatex
\documentclass[11pt]{article}
\usepackage[utf8]{inputenc}
\usepackage[T1]{fontenc}
\usepackage{graphicx}
\usepackage{grffile}
\usepackage{longtable}
\usepackage{wrapfig}
\usepackage{rotating}
\usepackage[normalem]{ulem}
\usepackage{amsmath}
\usepackage{textcomp}
\usepackage{amssymb}
\usepackage{capt-of}
\usepackage{hyperref}
\usepackage{amsthm}
\usepackage[citestyle=numeric,bibstyle=apa, hyperref=true,backref=true,maxcitenames=3,url=true,backend=biber,natbib=true] {biblatex}
\addbibresource{final.bib}
\author{Tan Yee Jian}
\date{\today}
\title{UPI2211 Final Paper}
\hypersetup{
 pdfauthor={Tan Yee Jian},
 pdftitle={UPI2211 Final Paper},
 pdfkeywords={},
 pdfsubject={},
 pdfcreator={Emacs 28.0.50 (Org mode 9.4)}, 
 pdflang={English}}
\begin{document}

\maketitle
\tableofcontents

\section{Introduction}
\label{sec:org15df530}
In the Smart Nation Launch Speech in 2014 \cite{lhl2014_smart_nation}, Singapore
Prime Minister Lee Hsien Loong announced the Smart Nation initiative to make
Singapore into ``a nation where people live meaningful and fulfilled lives,
enabled seamlessly by technology.'' In his speech, he listed numerous ways where
technology is deployed to create a better living environment for all
Singaporeans, from taking care of the elderly using an alert and monitoring
system to telemedicine which delivers healthcare to the homes, portraying a
Utopia full of technological solutions that are promised for Singaporeans. In an
era of rapid technological advancements, where everyone is connected virtually,
where modes of work and play are revolutionized by artificial intelligence, the
push for Smart Nation seems apt. In this initiative, Singapore goes further to
embrace technology as the deciding factor for Singapore's future and declare it
the path for Singapore to be a Utopia for its people.

However, having technology-based solutions to daily problems, and imagining a
Smart Nation as a Utopia for all Singaporeans are very different. As with other
Utopias, the Smart Nation initiative aims to create an ideal life in every
aspect for Singaporeans by using ``smart'' solutions, which are technological
solutions. In Lee's Speech \cite{lhl2014_smart_nation}, he says:

\begin{quote}
We want to transform our lives for the better, and we have what it takes to
achieve this vision – the capabilities and the daring to pull it all together
and make a quantum leap forward. I am looking forward to living in a Smart
Nation – better living for all of us; stronger communities in our society; and
more opportunities for all.
\end{quote}

where he promises to carry out the Smart Nation initiative to form a better
society for the whole nation, deeming it as a Utopia. However, promising Smart
Nation as a Utopia brings about its dangers, as this inherently implies that
technological solutions are always beneficial and empowering, and as much as
Utopias are societies which we might aspire to work towards, they carry their
flaws as literary texts such as Plato's Republic and Skinner's Walden Two
might reveal.

In this paper, I argue that Singapore's Smart Nation initiative has its gaps
from being a Utopia, and there are threats by equating them. I will highlight
the threats of equating Singapore's Smart Nation initiative as a Utopia by
showing the inadequacy of technology as a panacea, the potential loss of
democracy under the Smart Nation initiative, and lastly, the gaps between the
Smart Nation Singapore envisioned and a Utopia.

\section{Technology is not a silver bullet}
\label{sec:org123f904}
First of all, there is no denying that technology often improves lives for the
better. For example, Smart Nation deploys a series of technological solutions to
everyday living in Singapore, from more well-known solutions such as e-payment
platforms PayLah! and PayNow, to the less-known Moments of Life (MOL)
applications targeting families and aging population \cite{smart_nation}. As
Singaporeans try out these solutions for themselves, we naturally wonder whether
tech solutions are truly the only ways to solve problems. However, as the
Smart Nation initiative is thoroughly focused on creating ``Digital Economy,
Digital Government and Digital Society'', sometimes the most glorious solution
does not yield the best results. When we look at digital technology as a hammer,
then everything else we see will look like nails - we will try to force our
solutions onto whatever problems we come across and omit the wealth of other
more effective, alternative solutions.

Technology might give us a worse solution or even exacerbate the existing
problem. More importantly, technology is usually a useful tool for us to
implement a solution to problems developed by the humanities. These problems
often arise when we frame Smart Nation as a Utopia, the best society that we
come across, which implies adopting technology as the only solution.

\subsection{Low-tech, better solutions}
\label{sec:org04f5ac2}
Using the right technology to tackle a focused problem can be very effective,
but there is always a tendency to over-complicate high-tech solutions. If we
keep the mindset where Smart Nation is a Utopia and we must digitize as much as
possible, we might end up missing out on better solutions that are more
straightforward in many cases. For example, when Cinnamon College's dining hall
had to restrict the number of users to 50, students first suggested a
webcam-based artificial intelligence system to count the number of magnetic
buttons left on the board and transmit this data via the internet to the
students. As later realized, reducing the number of chairs in the dining hall
to 50 achieved the same effect with a higher rate of compliance than the digital
solution - if the students are not going to squat for the entirety of their
meals, there will always be less than 50 students eating in the dining hall at
the same time.
\subsection{Technology might exacerbate the problem}
\label{sec:orgeaee021}
Not every problem in the world can be solved by merely building an app,
especially when the app is the root cause of the problem. For example,
hackathons sometimes target a certain underprivileged community for the
participants to develop applications for, such as the elderly or the autistic
community. However, as the elderly are new to technological solutions, they
might not benefit from the application developed at all as their living
condition possibly cannot afford a smartphone or stable internet connection.
Applications developed for special needs children might make them more reliant
on the applications' content and overlook the intention of helping them to
integrate into society or learn valuable skills, instead of delving into the
virtual world the application has created. This example demonstrates that not
all problems can be solved by technological solutions, much less framing Smart
Nation as a Utopia and hence exaggerating the effectiveness of such solutions.
\subsection{Technology implementations depend on solid humanities solutions}
\label{sec:org7944843}
The predictive power of machine learning models is undeniable, as we have
increasingly accurate image recognition skills built into the cameras of our
smartphones, and spot-on search results and advertisements on the applications
that we use. However, the results would be far from Utopian if the data
predicted is not used to society's benefit. This is harder than it sounds,
since the machines learn essentially human-labeled data, and extrapolates them
based on our existing biases. For example, using machine learning to pile
through data of applicants of any kind makes sense - these jobs are often
time-consuming and tedious, and machines excel at doing such repetitive work.
However, it needs a sample of who are the people usually accepted, and it might
extrapolate the data from there \cite{racist_robots}.

For example, applicants for a job will provide their personal background
information, from education to family background, and their living areas.
However, a bad model that filters such applications may be trained to look for
some common correlation between rejected applicants, such as being of a certain
race or living under the same postcode. This creates a problem, where people
from a poor neighborhood lacking proper education will be discriminated in
getting the job, or merely because of the race of their neighbors. Furthermore,
predictive analytics may even create a self-fulfilling prophecy.

In Cathy O'Neil's book, \emph{Weapons of Math Destruction} \cite{wmd} how mathematical
models may go wrong, a crime prediction system, \emph{PredPol} was used to forecast
the number of crime happenings in a neighborhood in Reading, Pennsylvania
\cite{lum16_to_predic_serve}. The bias happens when the law enforcement team begin
more patrols in the neighborhood - they catch more people committing petty
crimes such as littering or causing a nuisance, hence making the predictive model
accurate, and more importantly, they are missing out on the unpredictable,
higher risk crime due to the effective prediction system which generates a
metric which they now can never omit. \cite{wmd}

In this case, technology can only aid our effort in preventing crime or
increasing efficiency in selecting applicants, but a lot of human intervention
and understanding of the intricacies of the model is essential for the effective
execution of these objectives. These problems can likely be prevented if we do
not over-rely on the technological, ``smart'' part of the solutions, and fall back
to our logical mind to investigate or improve the system before deploying it,
rather than worshipping such solutions as the panacea for building a city and
making all its decisions right away. Framing smart cities as a Utopia dismisses
the notion of other solutions being valuable and can lead to harmful outcomes.
\section{The loss of democracy}
\label{sec:org7e1ad13}
Lying at the core of smart urbanism is data. The generation of data relies on
digitization of real-world information. From our movements to personal identity,
to the temperature and crowdedness of the surroundings, all the data with what
is happening is being recorded down with the help of new technology.These data
are the lifeblood of smart solutions, where these numbers are crunched and
analyzed to make decisions \cite{kong18_ideol_align_smart_urban_singap}. This can
decide from things as small as automatically change the temperature of air
conditioner, inform someone with the congestion situation of a certain
expressway, to suggesting you what to purchase and prompting for your decisions,
to filling your news feed with targeted news, messages, and advertisements.

The loss of democracy begins with the stakeholder who owns the most data - they
get to decide with the greatest level of accuracy and have the most opportunity
to affect someone's life. Huge tech companies such as Google and Facebook hold
most of our data, as we post any photos on Instagram, comment on a post on
Facebook, or link to a service via Google accounts. This prompts the creation of
an online reenactment of the physical world, and from the various apps and services
linked to a central account, a huge number of deductions can already be made by
the company. However, data collection will only get more pervasive in the case
of Smart Nation, as not only virtual data is being collected, we are submitting
our real-life data, such as consuming pattern at supermarkets, daily movement
trajectories, even to preference in food choices via location stamps to the
``digitization'' of our society.

Even today, Singapore is already deemed as a data-controlled society.
\cite{Helbing2019} Having data consolidated at the hands of any party will give
them immense power to do the ``big nudge'' - nudging people's decisions using big
data. The pervasiveness of a single search engine can cause numerous people to
consume information from the same source, that predominate a country's thoughts,
even including the lawmakers. Most importantly, this directly threatens the
plurality and diversity of the people. This can be viewed as a kind of
conditioning, similar to that in Walden Two, where the people are condition to
get rid of hateful emotions such as anger and jealousy \cite{walden_two},
consolidating data to aid a city's decision-making process will directly
sacrifice the already limited diversity of Singapore.

\section{Is a perfect Smart Nation a Utopia, already?}
\label{sec:org5a0a762}
After 5 years of the announcement of the Smart Nation initiative, Singapore is
far from fully implementing smart solutions for the whole nation yet. It is
difficult as misuses and lack of uptake are prevalent in the community when new
technological solutions and measures are first announced. Examples are the
elderly emergency button and the mobile payment solution. When the elderly
emergency button was first introduced, the elderly were unaccepting of the new
gadget, some might not use it and deem it as troublesome. And the receptionists
are stressed due to the responsibility to immediately respond whenever the panic
button is sounded by any elderly \cite{kong18_ideol_align_smart_urban_singap}.
Similarly to the mobile payment system, businesses, especially smaller ones are
less keen to take up the new system due to the inherent complexity, considering
many food stalls and a good number of people dealing with payments are those who
are not tech-savvy, which include the older generation \cite{elderly_shun_tech}.
This gap in implementation is obvious in any creation of new Utopian community,
for example, in Plato's Republic, the creation of Kallipolis requires the
important ``noble lie'' for its citizen to work and live together for the land,
and protect it from any external invaders \cite{republic}.

A series of smart solutions will look good on any metric, but when it comes to
solving the root causes of problems, it might still be lagging. Singapore is
facing an aging population, and there has been a multitude of apps and gadgets
targeted to solve them, from teaching them to use instant messaging or video
calls, or installing panic buttons, but that still does not solve the problem of
isolation for the elderly \cite{kong18_ideol_align_smart_urban_singap}. Many crave
human interaction, and shoving them with new applications and smartphones will
make them feel like the environment is hostile to them since it put more stress
on them and does not address the root cause of solidarity.

Having smart solutions will indeed increase our efficiency, but the inherent
convenience might make us take the daily chores which forge our interpersonal
connections for granted. For example, tasks that we used to have to talk to
people in front of the counters to get done are now replaced by automated
applications and online procedures, and we lose the ability to connect,
understand and emphasize with one another, with different professions this way.
As Jared from the MCCY mentioned, in setting their tasks to onboard new citizens
into the Singaporean community, they often ``make some of the tasks difficult''
to prompt the participants to take initiative in finishing the tasks themselves,
such as leaving individual groups to create chat groups on \emph{WhatsApp} on their
own. This way, Jared argues, can make them more involved in the process of
bonding and hence make their orientation process more memorable and effective. I
agree with this argument since it connects with the thought that a good life is
not necessarily one where everyone is given what they wanted, but one where
their effort is valued and rewarded
\cite{fouriezos90_task_diffic_increas_thres_rewar_brain_stimul}.

\section{Conclusion}
\label{sec:orge7c1f21}
In all, technology is not a bad solution for many modern problems. For many
problems, technology equips us with an extra tool to effectively deliver our
solution with, especially with big data and artificial intelligence, but these
come with their hidden danger as well, when technology is deemed as the only
best solution to create a Utopia. To retain our democracy and free will
in decision making, extensive humanities research have to be carried out so that
a Smart Nation can make the right policies to benefit its people.
\printbibliography
\end{document}
